\documentclass[12pt]{article}
\title{M374M Homework 8 \\
  \normalsize{\S~3.1 \#8a, \S~3.2 1$^1$ac} \\
  Revision: \input{revision}}
\author{Hershal Bhave (hb6279)}
\date{Due 2016--04--01}

\usepackage{homework-macros}
\usepackage{xparse}
\tikzexternalize

\begin{document}
\maketitle

\section{\S~3.1}
\subsection{8a}
\subsubsection*{Problem}
\subsubsection*{Solution}

\section{\S~3.2}
\subsection{1$^1$ac}
\subsubsection*{Problem}
\subsubsection*{Solution}
\subsubsection*{Remarks}
Find the leading-order term in expansion of each root.

\section{Programming Minilab}
One of the first successes of the theory of relativity was to predict the
anomalous behavior of the orbit of Mercury (M) around the Sun (S). According to
the relativistic theory, the orbital curve $r(\theta)$ satisfies
\cref{eq:minilab-problem} where $u(\theta)=1/r(\theta)$ is the inverse of the
radius, $\rho$ is a constant determined by the angular momentum of the system,
$\gamma$ is a constant which quantifies relativistic effects, and $c$ is a
constant which defined the initial radius. Here we study
\cref{eq:minilab-problem} in the case corresponding to Newton's theory
($\gamma=0$) and the case corresponding to Einstein's theory
($0<\gamma\ll\rho^2$). Relevant physical dimensions are $[\rho]=L$,
$[\gamma]=L^2$ and $[c]=1$. We assume $\rho>0$ and $0<c<1$.

\begin{equation}
  \label{eq:minilab-problem}
  \begin{aligned}
    &\od[2]{u}{\theta} + u = \frac{1}{\rho} + \frac{\gamma u^2}{\rho},&\quad
    \theta&\ge\theta_0 \\
    &\od{u}{\theta}(\theta_0) = 0, &\quad u(\theta_0)&=\frac{1+c}{\rho} \\
  \end{aligned}
\end{equation}

\subsubsection*{Problem}
\begin{enumerate}[(a)]
\item Introduce the dimensionless variable $v=\rho u$ and shifted angle
  $\phi=\theta-\theta_0$ and show that \cref{eq:minilab-problem} can be written
  in the following form for an appropriate parameter $\epsilon$:
  \begin{equation}
    \label{eq:minilab-problem-part-a}
    \od[2]{v}{\phi} + v = 1 + \epsilon v^2,\quad \phi\ge0,\quad
    \od{v}{\phi}(0)=0,\quad v(0)=1+c
  \end{equation}
\item\label{itm:minilab-b} Solve \cref{eq:minilab-problem-part-a} in the
  Newtonian case when $\epsilon=0$. Using the fact that
  $r(\theta)=\rho/v(\theta-\theta_0)$, show that the smallest value of $r$ (the
  perihelion of the orbit) occurs at the angle $\theta_p=\theta_0+2n\pi$, and
  that the largest value of $r$ (the aphelion of the orbit) occurs at the angle
  $\theta_0=\theta_p+\pi$. Does Newton's theory predict that the location of the
  perihelion/aphelion change with each cycle of the orbit?
\item\label{itm:minilab-c} Approximate \cref{eq:minilab-problem-part-a} in the
  relativistic case when $0<\epsilon\ll1$. Use the Poincar\'{e}-Lindsted method
  to develop at two-term approximation of the form
  $$v(\phi)=v_0(s)+\epsilon v_1(s),\quad s=(\omega_0+\epsilon\omega_1)\phi.$$
  You need only determine $v_0$, $\omega_0$, and $\omega_1$. Using the
  approximation $r(\theta)=\rho/v_0(s)$, where
  $s=(\omega_0+\epsilon\omega_1)\phi$ and $\phi=\theta-\theta_0$, show that the
  smallest value of $r$ now occurs at
  $\theta_p=\theta_0+2n\pi(1+\frac{\gamma}{\rho^2-\gamma})$, and that the
  largest value of $r$ occurs at
  $\theta_a=\theta_p+\pi(1+\frac{\gamma}{\rho^2-\gamma})$. Does Einstein's
  theory predict that the location of the perihelion/aphelion change each cycle
  of the orbit?
\item Simulate the equations in \cref{eq:minilab-problem} for
  $\theta_0=\frac{\pi}{4}$, $\phi=1$, and $c=0.7$. For each of the cases
  $\gamma=0$, and $\gamma=0.01$, make a plot of the orbital curve
  $(x,y)=(r(\theta\cos\theta),r(\theta)\sin\theta)$ for
  $\theta\in[\theta_0,\,\theta_0+2n\pi]$  for about $n=10$ cycles. Do the
  simulations agree with your analysis in \cref{itm:minilab-b,itm:minilab-c}?
  Astronomical observations show that the perihelion/aphelion of Murcury advance
  with each sysle of the orbit in agreement with the relativistic theory.
\end{enumerate}

\subsubsection*{Solution}
\todo{}
\end{document}