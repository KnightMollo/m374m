\documentclass[12pt,twoside]{article}
\title{M374M Homework 11 \\
  \normalsize{\S~4.4 1a$^2$, 5bc, 6$^2$, 7} \\
  Revision: \input{revision}}
\author{Hershal Bhave (hb6279)}
\date{Due 2016--04--29}

\usepackage{homework-macros}
\tikzexternalize%

\begin{document}
\maketitle

\section{\S~4.4}
\subsection{1a}
\subsubsection*{Problem}
Find extremals for the functional
\begin{equation}
  \label{eq:1a-problem}
  \begin{aligned}
    J(y_1, y_2)=\int_0^{\pi/4}[4y_1^2+y_2^2+y_1'y_2']\dd{x}, \\
    y_1(0)=0, \quad y_1(\pi/4)=0,\quad y_2(0)=0,\quad y_2(\pi/4)=1. \\
  \end{aligned}
\end{equation}

\subsubsection*{Solution}
\todo{}

\subsection{5b}
\subsubsection*{Problem}
Find the extremal for
\begin{equation}
  \label{eq:5a-problem}
  \begin{aligned}
    J(y) = \int_0^3 e^{2x}(y'^2-y^2)\dd{x}, \\
    y(0)=1, y(3)=\text{free}. \\
  \end{aligned}
\end{equation}

\subsubsection*{Solution}
Let
\begin{align*}
  L(x,y,y') &= e^{2 x} \left({\left(y'\right)}^2-y^2\right) \\
  L_y(x,y,y') &= -2 e^{2 x} y \\
  L_{y'}(x,y,y') &= 2 e^{2 x} y' \\
\end{align*}
Invoking the Euler equation results in
\begin{equation}
\label{eq:5a-ode}
\begin{aligned}
  0 &= -2 e^{2 x} y - \od{}{x}2 e^{2 x} y' \\
    &= -2 e^{2 x} y - 4e^{2x}y' - 2e^{2x}y'' \\
  0 &= y + 2y' + y''
\end{aligned}
\end{equation}
Based on the non-free condition, this ODE may be solved by standard methods
\begin{equation*}
  \implies y(x) = e^{-x}(1+c_2x).
\end{equation*}
The remaining constant $c_2$ may be determined by natural boundary condition
\begin{align*}
  L_{y'}(b,y(b),y'(b)) &= 2 e^{6} y'(3) \\
  0 &= 2e^{6}y'(3) \\
  \implies y'(3) &= 3.
\end{align*}
We may re-solve \cref{eq:5a-ode} standard methods with this additional condition
\begin{equation*}
  \boxed{y(x) = -\frac{1}{2} e^{-x} (x-2).}
\end{equation*}

\subsection{5c}
\subsubsection*{Problem}
Find the extremal for
\begin{equation}
  \label{eq:5b-problem}
  \begin{aligned}
    J(y) = \int_0^1 y'^2/2+y'y+y'+y\;\dd{x}, \\
    y(0)=1/2, y(1)=\text{free}. \\
  \end{aligned}
\end{equation}

\subsubsection*{Solution}
Let
\begin{align*}
  L(x,y,y') &= y'^2/2+y'y+y'+y \\
  L_y(x,y,y') &= 1 + y' \\
  L_{y'}(x,y,y') &= 1 + y + y' \\
\end{align*}
Invoking the Euler equation results in
\begin{equation}
\label{eq:5b-ode}
\begin{aligned}
  0 &= 1 + y' - \od{}{x}(1 + y + y') \\
  &= -y''+1 \\
  \implies y'' &= 1 \\
\end{aligned}
\end{equation}
Based on the non-free condition, this ODE may be solved by standard methods
\begin{equation}
  \label{eq:5b-eq1}
  \implies y(x) = c_2 x+\frac{x^2}{2}+\frac{1}{2}
\end{equation}
The remaining constant $c_2$ may be determined by natural boundary condition
\begin{equation}
  \label{eq:5b-eq2}
  \begin{aligned}
    L_{y'}(b,y(b),y'(b)) &= 0 \\
    1 + y(1) + y'(1) &= 0 \\
  \end{aligned}
\end{equation}
We'll use \cref{eq:5b-eq1,eq:5b-eq2} to synthesize a condition to solve for the
remaining constant in \cref{eq:5b-eq1}. We'll get $y(1)$ from \cref{eq:5b-eq1}.
\begin{equation}
  \label{eq:5b-eq1-eval}
  \begin{aligned}
    y(x) &= c_2x + \frac{1}{2} + \frac{x^2}{2} \\
    y(1) &= c_2 + 1
  \end{aligned}
\end{equation}
and insert \cref{eq:5b-eq1} into \cref{eq:5b-eq2}
\begin{equation}
  \label{eq:5b-yp}
  \begin{aligned}
    0 &= y'(1) + y(1) + 1  + 1 \\
    0 &= y'(1) + c_2 + 2 \\
    y'(1) &= -c_2 - 2 \\
  \end{aligned}
\end{equation}
Deriving \cref{eq:5b-eq1-eval} and setting it equal to \cref{eq:5b-yp}.
\begin{equation*}
  \begin{aligned}
    c_2 + 1 &= -c_2 - 2 \\
    \implies c_2 &= -3/2 \\
  \end{aligned}
\end{equation*}
We may re-solve \cref{eq:5b-ode} by standard methods with this additional condition
\begin{equation*}
\boxed{y(x) = \frac{1}{2} \left(x^2-5 x+1\right).}
\end{equation*}

\subsection{6$^1$}
\subsubsection*{Problem}
Determine the natural boundary condition at $x=b$ for the variational problem
defined by
\begin{equation}
  \begin{aligned}
    J(y) = \int_a^b L(x,y,y')\dd{x}+G(y(b)), \\
    y\in C^2[a,b],\quad y(a)=y_0. \\
  \end{aligned}
\end{equation}
\subsubsection*{Remarks}
Determine the boundary-value problem that a local minimizer or maximizer must
satisfy.
\subsubsection*{Solution}
Let
\begin{equation*}
  J(y+\epsilon h) = \int_a^b L(x,y+\epsilon h, y'+\epsilon h')\dd{x} + G(y(b) + \epsilon h(b)).
\end{equation*}
Then
\begin{align*}
  \od{}{\epsilon} J(y+\epsilon h) &= \int_a^b\pd{L}{y}(x,y+\epsilon h, y'+\epsilon h') +
                                    \pd{L}{y'}(x,y+\epsilon h, y'+\epsilon h')h'\dd{x} +
                                    G'(y(b)+\epsilon h(b))h(b) \\
  \delta F(y,h) &= \int_a^b g h + f h'\dd{x} + G'(y(b)+\epsilon h(b))h(b) \\
  &= \int_a^b gh - f'h\dd{x} + {[fh]}_{x=a}^{x=b} + G'(y(b))h(b) \\
\end{align*}
Due to the properties of the normed vector space, $h(a)=0$ so that
\begin{align*}
  \delta F(y,h) &= \int_a^b gh - f'h\;\dd{x} + {[f]}_{x=b}h(b) + G'(y(b))h(b) \\
  0 &= \int_a^b gh - f'h\;\dd{x} + ({[f]}_{x=b} + G'(y(b)))h(b) \\
\end{align*}
Due to the fundamental lemma of the calculus of variations, $(g-f')=0$ and
${[f]}_{x=b} + G'(y(b))=0$. Thus, the natural boundary condition at $x=b$ is
\begin{equation*}
  {[f]}_{x=b} + G'(y(b))=0
\end{equation*}
or
\begin{equation*}
  \boxed{\pd{L}{y'}(b,y(b),y'(b)) + G'(y(b)) = 0 \quad y(a)=y_0.}
\end{equation*}

\subsection{7}
\subsubsection*{Problem}
Find the extremal{(s)} of the functional
\begin{equation}
  \label{eq:7-problem}
  J(y)=\int_0^b(\sqrt{1-k^2+y'^2}-ky') \dd{x}
\end{equation}
in the class of all $C^2[0,b]$ functions with $y(0)=0$ and $y(b)$ free. Take
$0<k<1$.

\subsubsection*{Solution}
\todo{}

\section{Mini-lab}

\end{document}
