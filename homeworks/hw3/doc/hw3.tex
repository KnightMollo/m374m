\documentclass[12pt]{article}
\title{M374M Homework 3 \\
  \normalsize{\S~1.3 \#1a$^1$, 1n$^1$, 8a$^2$, 9, 13}}
\author{Hershal Bhave (hb6279)}
\date{Due 2016--02--15}

\usepackage{macros}

\begin{document}
\maketitle

\section{\S~1.3}
\subsection{1a$^1$, 1n$^1$}
\subsubsection*{Problem}
Find the general solution of the following differential equations:
\begin{enumerate}
\item $u'+2u=e^{-t}$
\item $u''+\omega^2u=\cos\omega t$
\end{enumerate}

\subsubsection*{Solution}
\todo[]

\subsection{8a$^2$}
\subsubsection*{Problem}
The following model contains a parameter $h$. Find the equilibria in terms of
$h$ and determine its Stablity. Construct a bifurcation diagram showing how
equilibria depend upon $h$, and label the brancehs of the curves as stable or
unstable.

\subsubsection*{Remarks}
Assume the parameter $h$ can take any value: negative, zero, or positive.

\subsubsection*{Solution}
\todo[]

\begin{equation}
  \label{eq:8a-problem}
  u'=hu-u^2
\end{equation}

\subsection{9}
\subsubsection*{Problem}
Consider the model
\begin{equation}
  \label{eq:9-problem}
  u'-(\lambda - b) u-au^3,
\end{equation}
where $a$ and $b$ are fixed positive constants and $\lambda$ is a parameter that
varies.

\begin{enumerate}
\item If $\lambda < b$  show that there is a single equilibrium and that it is
  asymptotically stable.
\item If $\lambda > b$ find all equilibria and determine their stability
\item Sketch the bifurcation diagram showing how equilibria vary with $\lambda$.
  Label each branch of the curves shown in the bifurcation diagram as stable or
  unstable.
\end{enumerate}

\subsubsection*{Solution}
\todo[]

\subsection{13}
\subsubsection*{Problem}
A one-dimensional system is governed by the dynamical equation
\begin{equation}
  \label{eq:13-problem}
  u'-4u(a-u)-he^{-u}
\end{equation}
where $a$ and $h$ are positive constants. Holding $h$ constant, draw a
bifurcation diagram with respect to the parameter $a$. Indicate the stable and
unstable branches.

\subsubsection*{Solution}
\todo[]

\newpage
\section{Programming Minilab}
A simple model for the population of plants in a plant-herbivore ecosystem is
\begin{equation}
  \label{eq:minilab-problem}
  \od{p}{t} = rp\left( 1-\frac{p}{k} \right) - \frac{aqp}{1+bp}, \quad p(0)=p_0
\end{equation}
Here $p(t)$ is the number of plants, $q$ is the constant number of herbivores,
$r$ and $k$ are constants that describe the growth rate of the plants, and $a$
and $b$ are constants that describe the consumption rate of the plants by the
herbivores. Here we perform a qualitative analysis to understand the behavior of
solutions of the model in \cref{eq:minilab-problem}. To quantify the population
sizes we introduce the dimensions $[p] = \text{Plant}$, $[q] =
\text{Herbivore}$, and $[t] = \text{Time}$. All constants are assumed positive.
\begin{enumerate}
\item Find the dimensions of $r$, $k$, $a$, and $b$. Using the scales $t_c =
  1/r$ and $p_c = k$, show that the dimensionless version of
  \cref{eq:minilab-problem} takes the form
  \begin{equation}
    \label{eq:minilab-problem-a}
    \od{u}{\tau}=u(1-u)-\frac{hu}{1+cu},\quad u(0)=u_0
  \end{equation}
  where $u=p/p_c$, $\tau=t/t_c$, and $h,c$ are constants which you should
  identify.
\item Assuming $c>1$ is fixed, find (or characterize) all equilibrium solutions
  of \cref{eq:minilab-problem-a} and determine their stability in terms of the
  parameter $h>0$. Illustrate the results on a bifurcation diagram.
\label{item:minilab-part-b}
\item Use \verb|program3.m| and \verb|plant.m| to numerically simulate the model
  in \cref{eq:minilab-problem-a}. Produce portraits of solutions for various
  $u_0$ when $c=4$ and a few different values of $h$. Do the simulations agree
  with the analysis in \cref{item:minilab-part-b}? According to the stability
  results in the bifurcation diagram, do solutions grow, decay, or remain
  constant?
\item For what range of the parameter $h$ would the plant population survive if
  $\tau\rightarrow\infty$, $c=4$, and $u(0)=1/4$? For what range of $h$, if any,
  would the plant populationn die out as $\tau\rightarrow\infty$?

\end{enumerate}
\end{document}
