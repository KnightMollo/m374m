\documentclass[12pt]{article}
\title{M374M Homework 1 \\
  \normalsize{\S~1.1 \#4$^1$, 5, 9$^2$, 13, 14$^3$}}
\author{Hershal Bhave (hb6279)}
\date{Due 2016--02--01}

\usepackage{macros}

\begin{document}
\maketitle

\section{\S~1.1}
\subsection{4$^1$}
  \subsubsection*{Problem}
  Assume a physical law modeling a blast wave from an atomic explosion in the
  form
  \begin{equation*}
    g(t,r,\rho,E,P)=0,
  \end{equation*}
  where $t$ is time, $r$ is the blast radius, $\rho$ is density of the
  atom\footnote{$ML^{-3}$}, $E$ is the energy released during the explosion, and
  $P$ is the ambient air pressure. By inspection, find two independent
  dimensionless parameters formed from $t$, $r$, $\rho$, $E$, and $P$. We will
  name the two dimensionless parameters $\pi_1$ and $\pi_2$ and assume the law
  is equivalent to
  \begin{equation*}
    f(\pi_1,\pi_2)=0.
  \end{equation*}
  Does it still follow that $r$ varies like the two-fifths power of $t$?

  \subsubsection*{Remarks}
  Find the reduced form of $g(t,r,\rho,E,P)=0$. In special case when $P=0$,
  deduce that $r=C{(Et^2/\rho)}^{1/5}$ as before.

  \subsubsection*{Solution}
  We will find the $g(t,r,\rho,E,P)=0$ where $P\ne0$. Utilizing the mapping in
  \cref{fig:4-var-mappings}, we obtain the following relation.

  \begin{figure}
    \centering
    \begin{tabularx}{0.5\textwidth}{XXX}
      Variable & Dimension & Exponent \\ \midrule
      $t$ & $T$ & $a$ \\
      $r$ & $L$ & $b$ \\
      $\rho$ & $ML^{-3}$ & $c$ \\
      $E$ & $ML^{2}T^{-2}$ & $d$ \\
      $P$ & $ML^{-1}T^{-2}$ & $e$ \\
    \end{tabularx}
    \caption{Variable mappings for \#4}
\label{fig:4-var-mappings}
  \end{figure}
  \begin{equation*}
    \begin{aligned}
      1 &= {[t]}^a {[r]}^b {[\rho]}^c {[E]}^d {[P]}^e \\
      &= {(T)}^a{(L)}^b{(ML^{-3})}^c{(ML^2T^{-2})}^d{(ML^{-1}T^{-2})}^e \\
      &= M^{c+d+e}L^{b-3c+2d-e}T^{a-2d-2e} \\
    \end{aligned}
  \end{equation*}
  We may now isolate the exponents for each dimension.
  \begin{equation*}
    \begin{aligned}
      c + d + e &= 0 \\
      b - 3c + 2d - e &= 0 \\
      a - 2d - 2e &= 0 \\
    \end{aligned}
  \end{equation*}
  The resulting dimension matrix takes the form
  \begin{equation*}
    \begin{pmatrix}
      0 & 0 & 1 & 1 & 1 \\
      0 & 1 & -3 & 2 & -1 \\
      1 & 0 & 0 & -2 & -2 \\
    \end{pmatrix}
  \end{equation*}
  Which yields the homogeneous system
  \begin{equation*}
    \begin{pmatrix}
      1 & 0 & 0 & -2 & -2 \\
      0 & 1 & 0 & 5 & 2 \\
      0 & 0 & 1 & 1 & 1 \\
    \end{pmatrix} \;
    \begin{pmatrix}
      a \\ b \\ c \\ d \\ e
    \end{pmatrix}
    = 0
  \end{equation*}
  This homogeneous system be solved by standard methods for solving linear
  systems.
  \begin{equation*}
    \begin{pmatrix}
      a \\ b \\ c \\ d \\ e
    \end{pmatrix}
    =
    \begin{pmatrix}
      2 \\ -5 \\ -1 \\ 1 \\ 0
    \end{pmatrix} d
    +
    \begin{pmatrix}
      2 \\ -2 \\ -1 \\ 0 \\ 1
    \end{pmatrix}
    e
  \end{equation*}
  Thus there are two linearly independent solutions which take the form
  \begin{equation}
    \label{eq:4-p-ne-sol}
    \boxed{
      \begin{aligned}
        \implies \pi_1 &= t^2r^{-2}\rho^{-1}P, \\
        \implies \pi_2 &= t^2r^{-5}\rho^{-1}E. \\
      \end{aligned}
    }
  \end{equation}

  In the special case when $P=0$, we may illustrate how $r$ varies with $t$. We
  will use the Pi Theorem to obtain the relation.
  \begin{equation*}
    \begin{aligned}
      & g(t,r,\rho,E,P) = 0 \\
      \leftrightarrow\quad& f(\pi_1, \pi_2) = 0 \\
      \leftrightarrow\quad& \pi_2 = \tilde{f}(\pi_1) \\
      \implies\quad& t^2r^{-5}\rho^{-1}E = \tilde{f}(t^2r^{-2}\rho^{-1}P) \\
    \end{aligned}
  \end{equation*}
  When $P=0$, this turns into
  \begin{equation*}
    t^2r^{-5}\rho^{-1}E = \tilde{f}(0)
  \end{equation*}
  Where $\tilde{f}(0)$ is some dimensionless constant $C$. At this point we may
  solve for $r$ to confirm the relation.
  \begin{equation*}
    \begin{aligned}
      t^2r^{-5}\rho^{-1}E &= \tilde{f}(0) \\
      t^2r^{-5}\rho^{-1}E &= C \\
    \end{aligned}
  \end{equation*}

  \begin{equation*}
    \boxed{\implies r = C^{1/5}{\left(\frac{Et^2}{\rho}\right)}^{1/5}}
  \end{equation*}

  \subsubsection*{Check}
  We will verify the dimensions for $\pi_1$ and $\pi_2$.

  \begin{equation*}
    \begin{aligned}
      \pi_1 &= t^2r^{-2}\rho^{-1}P \\
      [\pi_1] &= {[t]}^2{[r]}^{-2}{[\rho]}^{-1}[P] \\
      1 &= T^2L^{-2}{(ML^{-3})}^{-1}(ML^{-1}T^{-2}) \\
      &= T^2L^{-2}M^{-1}L^{3}ML^{-1}T^{-2} \\
      &= M^{-1+1}L^{-2+3+(-1)}T^{2+(-2)} \\
      &= \cancel{M^{-1+1}}\cancel{L^{-2+3+(-1)}}\cancel{T^{2+(-2)}} \\
      \implies 1 &= 1 \quad\checkmark \\
    \end{aligned}
  \end{equation*}
  \begin{equation*}
    \begin{aligned}
      \pi_2 &= t^{2} r^{-5} \rho^{-1} E  \\
      [\pi_2] &= {[t]}^{2} {[r]}^{-5} {[\rho]}^{-1} [E] \\
      1 &= T^2 L^{-5} {(ML^{-3})}^{-1} (ML^2T^{-2}) \\
      &= T^2 L^{-5} M^{-1}L^{3} ML^2T^{-2} \\
      &= M^{-1+1} L^{-5+3+2} T^{2+(-2)} \\
      &= \cancel{M^{-1+1}} \cancel{L^{-5+3+2}} \cancel{T^{2+(-2)}} \\
      \implies 1 &= 1 \quad\checkmark \\
    \end{aligned}
  \end{equation*}

\subsection{5}
  \subsubsection*{Problem}
  A physical system is described by a law $f(E,P,A)=0$, where $E$, $P$, and $A$
  are energy, pressure, and area, respectively. Show that
  \begin{equation}
    \label{eq:5-problem}
    \frac{PA^{3/2}}{E} = C,
  \end{equation}
  where $C$ is some constant.

  \subsubsection*{Solution}
  \begin{figure}
    \centering
    \begin{tabularx}{0.5\textwidth}{XXX}
      Variable & Dimension & Exponent \\ \midrule
      $E$ & $ML^2T^{-2}$ & $a$ \\
      $P$ & $ML^{-1}T^{-2}$ & $b$ \\
      $A$ & $L^2$ & $c$ \\
    \end{tabularx}
    \caption{Variable mappings for \#5}
\label{fig:5-var-mappings}
  \end{figure}

  The Pi Theorem states that if there is a physical law $F(E, P, A)$, then there
  must be an equivalent physical law $f(\pi_1, \ldots, \pi_k)$ relating some $k$
  number of independent dimensionless quantities $\pi_1,\ldots,\pi_k$. We will
  first determine the number of independent dimensionless quantities which may
  be formed from $E$, $P$, and $A$. Utilizing the mapping in
  \cref{fig:5-var-mappings}, we obtain the following relation.
  \begin{equation*}
    \begin{aligned}
      1 &= {[E]}^a{[P]}^b{[A]}^c \\
      &= {(ML^2T^{-2})}^a{(ML^{-1}T^{-2})}^b{(L^2)}^c \\
      &= M^{a+b}L^{2a-b+2c}T^{-2a-2b} \\
    \end{aligned}
  \end{equation*}
  We may now isolate the exponents for each dimension.
  \begin{equation*}
    \begin{aligned}
      a + b &= 0 \\
      2a - b + 2c &= 0 \\
      -2a - 2b + 2c &= 0 \\
    \end{aligned}
  \end{equation*}
  The resulting dimension matrix takes the form
  \begin{equation*}
    \begin{pmatrix}
      1 & 1 & 0 \\
      2 & -1 & 2 \\
      -2 & -2 & 0 \\
    \end{pmatrix}
  \end{equation*}
  Which yields the homogeneous system
  \begin{equation*}
    \begin{pmatrix}
      1 & 0 & \frac{2}{3} \\
      0 & 1 & -\frac{2}{3} \\
      0 & 0 & 0 \\
    \end{pmatrix}
    \begin{pmatrix}
      a \\ b \\ c \
    \end{pmatrix} = 0
  \end{equation*}
  This homogeneous system may be solved by standard methods for solving linear
  systems.
  \begin{equation}
    \label{eq:5-linear-soln}
    \begin{pmatrix}
      a \\ b \\ c \
    \end{pmatrix} =
    \begin{pmatrix}
      -1 \\
      1 \\
      0 \\
    \end{pmatrix}\frac{2c}{3}
  \end{equation}
  Thus there is one linearly independent solution, which we may name $\pi_1$.

  Now we may show that \cref{eq:5-problem} holds true. By the Pi Theorem, there
  must exist a physical law $f(\pi_1)=0$ which is equivalent to $F(E,P,A)=0$. If
  we assume that \cref{eq:5-problem} is true, then there must be some $\pi_1=C$
  which is a root of $f$. By inspecting \cref{eq:5-problem,eq:5-linear-soln}, we
  may solve for $c$, which results in $c=3/2$. This implies
  \begin{equation*}
    \pi_1 = \frac{PA^{3/2}}{A},
  \end{equation*}
  which is a valid dimensionless root of $f$. Thus, by dimensional analysis and
  the Pi Theorem, we may confirm \cref{eq:5-problem} is valid.

  \newpage
\subsection{9$^2$}
  \subsubsection*{Problem}
  In modeling the digestion process in insects, it is believed that digestion
  yield rate $Y$, in mass per time, is related to the
  concentration\footnote{$M/L^3$} $C$, of the limiting nutrient, the residence
  time $T$ in the gut, the gut volume $V$, and the rate of nutrient
  breakdown\footnote{$MT^{-1}V^{-1}$} $r$. Show that for fixed $T$, $r$, $C$,
  the yield is positively related to the gut volume.

  \subsubsection*{Remarks}
  For fixed $(T,r,C)$, show that $Y$ must be proportional to $V$.

  \subsubsection*{Solution}
  \begin{figure}
    \centering
    \begin{tabularx}{0.5\textwidth}{XXX}
      Variable & Dimension & Exponent \\ \midrule
      $Y$ & $MT^{-1}$ & a \\
      $C$ & $MV^{-1}$ & b \\
      $T$ & $T$ & c \\
      $V$ & $V$ & d \\
      $r$ & $MT^{-1}V^{-1}$ & e \\
    \end{tabularx}
    \caption{Variable mappings for \#9}
\label{fig:9-var-mappings}
  \end{figure}
  The physical equation for the digestion process is given in \cref{eq:9-phy}.
  \begin{equation}
    \label{eq:9-phy}
    f(Y,C,T,V,r)=0
  \end{equation}
  Applying the mapping from \cref{fig:9-var-mappings} gives an equivalent
  dimensionless equation. We will use $V$ instead of $L^3$ for simplicity.
  \begin{equation*}
    \begin{aligned}
      1 &= {[Y]}^a {[C]}^b {[T]}^c {[V]}^d {[r]}^e \\
      &= {(MT^{-1})}^a {(MV^{-1})}^b {(T)}^c {(V)}^d {(MT^{-1}V)}^e \\
      &= M^{a+b+e}V^{-a+c-e}T^{-b+c-e} \\
    \end{aligned}
  \end{equation*}
  We may now isolate the expnents for each dimension.
  \begin{equation*}
    \begin{aligned}
      a + b + e &= 0 \\
      -a + c - e &= 0 \\
      -b + d - e &= 0 \\
    \end{aligned}
  \end{equation*}
  The resulting dimension matrix takes the form
  \begin{equation*}
    \begin{pmatrix}
      1 & 1 & 0 & 0 & 1 \\
      -1 & 0 & 1 & 0 & -1 \\
      0 & -1 & 0 & 1 & -1 \\
    \end{pmatrix}
  \end{equation*}
  Which yields the homogeneous system
  \begin{equation*}
    \begin{pmatrix}
      1 & 0 & 0 & 1 & 0 \\
      0 & 1 & 0 & -1 & 1 \\
      0 & 0 & 1 & 1 & -1 \\
    \end{pmatrix}
    \begin{pmatrix}
      a \\ b \\ c \\ d \\ e
    \end{pmatrix} = 0
  \end{equation*}
  This homogeneous system may be solved by standard methods for solving linear
  systems.
  \begin{equation}
    \label{eq:9-matrix}
    \implies
    \begin{pmatrix}
      a \\ b \\ c \\ d \\ e
    \end{pmatrix}
    =
    \begin{pmatrix}
      -1 \\ 1 \\ -1 \\ 1 \\ 0
    \end{pmatrix}d +
    \begin{pmatrix}
      0 \\ -1 \\ 1 \\ 0 \\ 1
    \end{pmatrix}e
  \end{equation}
  Thus there are two linearly independent solutions to the dimension matrix.
  By the Pi Theorem, we may reduce \cref{eq:9-phy} to
  \begin{equation}
    \tilde{f}(\pi_1, \pi_2) = 0
  \end{equation}
  where $\pi_1$ and $\pi_2$ are two linearly independent dimensionless variables
  which may be obtained from \cref{eq:9-matrix}. We may use one of the
  dimensionless variables $\pi_1$ ($d=1$, $e=0$)to show that $Y$ is proportional
  to $V$.
  \begin{equation*}
    \begin{aligned}
      &\tilde{f}(\pi_1,\pi_2)= 0 \\
      \leftrightarrow\quad& \pi_1 = \hat{f}(\pi_2) \\
      \leftrightarrow\quad& \frac{CV}{TY} = \hat{f}(\pi_2) \\
    \end{aligned}
  \end{equation*}
  Where $\hat{f}(\pi_2)$ is a dimensionless function. If $C$ and $T$ remain
  constant, then
  \begin{equation}
    \label{eq:9-prop}
    \boxed{
      \implies CV = TY\hat{f}(\pi_2)
    }
  \end{equation}
  Thus $Y$ is proportional to $V$.

  \subsubsection*{Check}
  We will verify that the dimensions in \cref{eq:9-prop} are correct.

  \begin{equation*}
    \begin{aligned}
      [T][Y] &= [C][V] \\
      TMT^{-1} &= MV^{-1}V \\
      \cancel{T}M\cancel{T^{-1}} &= M\cancel{V^{-1}}\cancel{V} \\
      \implies M &= M \quad\checkmark \\
    \end{aligned}
  \end{equation*}

  \newpage
  \subsection{13}
  \subsubsection*{Problem}
  Imagine an experiment where we set up a line of dominoes with spacing $d$
  between them. Further, we assume a typical domino has height $h$ and thickness
  $\tau$. We seek a formula that relates these quantities, the gravitational
  constant $g$, and the velocity $v$.

  \begin{enumerate}
  \item Use dimensional analysis to show
    that
    \begin{equation}
      \label{eq:13-1-question}
      v=\sqrt{gh}\; F\left(\frac{d}{h},\frac{\tau}{h}\right).
    \end{equation}
    Assume $\tau/h$ is very small and can be neglected. What does the law become?
  \item Experiments have been performed that show the graph of $v/\sqrt{gh}$ vs.
    $d/h$ is approximately constant, $1.5$, for $d/h$ varying over the range $0$
    to $0.8$. Using $h=0.05$ meters, what is the velocity of the toppling dominoes?
  \end{enumerate}

  \subsubsection*{Solution}
  \begin{enumerate}
  \item
    \begin{figure}
      \centering
      \begin{tabularx}{0.5\textwidth}{XXX}
        Variable & Dimension & Exponent \\ \midrule
        $d$ & $L$ & a \\
        $h$ & $L$ & b \\
        $\tau$ & $L$ & c \\
        $g$ & $L^3M^{-1}T^{-2}$ & d \\
        $v$ & $LT^{-1}$ & e \\
      \end{tabularx}
      \caption{Variable mappings for \#13}
\label{fig:13-var-mappings}
    \end{figure}
    We assume that there is a relation between the physical quantities
    \begin{equation}
      \label{eq:13-phy}
      f(d,h,\tau,g,v) = 0
    \end{equation}
    By the Pi Theorem there must be a relation between some number of
    dimensionless quantities which is equivalent to \cref{eq:13-phy}.
    \begin{equation*}
      \begin{aligned}
        1 &= {[d]}^a {[h]}^b {[\tau]}^c {[g]}^d {[v]}^e \\
        &= L^{a}L^{b}L^{c}L^{c}{(L^3M^{-1}T^{-2})}^{d}{(LT^{-1})}^{e} \\
        &= M^{d}L^{a+b+c+d+e}T^{-2d-e} \\
      \end{aligned}
    \end{equation*}
    We may now isolate the exponents for each dimension
    \begin{equation*}
      \begin{aligned}
        d &= 0 \\
        a + b + c + 3d + e &= 0 \\
        -2d - e &= 0 \\
      \end{aligned}
    \end{equation*}
    Which may be rewritten as simply
    \begin{equation*}
      a + b + c = 0
    \end{equation*}
    By inspection, the resulting solution is in the form
    \begin{equation*}
      \begin{pmatrix}
        a \\ b \\ c \\
      \end{pmatrix} =
      \begin{pmatrix}
        -1 \\ 1 \\ 0
      \end{pmatrix}b +
      \begin{pmatrix}
        -1 \\ 0 \\ 1
      \end{pmatrix}c.
    \end{equation*}
    From this result we obtain two linearly independent solutions: When
    $(b,c)=(-1,0),(-1,1)$ we obtain $d/h$ and $\tau/h$, respectively. This result
    corresponds to \cref{eq:13-1-question} so that
    \begin{equation*}
      f(d,h,\tau,g,v) = 0 \Longleftrightarrow F\left(\frac{d}{h},\frac{\tau}{h}\right) = 0.
    \end{equation*}

    Furthermore, by the Pi Theorem,
    \begin{equation*}
      \begin{aligned}
        F\left(\frac{d}{h},\frac{\tau}{h}\right) &= 0 \\
        G\left(\frac{d}{h}\right) = \frac{\tau}{h}
      \end{aligned}
    \end{equation*}
    If we assume that $\tau/h$ is small and may be neglected, then we may also
    assume that $\tau/h \ll 1$, or $\tau/h \approx 0$.
    \begin{equation*}
      G\left(\frac{d}{h}\right) = 0
    \end{equation*}
    So that the law becomes
    \begin{equation} \boxed{
        \label{eq:13-1-new-law}
        v=\sqrt{gh}\; G\left(\frac{d}{h}\right).
      }
    \end{equation}

  \item The experiment plotted the equation
    \begin{equation*}
      \frac{v}{\sqrt{gh}} = G\left(\frac{d}{h}\right),
    \end{equation*}
    where $g$, $h$, and $d$ are constants as given by the original question. If
    the plot of $v/\sqrt{gh}$ vs. $d/h$ is approximately $1.5$, then that
    implies that $G$ is approximately $1.5$. Therefore
    \begin{equation*}
      \begin{aligned}
        v &= \sqrt{gh}\;G\left(\frac{d}{h}\right) \\
        &= \sqrt{gh}\;(1.5) \\
        &= \sqrt{9.8\cdot0.05}\;(1.5) \\
        \end{aligned}
    \end{equation*}
    \begin{equation*}
      \boxed{\implies v = 1.05\;\text{m/s}.}
    \end{equation*}
  \end{enumerate}

\subsection{14$^3$}
  \subsubsection*{Problem}
  A perfect gas in equilibrium has specific energy\footnote{$L^2T^{-2}$} $E$,
  temperature $T$, and Boltzmann constant\footnote{$L^2T^{-2}\Theta^{-1}$} $k$.
  Derive a functional relationship of the form $E=f(T,k)$.

  \subsubsection*{Remarks}
  Assuming a law $F(E,T,k)=0$, derive an equivalent reduce law and show that
  $E=Tck$ for some constant $c$.

  \subsubsection*{Solution}

  \begin{figure}
    \centering
    \begin{tabularx}{0.5\textwidth}{XXX}
      Variable & Dimension & Exponent \\ \midrule
      $E$ & $L^2T^{-2}$ & a \\
      $T$ & $\Theta$ & b \\
      $k$ & $L^2T^{-2}\Theta^{-1}$ & c \\
    \end{tabularx}
    \caption{Variable mappings for \#14}
\label{fig:14-var-mappings}
  \end{figure}

  Given the physical relationship $F(E,T,k)=0$ and the variable mappings in
  \cref{fig:14-var-mappings}, we may derive an equivalent dimensionless physical
  relationship between $E$, $T$, and $k$.

  \begin{equation}
    \begin{aligned}
      \pi &= E^a T^b k^c \\
      &= {(L^2T^{-2})}^a {(\Theta)}^{b} {(L^2T^{-2}\Theta^{-1})}^c = 1\\
    \end{aligned}
  \end{equation}

  By inspection we can clearly see that $a=1$, $b=-1$, and $c=-1$. Thus

  \begin{equation}
    \pi = \frac{E}{Tk}
  \end{equation}

  This makes the equivalent dimensionless physical relationship

  \begin{equation}
    g\left(\frac{E}{Tk}\right) = 0
  \end{equation}

  Which means that the physical law must take the form

  \begin{equation}
    \frac{E}{Tk} = c
  \end{equation}

  Solving for $E$ gives us

  \begin{equation}
    \boxed{
      \begin{aligned}
        E &= Tck \\
        E &= f(T,k) \\
      \end{aligned}
    }
  \end{equation}

  \subsubsection*{Check}
  Utilizing the dimensions from the mapping in \cref{fig:14-var-mappings}, we
  may verify by inspection.

  \begin{equation}
    \begin{aligned}
      E &= Tck \\
      [E] &= [T][k] \\
      L^2T^{-2} &= L^2T^{-2}\Theta^{-1} \Theta \\
       &= L^2T^{-2} \cancel{\Theta^{-1}} \cancel{\Theta} \\
      \implies L^2T^{-2} &= L^2T^{-2} \quad\checkmark \\
    \end{aligned}
  \end{equation}


\section{Programming Minilab}
A model for an ideal pendulum released from rest is

\begin{equation}
  \label{eq:minilab-pendulum-model}
   \left\{
  \begin{aligned}
    &\ell\ddot{\theta}+g\sin\theta = 0, \quad &t &\ge0 \\
    &\dot{\theta} = 0, \quad &t &= 0 \\
    &\theta = \theta_0, \quad &t &= 0. \\
  \end{aligned}\right.
\end{equation}

Here $\theta$ is the pendulum angle, $\ell$ is the pendulum length, $g$ is the
gravitational acceleration constant, $t$ is time, and over-dots denote time
derivatives. A dimensional analysis of \cref{eq:minilab-pendulum-model} reveals
that the general solution can be written in the form

\begin{equation}
  \label{eq:minilab-general-solution}
  \theta = \tilde{f}(t\sqrt{g/\ell}, \;\theta_0)
\end{equation}
for some function $\tilde{f}$. Here we investigate various consequences of
\cref{eq:minilab-general-solution}. We will use meters and seconds as our units
for length and time, respectively.

\subsection{}
\label{sec:minilab-part-1}
  \subsubsection*{Problem}
  For the initial conditions $\theta_0=3\pi/8$ and interval $t\in[0,2.5]$,
  superimpose plots of $\theta$ versus $t$ for different values of $g$ and $\ell$,
  say, $(g,\ell)=(10, 0.25), (10, 0.5), (10, 1)$. Based on the general solution
  $\theta = f(t,g,\ell,\theta_0)$, briefly explain why different values of $g$ and
  $\ell$ produce different curves.

  \subsubsection*{Solution}
  Reference \cref{fig:minilab-part-1-solution}. It seems that different values of $g$ and
  $\ell$ produce the same curve with different scaling. Based on the general
  solution, $\theta$ is inversely proportional to the square root of $\ell$.
  Thus when $\ell$ varies between $0.25$ to $1$, the curve halves in frequency.

  \begin{figure}
    \centering
    \begin{tikzpicture}
      \begin{axis}[height=16cm, width=16cm,
        domain=0:2.5,
        xmin=0, xmax=2.5,
        ymin=-1.5, ymax=1.5,
        xlabel=$t$,
        ylabel=$\theta$]
        \addplot[] table {src-bin/minilab-1-1.mat};
        \addplot[dashed] table {src-bin/minilab-1-2.mat};
        \addplot[dashed,gray] table {src-bin/minilab-1-3.mat};
      \end{axis}
    \end{tikzpicture}
    \caption{$\theta$ vs $t$ for varying values of $\ell$}
\label{fig:minilab-part-1-solution}
  \end{figure}

\subsection{}
\label{sec:minilab-part-2}
  \subsubsection*{Problem}
  Repeat \cref{sec:minilab-part-1} but now plot $\theta$ vs $t\sqrt{g/\ell}$.
  Based on the form of the solution in \cref{eq:minilab-general-solution},
  briefly explain why different values of $g$ and $\ell$ produce the same curve
  (or portion thereof). If $\theta_0$ were changed in addition to $g$ and
  $\ell$, would we still get this same curve?

  \subsubsection*{Solution}
  Reference \cref{fig:minilab-part-2-solution}. Dimensional analysis of the quantity
  $t\sqrt{g/\ell}$ reveals that $t\sqrt{g/\ell}$ is dimensionless. This means
  that the plot of $\theta$ vs. $t\sqrt{g/\ell}$ effectively re-scales the
  dimensions in \cref{fig:minilab-part-1-solution} to produce \cref{fig:minilab-part-2-solution}.
  Since the pendulum with the smallest length has the fastest frequency, its
  curve is the longest. Furthermore, changing $\theta_0$ in addition to $g$ and
  $\ell$ should produce the same curve since angles are dimensionless.

  \begin{figure}
    \centering
    \begin{tikzpicture}
      \begin{axis}[height=16cm, width=16cm,
        domain=0:20,
        xmin=0, xmax=20,
        ymin=-1.5, ymax=1.5,
        xlabel=$t\sqrt{g/L}$,
        ylabel=$\theta$]
        \addplot[] table {src-bin/minilab-2-1.mat};
        \addplot[dashed] table {src-bin/minilab-2-2.mat};
        \addplot[dashed,gray] table {src-bin/minilab-2-3.mat};
      \end{axis}
    \end{tikzpicture}
    \caption{$\theta$ vs $t\sqrt{g/L}$ for varying values of $\ell$}
\label{fig:minilab-part-2-solution}
  \end{figure}

\subsection{}
  \subsubsection*{Problem}
  Let $T$ be the period of the pendulum motion for given $\theta_0$, $g$,
  $\ell$. Using the form of the solution in \cref{eq:minilab-general-solution}
  and for some function $h$, show that

  \begin{equation}
    \label{eq:2-3-prob}
    T=\sqrt{\ell/g}\;h(\theta_0).
  \end{equation}
  Does this agree with the observations in \cref{sec:minilab-part-1}?
  Specifically, for fixed $\theta_0$ and $g$, does $T$ increase by a factor of
  two when $\ell$ is increased by a factor of four?

  \subsubsection*{Solution}

  Since $\theta$ in \cref{eq:minilab-general-solution} is dimensionless, we may
  utilize the Pi theorem to obtain a relation for $h(\theta_0)$.

  \begin{equation}
    \begin{aligned}
      \theta &= \tilde{f}(t\sqrt{g/\ell},\;\theta_0) \\
      &= \tilde{f}(\pi_1,\;\pi_2) \\
      0 &= F(\pi_1,\;\pi_2) \\
      \pi_1 &= h(\pi_2) \\
      \implies t\sqrt{g/\ell} &= h(\theta_0) \\
    \end{aligned}
  \end{equation}

  Since $t\sqrt{g/\ell}$ is dimensionless, we may confirm that $h(\theta_0)$ is
  also dimensionless. Through dimensional analysis we may also confirm that the
  dimensions in \cref{eq:2-3-prob} agree.

  \begin{equation}
    \begin{aligned}
      T &= \sqrt{\ell/g}\;h(\theta_0) \\
      [T] &= {[\ell]}^{-1/2}{[g]}^{1/2} \\
      T &= L^{-1/2}{(LT^2)}^{1/2} \\
      &= L^{-1/2}L^{1/2}T \\
      &= \cancel{L^{-1/2}}\cancel{L^{1/2}}T \\
      T &= T \quad\checkmark
    \end{aligned}
  \end{equation}

  This agrees with the observations in \cref{sec:minilab-part-1}; since $T$
  varies in relation to $\ell$ by square root, quadrupling $\ell$ doubles $T$
  when $\theta_0$ and $g$ are held constant.

\end{document}
