\documentclass[12pt,twoside]{article}
\title{M374M Homework 5 \\
  \normalsize{\S~2.4 \#1e$^1$, 4f$^2$, 8$^3$} \\
  Revision: \input{revision}}
\author{Hershal Bhave (hb6279)}
\date{Due 2016--03--02}

\usepackage{homework-macros}
\tikzexternalize

\begin{document}
\maketitle
\section{\S~2.4}
\subsection{1e$^1$}
\subsubsection*{Problem}
Consider the system in \cref{eq:2.4.1e-problem}. Determine the nature and
stability properties of the critical points and sketch the phase diagram.
\begin{equation}
  \label{eq:2.4.1e-problem}
  x'=x^2+y^2-4,\quad y'=y-2x
\end{equation}

\subsubsection*{Remarks}
Find all equilibrium solutions; sketch phase diagrams in a small region around
each.

\subsubsection*{Solution}
Let
\begin{equation}
  \label{eq:2.4.1e-let}
  f(x,y)=x^2+y^2-4,\quad g(x,y)=y-2x.
\end{equation}
Since \cref{eq:2.4.1e-let} is a nonlinear system, we may obtain equilibria by
substitution.
\begin{equation*}
  \begin{aligned}
    g(x,y) &= 0 \\
    y - 2x &= 0 \\
    \implies y &= 2x
  \end{aligned}
\end{equation*}
Furthermore,
\begin{equation*}
  \begin{aligned}
    f(x,y) &= 0 \\
    x^2+y^2-4 &= 0 \\
    x^2 + {(2x)}^2 - 4 &= 0 \\
    5x^2 &= 4 \\
    \implies x &= \pm\frac{2}{\sqrt{5}}.
  \end{aligned}
\end{equation*}
From this result the critical points for the system are
$(2/\sqrt{5},4/\sqrt{5})$ and $(-2/\sqrt{5},-4/\sqrt{5})$. We turn to the
linearization of \cref{eq:2.4.1e-let} about each of the critical points to
characterize the stability of the critical points. The linearization of
\cref{eq:2.4.1e-let} may be obtained by first computing the Jacobian matrix. In
general,
\begin{equation*}
  J =
  \begin{pmatrix}
    f_x(x,y) & f_y(x,y) \\
    g_x(x,y) & g_y(x,y) \\
  \end{pmatrix}
\end{equation*}
So that for our problem,
\begin{equation*}
  J =
  \begin{pmatrix}
    2x & 2y \\ -2 & 1 \\
  \end{pmatrix}
\end{equation*}
We'll start by examining $J(2/\sqrt{5},4/\sqrt{5})$.
\begin{equation*}
  J\left(\frac{2}{\sqrt{5}},\frac{4}{\sqrt{5}}\right) =
  \begin{pmatrix}
    \frac{4}{\sqrt{5}} & \frac{8}{\sqrt{5}} \\ -2 & 1 \\
  \end{pmatrix}
\end{equation*}
The eigenvalues of this system characterize the solution.
\begin{equation*}
  \lambda^2 - \left(\frac{4}{\sqrt{5}}+1\right)\lambda + \frac{20}{\sqrt{5}} = 0
\end{equation*}
\begin{equation*}
  \lambda = \frac{\left(\frac{4}{\sqrt{5}}+1\right) \pm
    \sqrt{{\left(\frac{4}{\sqrt{5}}+1\right)}^2-4\left(\frac{20}{\sqrt{5}}\right)}}{2}
\end{equation*}
\begin{equation*}
  \implies \lambda_1 = 1.3944 - 2.6457i, \quad \implies \lambda_2 = 1.3944 + 2.6457i
\end{equation*}
Thus $(2/\sqrt{5},4/\sqrt{5})$ is an unstable spiral. Now we'll examine
$J(-2/\sqrt{5},-4/\sqrt{5})$.
\begin{equation*}
  J\left(-\frac{2}{\sqrt{5}},-\frac{4}{\sqrt{5}}\right) =
  \begin{pmatrix}
    -\frac{4}{\sqrt{5}} & -\frac{8}{\sqrt{5}} \\ -2 & 1 \\
  \end{pmatrix}
\end{equation*}
The eigenvalues of this system characterize the solution.
\begin{equation*}
  \lambda^2 - \left(-\frac{4}{\sqrt{5}}+1\right)\lambda - \frac{20}{\sqrt{5}} = 0
\end{equation*}
\begin{equation*}
  \lambda = \frac{\left(-\frac{4}{\sqrt{5}}+1\right) \pm
    \sqrt{{\left(-\frac{4}{\sqrt{5}}+1\right)}^2-4\left(-\frac{20}{\sqrt{5}}\right)}}{2}
\end{equation*}
\begin{equation*}
  \implies \lambda_1 = 2.6222, \quad \implies \lambda_2 = -3.4110
\end{equation*}
Thus $(-2/\sqrt{5},-4/\sqrt{5})$ is a saddle. \Cref{fig:2.4.1e-phase-diagram}
summarizes this result.

\begin{figure}
  \centering
  \begin{tikzpicture}
    \begin{axis}[height=16cm, width=16cm,
      domain=-2.5:2.5,
      restrict x to domain=-2.5:2.5, xmin=-2.5, xmax=2.5,
      restrict y to domain=-2.5:2.5, ymin=-2.5, ymax=2.5,
      view={0}{90},
      xlabel=$x$,
      ylabel=$y$]
        \foreach \i in {1,...,4} {
          \addplot3[decoration={
          markings,
          mark=between positions 0.4 and 0.6 step 5em with {\arrow [scale=1.5]{stealth}}
        }, postaction=decorate] table {src-bin/2.4.1e-1-\i.mat};
        \addplot3[decoration={
          markings,
          mark=between positions 0.4 and 0.6 step 5em with {\arrow [scale=1.5]{stealth}}
        }, postaction=decorate] table {src-bin/2.4.1e-2-\i.mat};
      }
      \addplot3[gray, quiver={u={x^2+y^2-4}, v={y-2*x}, scale arrows=0.05}, -stealth,samples=20]{0};
      \addplot3[gray, dashed, very thin] coordinates {(-2.5,-1.7889,0) (2.5,-1.7889,0)};
      \addplot3[gray, dashed, very thin] coordinates {(-0.89443,-2.5,0) (-0.89443,2.5,0)};
      \addplot3[gray, dashed, very thin] coordinates {(-2.5,1.7889,0) (2.5,1.7889,0)};
      \addplot3[gray, dashed, very thin] coordinates {(0.89443,-2.5,0) (0.89443,2.5,0)};
      \addplot3[only marks] table {
        -0.89443 -1.7889 0
        0.89443 1.7889 0
      };
    \end{axis}
  \end{tikzpicture}
  \caption{Phase Diagram for \S~2.4\#1e}
  \label{fig:2.4.1e-phase-diagram}
\end{figure}

\subsection{4f$^2$}
\subsubsection*{Problem}
Consider the system in \cref{eq:2.4.4-problem}. Find the values of $\mu$ where
solutions bifurcate and examine the stability of the origin.
\begin{equation}
  \label{eq:2.4.4-problem}
  x'=y,\quad y'=x^2-x+\mu y
\end{equation}

\subsubsection*{Remarks}
Assume $\mu$ can take any value; find all equilibrium solutions; describe type
and stability of each in terms of $\mu$.

\subsubsection*{Solution}
Let
\begin{equation}
  \label{eq:2.4.4f-let}
  f(x,y) = y,\quad g(x,y) = x^2-x+\mu y.
\end{equation}
We'll start by obtaining all equilibria for the system in \cref{eq:2.4.4f-let}.
\begin{equation*}
  \begin{aligned}
    f(x,y) &= 0 \\
    \implies y&=0 \\
  \end{aligned}
\end{equation*}
Furthermore,
\begin{equation*}
  \begin{aligned}
    g(x,y) &= 0 \\
    x^2-x+\cancel{\mu y} &= 0 \\
    x(x-1) &= 0 \\
    \implies x &= 0,\;1 \\
  \end{aligned}
\end{equation*}
The equilibrium points are $(0,0)$ and $(1,0)$. We'll analyze the type and
stability of each, starting with $(0,0)$. We'll do this by computing the
linearization of \cref{eq:2.4.4f-let} about $(0,0)$ and characterize the
equilibrium by the eigenvalues of the linearized model. The linearization may be
computed by determining the Jacobian matrix for \cref{eq:2.4.4f-let}.
\begin{equation*}
  \begin{aligned}
    J &= \begin{pmatrix}
    f_x(x,y) & f_y(x,y) \\
    g_x(x,y) & g_y(x,y) \\
  \end{pmatrix} \\
  &=
  \begin{pmatrix}
    0 & 1 \\ 2x-1 & \mu \\
  \end{pmatrix}
  \end{aligned}
\end{equation*}
So that
\begin{equation*}
  J(0,0) =
  \begin{pmatrix}
    0 & 1 \\ -1 & \mu \\
  \end{pmatrix}.
\end{equation*}
Now we'll compute the eigenvalues for $J(0,0)$.
\begin{equation*}
  0 = \lambda^2-\mu\lambda + 1
\end{equation*}
\begin{equation}
  \label{eq:2.4.4f-lambda-1}
  \implies\lambda = \frac{\mu \pm \sqrt{\mu^2-4}}{2}
\end{equation}
From \cref{eq:2.4.4f-lambda-1}, observe that
\begin{equation}
  \label{eq:2.4.4f-inequality-1}
  \abs{\mu}>\abs{\sqrt{\mu^2-4}}.
\end{equation}
From \cref{eq:2.4.4f-lambda-1,eq:2.4.4f-inequality-1}, we may characterize the
solutions with various values of $\mu$. If $\abs{\mu}\ge2$, then the
discriminant is guaranteed to be real and greater than or equal to zero. If
$\mu<2$, the discriminant is complex, and if $\mu=0$, the the discriminant is
purely imaginary. Further analysis of $\lambda$ yields:
\begin{itemize}
\item $\abs{\mu}\ge2,\;\mu>0 \implies$ two real eigenvalues with $\lambda>0$, unstable node.
\item $\abs{\mu}\ge2,\;\mu<0 \implies$ two real eigenvalues with $\lambda<0$, stable node.
\item $\abs{\mu}<2,\;\mu>0 \implies$ two complex eigenvalues with $\alpha>0$, unstable spiral.
\item $\abs{\mu}<2,\;\mu<0 \implies$ two complex eigenvalues with $\alpha<0$, stable spiral.
\item $\mu=0 \implies$ no conclusion.
\end{itemize}
We'll do the same for $(1,0)$. The Jacobian about $(1,0)$ is
\begin{equation*}
  J(1,0) =
  \begin{pmatrix}
    0 & 1 \\ 1 & \mu
  \end{pmatrix}.
\end{equation*}
Now we'll compute the eigenvalues for $J(1,0)$.
\begin{equation*}
  0 = \lambda^2-\mu\lambda-1
\end{equation*}
\begin{equation}
  \label{eq:2.4.4f-lambda-2}
  \implies \lambda = \frac{\mu \pm \sqrt{\mu^2+4}}{2}
\end{equation}
From \cref{eq:2.4.4f-lambda-2}, observe that
\begin{equation}
  \label{eq:2.4.4f-inequality-2}
  \abs{\mu}<\abs{\sqrt{\mu^2+4}},\quad \sqrt{\mu^2+4}>0.
\end{equation}
This implies that $\lambda$ is always real and opposing signs. Therefore $(1,0)$
is a saddle point regardless of $\mu$.

\subsection{8$^3$}
\subsubsection*{Problem}
Consider the dynamical equation in \cref{eq:2.4.8-problem}. If $p(0,0)$ and $h(0)$
are positive, prove that the origin is an asymptotically stable critical point.
\begin{equation}
  \label{eq:2.4.8-problem}
  x''+p(x,x')x' + xh(x) = 0
\end{equation}
which may be solved by our standard methods.
\subsubsection*{Remarks}
Introduce $y=x'$ and rewrite as a first-order system.

\subsubsection*{Solution}
As per the remark, we may observe
\begin{equation*}
  \implies x'=y
\end{equation*}
Furthermore,
\begin{equation*}
  \begin{aligned}
    y' &= x'' \\
    \implies y' &= -p(x,y)y - h(x)x \\
  \end{aligned}
\end{equation*}
Let
\begin{equation}
  \label{eq:2.5.8-let}
  f(x,y) = y,\quad g(x,y) = -p(x,y)y - h(x)x.
\end{equation}
\Cref{eq:2.5.8-let} is a first-order linear system. We'll construct a matrix to
analyze this linear system.
\begin{equation*}
  A =
  \begin{pmatrix}
    0 & 1 \\ -y\od{}{x}p(x,y)-(x\od{}{t}h(x)+h(x)) & -(y\od{}{y}p(x,y)+p(x,y)) \\
  \end{pmatrix}.
\end{equation*}
At the equilibrium $(0,0)$,
\begin{equation*}
  A(0,0) =
  \begin{pmatrix}
    0 & 1 \\ -h(0) & -p(0,0) \\
  \end{pmatrix}.
\end{equation*}
We may obtain the eigenvalues of $A(0,0)$ to characterize the equilibrium. Let
$p=p(0)$ and $h=h(0)$.
$$\lambda^2 + p\lambda + h$$
\begin{equation}
  \label{eq:2.4.8-lambda}
  \implies \lambda = \frac{-p\pm\sqrt{p^2-4h}}{2}
\end{equation}
Observe that
\begin{equation}
  \label{eq:2.4.8-discriminant}
  \abs{p} < \abs{\sqrt{p^2-4h}}
\end{equation}
Note that $p(0,0)$, $h(0)>0$. From \cref{eq:2.4.8-lambda,eq:2.4.8-discriminant}
we may conclude that $(0,0)$ is a stable node if $p^2-4h>0$ ($\implies
\lambda_i<0$) and a stable spiral if $p^2-4h<0$ ($\implies\alpha_i<0$).

\section{Programming Minilab}
In the biochemical process of glycolysis, living cells obtain energy by breaking
down sugar. Various intermediate compounds are involved, and the overall process
requires that some of these be intercon- verted from one type to the other,
while experiencing some gains and losses. A simple model for the dynamics of two
of these compounds is, in non-dimensional form,
\begin{equation}
  \label{eq:minilab-problem}
  \begin{aligned}
    \od{x}{t} &= -x+ay+x^2y,&\quad x(0)&=x_0 \\
    \od{y}{t} &= b-ay-x^2y,&\quad y(0)&=y_0 \\
  \end{aligned}
\end{equation}
Here $x(t)$ is the concentration of compound $X$ (adenosine diphosphate), $y(t)$
is the concentration of compound $Y$ (fructose-6-phosphate), and $a$ and $b$ are
positive constants that describe the reaction kinetics. Here we perform a
qualitative analysis to understand the behavior of solutions of this system
depending on $a$ and $b$.
\subsubsection*{Problem}
\begin{enumerate}[(a)]
\item Sketch the nullclines and show that the system has only one equilibrium
  solution $(x_∗, y_∗)$ for any $a > 0$ and $b > 0$. Explicitly find the
  equilibrium.
\item \label{item:minilab-part-2} For simplicity, assume $b = 1/2$ is fixed.
  Determine the stability of the equilibrium in terms of $a > 0$. Show that the
  equilibrium is unstable for $0 < a < a_{\#}$, and asymptotically stable for $a >
  a_{\#}$, for an appropriate constant $a_{\#}$.
\item \label{item:minilab-part-3} Consider a shaded region $R$ of the phase
  plane, where the circular hole is centered at $(x_∗, y_∗)$. Show that the
  straight edges of $R$ can be chosen such that the direction field along these
  edges either points inward or along the edge. Using the result in
  \cref{item:minilab-part-2}, and the Poincar\'{e}-Bendixson Theorem, deduce that
  the system must contain a closed orbit in $R$ for any $0 < a < a_{\#}$. Give
  explicit locations for the vertices of $R$.
\item Using $b = 1/2$, illustrate the behavior of the system in the cases $0 < a
  < a_{\#}$ and $a > a_{\#}$. Is the closed orbit predicted in
  \cref{item:minilab-part-3} a stable limit cycle? What happens to the closed
  orbit as $a$ is increased toward $a_{\#}$? Does there appear to be a periodic
  solution for $a > a_{\#}$?
\end{enumerate}
\subsubsection*{Solution}
\begin{enumerate}[(a)]
\item
  Let
  \begin{equation}
    \label{eq:minilab-a-let}
    f(x,y)=-x+ay+x^2y,\quad g(x,y)=b-ay-x^2y.
  \end{equation}
  We may obtain equilibria by solving $f(x,y)=0$ and $g(x,y)=0$ in \cref{eq:minilab-a-let}.
  \begin{equation}
    \label{eq:minilab-a-y1}
    \begin{aligned}
      f(x,y) &= 0 \\
      -x+ay+x^2y &= 0 \\
      y(a+x^2) &= x \\
      y &= \frac{x}{a+x^2} \\
    \end{aligned}
  \end{equation}
  And
  \begin{equation}
    \label{eq:minilab-a-y2}
    \begin{aligned}
      g(x,y) &= 0 \\
      b-ay-x^2y &= 0 \\
      y(a+x^2) &= b \\
      y &= \frac{b}{a+x^2}. \\
    \end{aligned}
  \end{equation}
  These are the x and y nullcline equations for $x$ and $y$, respectively. We
  may set \cref{eq:minilab-a-y1} equal to \cref{eq:minilab-a-y2} to find $x$
  equilibrium.
  \begin{equation*}
    \begin{aligned}
      y &= y \\
      \frac{x}{a+x^2} &= \frac{b}{a+x^2} \\
    \end{aligned}
  \end{equation*}
  \begin{equation*}
    \implies x = b
  \end{equation*}
  From this result,
  \begin{equation*}
    \implies y = \frac{b}{a+b^2}
  \end{equation*}
  Thus, the equilibrium is $(b,b/(a+b^2))$. Reference
  \cref{fig:minilab-1-1,fig:minilab-2-1} for nullclines.
\item If $b=1/2$, then $x_*=1/2$ and $y_*=2/(4a+1)$. We may linearize
  \cref{eq:minilab-a-let} along these points using the Jacobian matrix about
  $(1/2, 2/(4a+1))$.
  \begin{equation*}
    J\left(\frac{1}{2},\frac{2}{4a+1}\right) =
    \begin{pmatrix}
      -1 + \frac{2}{a+1} & a+\frac{1}{4} \\
      -\frac{2}{a+1} & -a-\frac{1}{4} \\
    \end{pmatrix}
  \end{equation*}
  The eigenvalues of this matrix characterize the solution.
  % \begin{equation*}
  %   \lambda^2 - \left(-1-\frac{2}{a+1}-a-\frac{1}{4}\right)\lambda +
  %   \left(-1+-\frac{2}{a+1}\right)\left(-a-\frac{1}{4}\right) -
  %   \left(a+\frac{1}{4}\right)\left(-\frac{2}{a+1}\right)
  % \end{equation*}
  % Or simplified,
  \begin{equation*}
    \lambda^2 - \left(-\frac{5}{4}-a+\frac{2}{4a+1}\right)\lambda + \left(a+\frac{1}{4}\right)=0.
  \end{equation*}
  \begin{equation}
    \label{eq:minilab-2-lambda}
    \implies \lambda = \frac{\left(-\frac{5}{4}-a+\frac{2}{4a+1}\right)
      \pm\sqrt{{\left(-\frac{5}{4}-a+\frac{2}{4a+1}\right)}^2-(4a+1)}}{2}
  \end{equation}
  Observe that
  \begin{equation}
    \label{eq:minilab-2-observation}
    \abs{-\frac{5}{4}-a+\frac{2}{4a+1}} >
    \abs{\sqrt{{\left(-\frac{5}{4}-a+\frac{2}{4a+1}\right)}^2-(4a+1)}}.
  \end{equation}
  In addition, solving ${\left(-\frac{5}{4}-a+\frac{2}{4a+1}\right)}^2-(4a+1)=0$
  shows that the right side of \cref{eq:minilab-2-observation} is negative from
  $-0.0212116$ until $1.93543$. This information combined with the information
  from \cref{eq:minilab-2-lambda,eq:minilab-2-observation} implies that the
  solution changes stability between an unstable center to a stable center at
  $a=a_{\#}$. Solving $\left(-\frac{5}{4}-a+\frac{2}{4a+1}\right)=0$ for $a$
  implies that $a_{\#}=0.116025$; that is, the solution change from unstable to
  stable at $a=a_{\#}=0.116025$ and is complex, i.e.\ a spiral solution.
\item \todo{}
\item Reference \cref{fig:minilab-1-1,fig:minilab-2-1}. The closed orbit is a
  limit cycle. As $a$ increases, the orbit ``radius'' seems to decrease until it
  is a a stable center. There seems to be a stable solution for $a>a_{\#}$.

\begin{figure}
  \centering
  \begin{subfigure}{0.9\textwidth}
  \begin{tikzpicture}
      \begin{axis}[width=12cm, height=12cm,
        domain=0:2.5,
        xmin=0, xmax=1.5, restrict x to domain=0:1.5,
        ymin=0, ymax=2.5, restrict y to domain=0:2.5,
        view={0}{90}, xlabel=$x$, ylabel=$y$]
        \foreach \i in {1,...,4} {
          \addplot3[decoration={markings, mark=between positions 0.1 and 0.9 step 4em with
            {\arrow [scale=1.5]{stealth}} }, postaction=decorate] table {src-bin/minilab-1-1-u-\i.mat};
        }
        \addplot3[gray, quiver={u={-x+.1*y+y*x^2}, v={.5-.1*y-y*x^2)}, scale arrows=0.05}, -stealth,samples=15]{0};
        \addplot3[gray, dashed, very thin] coordinates {(0.5,0,0) (0.5,2.5,0)};
        \addplot3[gray, dashed, very thin] coordinates {(0,1.4286,0) (1.5,1.4286,0)};
        \addplot3[only marks] table {
          0.5 1.4286 0
        };
      \end{axis}
    \end{tikzpicture}
    \caption{Phase Diagram}
    \label{fig:minilab-1-1-phase-diagram}
  \end{subfigure} \\
  \begin{subfigure}{0.49\textwidth}
  \begin{tikzpicture}
      \begin{axis}[width=8cm, height=8cm,
        domain=0:40,
        xmin=0, xmax=40, restrict x to domain=0:40,
        ymin=0, ymax=2, restrict y to domain=0:2,
        view={0}{90}, xlabel=$t$, ylabel=$x$]
        \foreach \i in {1,...,4} {
          \addplot3[decoration={markings, mark=between positions 0.1 and 0.9 step 4em with
            {\arrow [scale=1.5]{stealth}} }, postaction=decorate] table {src-bin/minilab-1-1-x-\i.mat};
        }
        \addplot3[gray, dashed, very thin] coordinates {(0,0.5,0) (40,0.5,0)};
      \end{axis}
    \end{tikzpicture}
    \label{fig:minilab-1-1-x}
    \caption{$x$ vs $t$}
  \end{subfigure}
  \begin{subfigure}{0.49\textwidth}
  \begin{tikzpicture}
      \begin{axis}[width=8cm, height=8cm,
        domain=0:40,
        xmin=0, xmax=40, restrict x to domain=0:40,
        ymin=0, ymax=2.5, restrict y to domain=0:2.5,
        view={0}{90}, xlabel=$t$, ylabel=$y$]
        \foreach \i in {1,...,4} {
          \addplot3[decoration={markings, mark=between positions 0.1 and 0.9 step 4em with
            {\arrow [scale=1.5]{stealth}} }, postaction=decorate] table {src-bin/minilab-1-1-y-\i.mat};
        }
        \addplot3[gray, dashed, very thin] coordinates {(0,1.4286,0) (40,1.4286,0)};
      \end{axis}
    \end{tikzpicture}
    \caption{$y$ vs $t$}\label{fig:minilab-1-1-y}
  \end{subfigure}
  \caption{Minilab Part 4 Results, $\abs{a}<a_{\#}$}
  \label{fig:minilab-1-1}
\end{figure}

\begin{figure}
  \centering
  \begin{subfigure}{0.9\textwidth}
  \begin{tikzpicture}
      \begin{axis}[width=12cm, height=12cm,
        domain=0:2.5,
        xmin=0, xmax=1.5, restrict x to domain=0:1.5,
        ymin=0, ymax=2.5, restrict y to domain=0:2.5,
        view={0}{90}, xlabel=$x$, ylabel=$y$]
        \foreach \i in {1,...,4} {
          \addplot3[decoration={markings, mark=between positions 0.1 and 0.9 step 4em with
            {\arrow [scale=1.5]{stealth}} }, postaction=decorate] table {src-bin/minilab-1-2-u-\i.mat};
        }
        \addplot3[gray, quiver={u={-x+.2*y+y*x^2}, v={.5-.2*y-y*x^2}, scale arrows=0.05}, -stealth,samples=20]{0};
        % \addplot3[gray, dashed, very thin] coordinates {(-1,0,0) (1,0,0)};
        % \addplot3[gray, dashed, very thin] coordinates {(0,-1,0) (0,1,0)};
        \addplot3[gray, dashed, very thin] coordinates {(0.5,0,0) (0.5,2.5,0)};
        \addplot3[gray, dashed, very thin] coordinates {(0,1.1111,0) (1.5,1.1111,0)};
        \addplot3[only marks] table {
          0.5 1.1111 0
        };
      \end{axis}
    \end{tikzpicture}
    \caption{Phase Diagram}
    \label{fig:minilab-2-1-phase-diagram}
  \end{subfigure} \\
  \begin{subfigure}{0.49\textwidth}
  \begin{tikzpicture}
      \begin{axis}[width=8cm, height=8cm,
        domain=0:40,
        xmin=0, xmax=40, restrict x to domain=0:40,
        ymin=0, ymax=2, restrict y to domain=0:2,
        view={0}{90}, xlabel=$t$, ylabel=$x$]
        \foreach \i in {1,...,4} {
          \addplot3[decoration={markings, mark=between positions 0.1 and 0.9 step 4em with
            {\arrow [scale=1.5]{stealth}} }, postaction=decorate] table {src-bin/minilab-1-2-x-\i.mat};
        }
        \addplot3[gray, dashed, very thin] coordinates {(0,0.5,0) (40,0.5,0)};
      \end{axis}
    \end{tikzpicture}
    \label{fig:minilab-2-1-x}
    \caption{$x$ vs $t$}
  \end{subfigure}
  \begin{subfigure}{0.49\textwidth}
  \begin{tikzpicture}
      \begin{axis}[width=8cm, height=8cm,
        domain=0:40,
        xmin=0, xmax=40, restrict x to domain=0:40,
        ymin=0, ymax=2, restrict y to domain=0:2,
        view={0}{90}, xlabel=$t$, ylabel=$y$]
        \foreach \i in {1,...,4} {
          \addplot3[decoration={markings, mark=between positions 0.1 and 0.9 step 4em with
            {\arrow [scale=1.5]{stealth}} }, postaction=decorate] table {src-bin/minilab-1-2-y-\i.mat};
        }
        \addplot3[gray, dashed, very thin] coordinates {(0,1.1111,0) (40,1.1111,0)};
      \end{axis}
    \end{tikzpicture}
    \caption{$y$ vs $t$}\label{fig:minilab-2-1-y}
  \end{subfigure}
  \caption{Minilab Part 4 Results, $\abs{a}>a_{\#}$}
  \label{fig:minilab-2-1}
\end{figure}

\end{enumerate}
\end{document}