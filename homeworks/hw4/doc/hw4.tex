\documentclass[12pt]{article}
\title{M374M Homework 4 \\
  \normalsize{\S~2.1 \#3, 5$^1$; \S~2.2 \#1a, 1f, 5$^2$}}
\author{Hershal Bhave (hb6279)}
\date{Due 2016--02--22}

\usepackage{macros}

\begin{document}
\maketitle
\section{\S~2.1}
\subsection{3}
\subsubsection*{Problem}
For the system
\begin{equation}
  \begin{aligned}
    x'&=y^2 \\
    y'&=-\frac{2}{3}x, \\
  \end{aligned}
\end{equation}
find critical points, nullclines, and the direction of the orbits in the regions
separated by the nullclines. Next, find the equations of the orbits and plot the
phase diagram. Is the origin stable or unstable?
\subsubsection*{Solution}
Define
\begin{equation}
  \begin{aligned}
    f(x,y) &= y^2 \\
    g(x,y) &= -\frac{2}{3}x.
  \end{aligned}
\end{equation}
This is a coupled system. Both the critical points and nullclines involve
solving $f(x,y)=0$ and $g(x,y)=0$.
\begin{multicols}{2}
  \begin{equation*}
    \begin{aligned}
      y^2 &= 0 \\
      \implies y &= 0.
    \end{aligned}
  \end{equation*}
  \begin{equation*}
    \begin{aligned}
      -\frac{2}{3}x &= 0 \\
      \implies x &= 0.
    \end{aligned}
  \end{equation*}
\end{multicols} \noindent
From this result we can conclude that the critical point is $(0,0)$ and that the
nullclines follow the lines $y=0$, $x=0$. To find the equations of the orbits we
must solve the ODE.
\begin{equation}
  \begin{aligned}
    \od{y}{t} &= -\frac{2}{3}x \\
    \od{y}{t} \od{t}{x} &= -\frac{2}{3}x \od{t}{x} \\
    \od{y}{x} &= -\frac{2}{3}\frac{x}{y^2} \\
    \int 3y^2 \dd{y} &= \int -2x^2 \dd{x} \\
    y^3 &= -x^2 + C \\
    \implies y^3 + x^2 &= C
  \end{aligned}
\end{equation}
The phase diagram is plotted in \cref{fig:3-phase-diagram}.

\todo[clean up code comments and phase diagram]
% \begin{tikzpicture}[domain=-3:3,x=2cm,y=1cm]
%   \begin{axis}[axis x line=center,
%     axis y line=center,
%     xlabel=$x$,ylabel=$y$];

%     \foreach \c in {0,2,4,...,10} {
%       \addplot[blue,domain=-3:3] {(-(x^2)+\c)^(1/3)};
%     }
%   \end{axis}
% \end{tikzpicture}

% \begin{tikzpicture}[xscale=1,yscale=.25]
%     \draw plot[id=curve, raw gnuplot] function{
%       f(x,y) = y**2 + (x**2 - 5)*(4*x**4 - 20*x**2 + 25);
%       set xrange [-4:4];
%       set yrange [-15:15];
%       set view 0,0;
%       set isosample 1000,1000;
%       %set size square;
%       set cont base;
%       set cntrparam levels incre 0,0.1,0;
%       unset surface;
%       splot f(x,y)
%     };
%   \end{tikzpicture}

\tikzset{
    set arrow inside/.code={\pgfqkeys{/tikz/arrow inside}{#1}},
    set arrow inside={end/.initial=>, opt/.initial=},
    /pgf/decoration/Mark/.style={
        mark/.expanded=at position #1 with
        {
            \noexpand\arrow[\pgfkeysvalueof{/tikz/arrow inside/opt}]{\pgfkeysvalueof{/tikz/arrow inside/end}}
        }
    },
    arrow inside/.style 2 args={
        set arrow inside={#1},
        postaction={
            decorate,decoration={
                markings,Mark/.list={#2}
            }
        }
    },
}
\tikzset{->-/.style={decoration={
  markings,
  mark=at position #1 with {\arrow{>}}},postaction={decorate}}}

% \tikzset{
%     on each segment/.style={
%         decorate,
%         decoration={
%             show path construction,
%             moveto code={},
%             lineto code={
%                 \path [#1]
%                 (\tikzinputsegmentfirst) -- (\tikzinputsegmentlast);
%             },
%             curveto code={
%                 \path [#1] (\tikzinputsegmentfirst)
%                 .. controls
%                 (\tikzinputsegmentsupporta) and (\tikzinputsegmentsupportb)
%                 ..
%                 (\tikzinputsegmentlast);
%             },
%             closepath code={
%                 \path [#1]
%                 (\tikzinputsegmentfirst) -- (\tikzinputsegmentlast);
%             },
%         },
%     },
% }
% \begin{tikzpicture}
%   \draw[->] (0,-4) -- (0,4) node[left] {$y$};
%   \draw[->] (-4,0) -- (4,0) node[below] {$x$};
%   \draw[color=blue] plot[id=curve, raw gnuplot, smooth,tension=1] function{
%     f(x,y,c) = y**3+x**2+c;
%     set xrange [-4:4];
%     set yrange [-4:4];
%     set view 0,0;
%     set isosample 50,50;
%     % set table;
%     set size square;
%     set cont base;
%     set cntrparam levels incre 0,0.1,0;
%     unset surface;
%     splot for [c=-4:4] f(x,y,c)
%   };
% \end{tikzpicture}

\begin{figure}
  \centering
  \begin{tikzpicture}
    \begin{axis}[ domain=-4:4, view={0}{90}, axis x line=center, axis y
      line=center, xlabel=$x$, ylabel=$y$, x label style={anchor=west}, y label
      style={anchor=south}, ]
      \addplot3[contour gnuplot={number=20,labels=false}]{y^3+x^2};
      % \addplot3[contour gnuplot={number=20}, decoration={
      % markings,
      % mark=between positions 0.1 and 1 step 8em with
      % {\arrow[scale=1.5]{stealth}}},postaction=decorate]{y^3+x^2+3};
      \addplot3[blue, quiver={ u={y^2}, v={-2/3*x}, scale arrows=0.05 },
      -stealth,samples=10]{0};
    \end{axis}
  \end{tikzpicture}
  \caption{Phase Diagram for \S~2.1\#3}
  \label{fig:3-phase-diagram}
\end{figure}

% \begin{tikzpicture}
%   \begin{axis}[
%       axis equal,
%       axis x line = middle,
%       axis y line = middle,
%       xlabel      = {$x$},
%       ylabel      = {$y$},
%     ]
%     \foreach \c in {-4,-3,...,4} {
%       \addplot [domain = -4:4,samples=100, unbounded coords=jump]
%       {(-x^2+\c)^(1/3)};
%     }
%   \end{axis}
% \end{tikzpicture}

\newpage
\subsection{5$^2$}
\subsubsection*{Problem}
Find the linearization of the system
\begin{equation}
  \begin{aligned}
    x'&=y-x \\
    y'&=-y+\frac{5x^2}{4+x^2} \\
  \end{aligned}
\end{equation}
about the equilibrium point $(1,1)$. Draw the nullclines and indicate the
direction of the orbits in each region separated by the nullclines. Can you
conclude anything baout the stability of the critical point?
\subsubsection*{Remarks}
Instead of the linearization, find all equilibria; sketch the direction of
solution curves in a small region around each equilibrium in the phase plane.
\subsubsection*{Solution}
Let
\begin{equation}
  \begin{aligned}
    f(x,y) &= y-x \\
    g(x,y) &= -y+\frac{5x^2}{4+x^2} \\
  \end{aligned}
\end{equation}
We must solve $f(x,y)=0$ and $g(x,y)=0$ to obtain the equilibria of the system.
Since this system is nonlinear we may solve using substitution.
\begin{equation}
  y - x = 0 \implies y = x
\end{equation}
From this result,
\begin{equation}
  \begin{aligned}
    -y+\frac{5x^2}{4+x^2} &= 0 \\
    -x+\frac{5x^2}{4+x^2} &= 0 \\
    \frac{5x^2}{4+x^2} &= x \\
    5x^2 &= x(4+x^2) \\
    5x &= 4+x^2 \\
    x^2-5x+4 &= 0 \\
    (x-5)(x-1) &= 0 \\
    \implies x &= 1, 4. \\
  \end{aligned}
\end{equation}
We may conclude that the equilibria are $(1,1)$, $(4,4)$.
\todo[find the solution curves]
\section{\S~2.2}
\subsection{1a}
\subsubsection*{Problem}
For the system
\begin{equation}
  \label{eq:1a-problem}
  \begin{aligned}
    x'&=x-3y\\
    y'&=-3x+y,\\
  \end{aligned}
\end{equation}
find the general solution of \cref{eq:1a-problem} and sketch the phase diagrams
for the system; characterize the equilibria as to type (node, etc.) and
stability.
\subsubsection*{Solution}
\todo[]

\subsection{1f}
\subsubsection*{Problem}
For the system
\begin{equation}
  \label{eq:1f-problem}
  \begin{aligned}
    x'&=-2x-3y\\
    y'&=3x-2y,\\
  \end{aligned}
\end{equation}
Find the general solution of \cref{eq:1f-problem} and sketch the phase diagrams
for the system; characterize the equilibria as to type (node, etc.) and
stability.
\subsubsection*{Solution} \todo[]

\subsection{5$^2$}
\subsubsection*{Problem}
At time $t=0$ a chemical herbicide is sprayed on a soil in a field of crops. Let
$x$ be the amount of herbicide in the crop, and let $y$ be the amount of
herbicide in the soil. Suppose herbicide is transferred from the soil to the
crop at rate $\beta y$ and transferred from the crop to the soil at rate $\alpha
x$; further, assume that the chemical degrades in the soil at rate $\gamma y$.
Set up a modelfor the amounts of herbicide int he crop and in soil. Sketch a
phase plane diagram indicating how the system evolves in time.
\subsubsection*{Solution}
\todo[]

\section{Programming Minilab}
\todo[]

\end{document}
