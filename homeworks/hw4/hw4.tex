\documentclass[12pt,twoside]{article}
\title{M374M Homework 4 \\
  \normalsize{\S~2.1 \#3, 5$^1$; \S~2.2 \#1a, 1f, 5$^2$} \\
  Revision: \input{revision}}
\author{Hershal Bhave (hb6279)}
\date{Due 2016--02--22}

\usepackage{homework-macros}
\tikzexternalize

\begin{document}
\maketitle
\section{\S~2.1}
\subsection{3}
\subsubsection*{Problem}
For the system described in \cref{eq:2.1.3-problem}, find critical points,
nullclines, and the direction of the orbits in the regions separated by the
nullclines. Next, find the equations of the orbits and plot the phase diagram.
Is the origin stable or unstable?
\begin{equation}
  \label{eq:2.1.3-problem}
  x'=y^2, \qquad y'=-\frac{2}{3}x,
\end{equation}
\subsubsection*{Solution}
Let $f(x,y) = x'$, $g(x,y) = y'$. This is a coupled system. We must solve
$f(x,y)=0$ and $g(x,y)=0$ to obtain the critical points/equilibria and
nullclines of the system.
\begin{multicols}{2}
  \begin{equation*}
    \begin{aligned}
      y^2 &= 0 \\
      \implies y &= 0.
    \end{aligned}
  \end{equation*}
  \begin{equation*}
    \begin{aligned}
      -\frac{2}{3}x &= 0 \\
      \implies x &= 0.
    \end{aligned}
  \end{equation*}
\end{multicols} \noindent
From this result we may conclude that the critical point is $(0,0)$ and that the
nullclines follow the lines $y=0$, $x=0$. To find the equations of the orbits we
must solve the ODE\@.
\begin{equation}
  \begin{aligned}
    \od{y}{t} &= -\frac{2}{3}x \\
    \frac{\dd{y}}{\cancel{\dd{t}}} \frac{\cancel{\dd{t}}}{\dd{x}} &=
    -\frac{2}{3}x \od{t}{x} \\
    \int 3y^2 \dd{y} &= \int -2x^2 \dd{x} \\
    y^3 &= -x^2 + C \\
    \implies y^3 + x^2 &= C
  \end{aligned}
\end{equation}
The phase diagram is plotted in \cref{fig:3-phase-diagram}.

\begin{figure}
  \centering
  \begin{tikzpicture}
    \begin{axis}[height=16cm, width=16cm,
      domain=-1:1,
      restrict x to domain=-1:1, xmin=-1, xmax=1,
      restrict y to domain=-1:1, ymin=-1, ymax=1,
      xtick={-1,-0.5,0,0.5,1}, ytick={-1,-0.5,0,0.5,1},
      view={0}{90},
      xlabel=$x$, ylabel=$y$]
      \foreach \i in {1,...,5} {
        \addplot3[decoration={
          markings,
          mark=between positions 0.4 and 0.6 step 5em with {\arrow [scale=1.5]{stealth}}
        }, postaction=decorate] table {src-bin/2.1.3-\i.mat};
      }
      % \addplot3[black, contour gnuplot={levels={0,0.5,1,2}, labels=false, draw color=black}]{y^3+x^2};
      \addplot3[gray, quiver={u={y^2}, v={-2/3*x}, scale arrows=0.1}, -stealth,samples=10]{0};
      \addplot3[gray, dashed, very thin] coordinates {(-1,0,0) (1,0,0)};
      \addplot3[gray, dashed, very thin] coordinates {(0,-1,0) (0 ,1,0)};
    \end{axis}
  \end{tikzpicture}
  \label{fig:3-phase-diagram}
  \caption{Phase Diagram for \S~2.1\#3}
\end{figure}

\subsection{5$^2$}
\subsubsection*{Problem}
Find the linearization of \cref{eq:2-1-5-problem} about the equilibrium point
$(1,1)$. Draw the nullclines and indicate the direction of the orbits in each
region separated by the nullclines. Can you conclude anything about the
stability of the critical point?
\begin{equation}
  \label{eq:2-1-5-problem}
  x'=y-x, \qquad y'=-y+\frac{5x^2}{4+x^2}
\end{equation}
\subsubsection*{Remarks}
Instead of the linearization, find all equilibria; sketch the direction of
solution curves in a small region around each equilibrium in the phase plane.
\subsubsection*{Solution}
Let $f(x,y) = x'$, $g(x,y) = y'$. We must solve $f(x,y)=0$ and $g(x,y)=0$ to
obtain the equilibria of the system. Since this system is nonlinear we may solve
using substitution.
\begin{equation}
  y - x = 0 \implies y = x
\end{equation}
From this result,
\begin{equation}
  \begin{aligned}
    -y+\frac{5x^2}{4+x^2} &= 0 \\
    -x+\frac{5x^2}{4+x^2} &= 0 \\
    x\left(\frac{5x\ }{4+x^2}-1 \right) &= 0 \\
    \implies x = 0 \\
  \end{aligned}
\end{equation}
This implies that one equilibrium is $(0,0)$. Furthermore,
\begin{equation}
  \begin{aligned}
    \frac{5x\ }{4+x^2} -1 &= 0 \\
    5x &= 4+x^2 \\
    x^2-5x+4 &= 0 \\
    (x-5)(x-1) &= 0 \\
    \implies x &= 1, 4. \\
  \end{aligned}
\end{equation}
We may conclude that the equilibria are $(0,0)$, $(1,1)$, and $(4,4)$. The
stability has been determined via test points and the phase diagram in
\cref{fig:2.1.5-phase-diagram}. From these datum we may conclude that $(0,0)$ is
stable, $(1,1)$ is unstable, and $(4,4)$ is stable.

\begin{figure}
  \centering
  \begin{tikzpicture}
    \begin{axis}[height=16cm, width=16cm,
      domain=-2:8,
      xmin=-2, xmax=8, ymin=-2, ymax=8,
      view={0}{90},
      xlabel=$x$,
      ylabel=$y$]
      \foreach \i in {1,...,6}{
        \foreach \j in {1,...,6}
        \addplot3[decoration={
          markings,
          mark=between positions 0.4 and 0.6 step 5em with {\arrow [scale=1.5]{stealth}}
        }, postaction=decorate] table {src-bin/2.1.5-\i-\j.mat};
      }
      \addplot3[gray, quiver={u={y-x}, v={-y+(5*x^2)/(4+x^2)}, scale arrows=0.05}, -stealth,samples=15]{0};
      \addplot3[gray, dashed, very thin] coordinates {(-2,4,0) (8,4,0)};
      \addplot3[gray, dashed, very thin] coordinates {(4,-2,0) (4,8,0)};
      \addplot3[gray, dashed, very thin] coordinates {(-2,1,0) (8,1,0)};
      \addplot3[gray, dashed, very thin] coordinates {(1,-2,0) (1,8,0)};
      \addplot3[gray, dashed, very thin] coordinates {(-2,0,0) (8,0,0)};
      \addplot3[gray, dashed, very thin] coordinates {(0,-2,0) (0 ,8,0)};
      \addplot3[only marks] coordinates {(0,0,0) (1,1,0) (4,4,0)};
    \end{axis}
  \end{tikzpicture}
  \label{fig:2.1.5-phase-diagram}
  \caption{Phase Diagram for \S~2.1\#5}
\end{figure}

\section{\S~2.2}
\subsection{1a}
\subsubsection*{Problem}
Find the general solution of \cref{eq:1a-problem} and sketch the phase diagrams
for the system; characterize the equilibria as to type (node, etc.) and
stability.
\begin{equation}
  \label{eq:1a-problem}
  x'=f(x,y)=x-3y,\qquad y'=g(x,y)=-3x+y
\end{equation}
\subsubsection*{Solution}
We must solve $f(x,y)=0$ and $g(x,y)=0$ to obtain the equilibria of the system.
Since this problem is linear we may construct a matrix and use various linear
techniques to characterize and ultimately solve the system. We'll start off by
obtaining the equlibria.

\begin{equation*}
  A = \begin{pmatrix}1&-3\\-3&1\\\end{pmatrix}
  = \begin{pmatrix}1&0\\0&1\\\end{pmatrix}
\end{equation*}
The null space of $A$ is
$$\begin{pmatrix}x\\y\end{pmatrix} = \begin{pmatrix}0\\0\end{pmatrix},$$ which
implies that the equlibrium point is $(0,0)$. Now we will find the eigenvalues of
$A$ to characterize the solution about its equilibrium.
\begin{equation*}
  \begin{vmatrix}1-\lambda&-3\\-3&1-\lambda\end{vmatrix} = {(1-\lambda)}^2 - 9 = 0
\end{equation*}
From this point we know there are two distinct, real eigenvalues.
\begin{equation*}
  \begin{aligned}
    1-\lambda_1&=3 &\quad -1+\lambda_2&=3 \\
    \implies\lambda_1 &=-2 &\quad \implies\lambda_2 &= 4 \\
  \end{aligned}
\end{equation*}
Now we require the eigenvectors of $A$ so that the nullclines may be drawn atop
the solution.
\begin{equation*}
  \begin{aligned}
    &\begin{pmatrix}1-\lambda_1&-3\\-3&1-\lambda_1\end{pmatrix}
    \rightarrow\begin{pmatrix}3&-3\\-3&3\end{pmatrix}
    \rightarrow\begin{pmatrix}1&-1\\0&0\end{pmatrix}
    &&\implies \vec{v_1} = \begin{pmatrix}1\\1\end{pmatrix}y, \\
    &\begin{pmatrix}1-\lambda_2&-3\\-3&1-\lambda_2\end{pmatrix}
    \rightarrow\begin{pmatrix}-3&-3\\-3&-3\end{pmatrix}
    \rightarrow\begin{pmatrix}1&1\\0&0\end{pmatrix}
    &&\implies \vec{v_2} = \begin{pmatrix}1\\-1\end{pmatrix}y.
  \end{aligned}
\end{equation*}
$$\boxed{\begin{pmatrix}x(t)\\y(t)\end{pmatrix}=C_1\begin{pmatrix}1\\1\end{pmatrix}e^{-2t}+C_2\begin{pmatrix}1\\-1\end{pmatrix}e^{4t}}$$
The general solution is unstable with a saddle equilibrium. Reference
\cref{fig:2.2.1a-phase-diagram}.

\begin{figure}
  \centering
  \begin{tikzpicture}
    \begin{axis}[height=16cm, width=16cm,
      domain=-2:2,
      restrict x to domain=-2:2, xmin=-2, xmax=2,
      restrict y to domain=-2:2, ymin=-2, ymax=2,
      xtick={-2,-1,0,1,2}, ytick={-2,-1,0,1,2},
      view={0}{90},
      xlabel=$x$,
      ylabel=$y$]
      \foreach \i in {1,...,6}{
        \foreach \j in {1,...,6}
        \addplot3[decoration={
          markings,
          mark=between positions 0.1 and 0.9 step 5em with {\arrow [scale=1.5]{stealth}}
        }, postaction=decorate] table {src-bin/2.2.1a-\i-\j.mat};
      }
      \addplot3[gray, quiver={u={x-3*y}, v={-3*x+y}, scale arrows=0.05}, -stealth,samples=15]{0};
      \addplot3[gray, dashed, very thin] coordinates {(-2,0,0) (2,0,0)};
      \addplot3[gray, dashed, very thin] coordinates {(0,-2,0) (0,2,0)};
      \addplot3[only marks] coordinates {(0,0,0)};
    \end{axis}
  \end{tikzpicture}
  \label{fig:2.2.1a-phase-diagram}
  \caption{Phase Diagram for \S~2.2\#1a}
\end{figure}

\subsection{1f}
\subsubsection*{Problem}
Find the general solution of \cref{eq:1f-problem} and sketch the phase diagrams
for the system; characterize the equilibria as to type (node, etc.) and
stability.
\begin{equation}
  \label{eq:1f-problem}
  x'=f(x,y)=-2x-3y,\qquad y'=g(x,y)=3x-2y
\end{equation}
\subsubsection*{Solution}
We must solve $f(x,y)=0$ and $g(x,y)=0$ to obtain the equilibria of the system.
Since this problem is linear we may construct a matrix and use various linear
techniques to characterize and ultimately solve the system. We'll start off by
obtaining the equlibria.
\begin{equation*}
  A = \begin{pmatrix}-2&3\\-3&-2\\\end{pmatrix}
  = \begin{pmatrix}1&0\\0&1\\\end{pmatrix}
\end{equation*}
The null space of $A$ is
$$\begin{pmatrix}x\\y\end{pmatrix} = \begin{pmatrix}0\\0\end{pmatrix},$$ which
implies that the equlibrium point is $(0,0)$. Now we will find the eigenvalues of
$A$ to characterize the solution about its equilibrium.
\begin{equation*}
  \begin{vmatrix}-2-\lambda&3\\-3&-2-\lambda\end{vmatrix} = {(-2-\lambda)}^2 + 9 = 0
\end{equation*}
From this point we know there are two distinct, complex eigenvalues.
\begin{equation*}
  \begin{aligned}
    -2-\lambda_1&=3i &\quad 2+\lambda_2&=3i \\
    \implies\lambda_1 &=-3i-2 &\quad \implies\lambda_2 &= 3i-2 \\
  \end{aligned}
\end{equation*}
Now we require the eigenvectors of $A$ so that the nullclines may be drawn atop
the solution.
\begin{equation*}
  \begin{aligned}
    &\begin{pmatrix}-2-\lambda_1&-3\\3&-2-\lambda_1\end{pmatrix}
    \rightarrow\begin{pmatrix}3i&-3\\3&3i\end{pmatrix}
    \rightarrow\begin{pmatrix}1&i\\0&0\end{pmatrix}
    &&\implies \vec{v_1} = \begin{pmatrix}-i\\1\end{pmatrix}y, \\
    &\begin{pmatrix}-2-\lambda_1&-3\\3&-2-\lambda_1\end{pmatrix}
    \rightarrow\begin{pmatrix}3i&-3\\3&3i\end{pmatrix}
    \rightarrow\begin{pmatrix}1&-i\\0&0\end{pmatrix}
    &&\implies \vec{v_2} = \begin{pmatrix}i\\1\end{pmatrix}y.
  \end{aligned}
\end{equation*}
$$
\boxed{\begin{pmatrix}x(t)\\y(t)\end{pmatrix}=C_1e^{-2t}\left[\begin{pmatrix}0\\1\end{pmatrix}\cos(-3t)
    -\begin{pmatrix}-1\\0\end{pmatrix}\sin(-3t)\right]+
  C_2e^{-2t}\left[\begin{pmatrix}0\\1\end{pmatrix}\cos(3t)
    -\begin{pmatrix}1\\0\end{pmatrix}\sin(3t)\right]}
$$
The general solution is stable with a spiral equilibrium. Reference
\cref{fig:2.2.1f-phase-diagram}.

\begin{figure}
  \centering
  \begin{tikzpicture}
    \begin{axis}[height=16cm, width=16cm,
      domain=-2:2,
      xmin=-2, xmax=2, ymin=-2, ymax=2,
      xtick={-2,-1,0,1,2}, ytick={-2,-1,0,1,2},
      view={0}{90},
      xlabel=$x$,
      ylabel=$y$]
      \foreach \i in {1,...,3}{
        \foreach \j in {1,...,3}
        \addplot3[decoration={
          markings,
          mark=between positions 0.1 and 0.9 step 5em with {\arrow [scale=1.5]{stealth}}
        }, postaction=decorate] table {src-bin/2.2.1f-\i-\j.mat};
      }
      \addplot3[gray, quiver={u={-2*x-3*y}, v={3*x-2*y}, scale arrows=0.05}, -stealth,samples=15]{0};
      \addplot3[gray, dashed, very thin] coordinates {(-2,0,0) (2,0,0)};
      \addplot3[gray, dashed, very thin] coordinates {(0,-2,0) (0,2,0)};
      \addplot3[only marks] coordinates {(0,0,0)};
    \end{axis}
  \end{tikzpicture}
  \label{fig:2.2.1f-phase-diagram}
  \caption{Phase Diagram for \S~2.2\#1f}
\end{figure}

\subsection{5$^2$}
\subsubsection*{Problem}
At time $t=0$ a chemical herbicide is sprayed on a soil in a field of crops. Let
$x$ be the amount of herbicide in the crop, and let $y$ be the amount of
herbicide in the soil. Suppose herbicide is transferred from the soil to the
crop at rate $\beta y$ and transferred from the crop to the soil at rate $\alpha
x$; further, assume that the chemical degrades in the soil at rate $\gamma y$.
Set up a model for the amounts of herbicide in the crop and in soil. Sketch a
phase plane diagram indicating how the system evolves in time.
\subsubsection*{Remarks}
Assume $\alpha$, $\beta$, and $\gamma$ are arbitrary positive constants; account
for gains and losses in $x$ and $y$ with appropriate signs in the model
equations.
\subsubsection*{Solution}
Let
\begin{equation}
  \label{eq:2.2.5-model}
  \begin{aligned}
    \od{x}{t} &= -\alpha x + \beta y \\
    \od{y}{t} &= \alpha x - (\beta + \gamma)y. \\
  \end{aligned}
\end{equation}
Since \cref{eq:2.2.5-model} is a linear system, it may be placed in matrix form.
\begin{equation*}
  \implies A = \begin{pmatrix}
    -\alpha & \beta \\ \alpha & -(\beta+\gamma) \\
  \end{pmatrix}
\end{equation*}
Since $\det(A) \ne 0$, the system has one equilibrium at $(0,0)$. We may
categorize the solution based on its eigenvalues, which may be obtained by
solving $\det(A-\lambda I)=0$.
\begin{equation}
  \label{eq:2.2.5-eigenvalue-equation}
  \begin{aligned}
    \det(A-\lambda I) &=
    \begin{vmatrix}
      -\alpha-\lambda & \beta \\ \alpha & -(\beta+\gamma)-\lambda \\
    \end{vmatrix} \\
    0 &= (-\alpha-\lambda)(-(\beta+\gamma)-\lambda) - \beta\alpha \\
    &= (-\alpha-\lambda)(-(\beta+\gamma+\lambda)) - \beta\alpha \\
    &= (\alpha+\lambda)(\beta+\gamma+\lambda) - \beta\alpha \\
    &= \cancel{\alpha\beta} + \alpha\gamma + \alpha\lambda + \beta\lambda +
    \gamma\lambda + \lambda^2 - \cancel{\alpha\beta} \\
    &= \lambda^2 + (\alpha+\beta+\gamma)\lambda + \alpha\gamma \\
    \implies \lambda &= \frac{-(\alpha+\beta+\gamma)\pm\sqrt{{(\alpha+\beta+\gamma)}^2-4(\alpha\gamma)}}{2} \\
  \end{aligned}
\end{equation}
At this point we may categorize the eigenvalues and thus the solution based on
the discriminant in \cref{eq:2.2.5-eigenvalue-equation}. We will determine if
the eigenvalues are positive, negative, or zero.
\begin{equation}
  \label{eq:2.2.5-positive-illustration}
  \begin{aligned}
    & {(\alpha+\beta+\gamma)}^2-4(\alpha\gamma) \\
    \longleftrightarrow\quad& \alpha^2 + 2\alpha\beta + \beta^2 + 2\beta\gamma +
    \gamma^2 - 2\alpha\gamma \\
    \longleftrightarrow\quad& \alpha^2 - 2\alpha\gamma + \gamma^2 + \beta^2 +
    2\alpha\beta + 2\beta\gamma \\
    \longleftrightarrow\quad& {(\alpha-\gamma)}^2 + \gamma^2 + \beta^2 + 2\alpha\beta +
    2\beta\gamma \\
  \end{aligned}
\end{equation}
\Cref{eq:2.2.5-positive-illustration,eq:2.2.5-eigenvalue-equation} illustrate
that the system is guaranteed to have two distinct, real eigenvalues since
${(\alpha-\gamma)}^2$ is guaranteed greater than zero. We must now determine if
the eigenvalues are positive, negative, or zero. The first eigenvalue
corresponds to the negative condition of the quadratic equation in
\cref{eq:2.2.5-eigenvalue-equation} and is guaranteed negative. We will
determine the sign of the second eigenvalue (the positive condition of the
quadratic equation). We'll do this by the assume-and-verify method. Assume
$\lambda_2<0$:
\begin{equation*}
  \begin{aligned}
    \frac{-(\alpha+\beta+\gamma)+\sqrt{{(\alpha+\beta+\gamma)}^2-4(\alpha\gamma)}}{2} &<0 \\
    -(\alpha+\beta+\gamma)+\sqrt{{(\alpha+\beta+\gamma)}^2-4(\alpha\gamma)} &<0 \\
    \sqrt{{(\alpha+\beta+\gamma)}^2-4(\alpha\gamma)} &> (\alpha+\beta+\gamma) \\
  \end{aligned}
\end{equation*}
We may square both sides and retain the inequality since we know both sides are
positive.
\begin{equation}
  \begin{aligned}
    {(\alpha+\beta+\gamma)}^2-4(\alpha\gamma) &> {(\alpha+\beta+\gamma)}^2 \\
    -4(\alpha\gamma) &< 0 \\
    \implies \alpha\gamma &> 0 \quad\checkmark
  \end{aligned}
\end{equation}
Therefore $\lambda_1,\lambda_2<0$; both eigenvalues are distinct, real, and
negative. The solution converges to the equlibrium $(0,0)$. Since we cannot
directly obtain the eigenvectors of the system, a general phase portrait of the
solution curves is illustrated in \cref{fig:2-2-5-phase-diagram}.
\begin{figure}
  \centering
  \begin{tikzpicture}
    \begin{axis}[height=16cm, width=16cm,
      domain=-2:2,
      xmin=-2, xmax=2, ymin=-2, ymax=2,
      xtick={-2,-1,0,1,2}, ytick={-2,-1,0,1,2},
      view={0}{90},
      xlabel=$x$,
      ylabel=$y$]
      \foreach \i in {1,...,3}{
        \foreach \j in {1,...,3}
        \addplot3[decoration={
          markings,
          mark=between positions 0.1 and 0.9 step 5em with {\arrow [scale=1.5]{stealth}}
        }, postaction=decorate] table {src-bin/2.2.5-\i-\j.mat};
      }
      \addplot3[gray, quiver={u={-2*x+3*y}, v={2*x-(3+4)*y}, scale arrows=0.01}, -stealth,samples=15]{0};
      \addplot3[gray, dashed, very thin] coordinates {(-2,0,0) (2,0,0)};
      \addplot3[gray, dashed, very thin] coordinates {(0,-2,0) (0,2,0)};
      \addplot3[only marks] coordinates {(0,0,0)};
    \end{axis}
  \end{tikzpicture}
  \label{fig:2-2-5-phase-diagram}
  \caption{Phase Diagram for \S~2.2\#5}
\end{figure}

\section{Programming Minilab}
A simple model for the relationship dynamics between two people X and Y is
\begin{equation}
  \label{eq:minilab-problem}
  \begin{aligned}
    \dot{x} &= ax+by,\quad x(0)=x_0 \\
    \dot{y} &= cx+dy,\quad y(0)=y_0, \\
  \end{aligned}
\end{equation}
where $x(t)$ is the intensity of X's feelings (for Y), and $y(t)$ is the
intensity of Y's feelings (for X), where positive values indicate love, and
negative values indicate hate. The constants $a$, $b$, $c$ and $d$ can take any
value and characterize their personalities: $a > 0$ and $b > 0$ means X is eager
and responsive, respectively; $a < 0$ and $b < 0$ means X is cautious and
manipulative, respectively. Similar for Y in $c$ and $d$. Here we perform a
qualitative analysis to understand the ultimate fate of a relationship depending
on the personality types.

\subsubsection*{Problem}
\begin{enumerate}
\item Consider the case of $a = 0$, $b > 0$, and $c < 0$, $d = 0$: so X is
  responsive and Y is manipulative, and neither is eager or cautious. In words,
  X warms up when Y is warm, cools down when Y is cool, and has no
  self-amplifying or self-suppressing tendencies. Y behaves analogously, but
  cools down when X is warm, and warms up when X is cool. Show that, if
  $(x_0,y_0)\ne(0,0)$, then the relationship evolves as a never-ending cycle of
  love and hate. Illustrate the behavior of the system on a phase diagram.
\item Consider the case of $a = d < 0$ and $b = c > 0$, so that X and Y are both
  cautious and responsive with identical characteristics. Show that if $\abs{a}
  > b$ (more cautious than responsive), then the relationship always fizzles out
  to mutual apathy. On the other hand, if $\abs{a} < b$ (more responsive than
  cautious), then the relationship is explosive: it will generally end up in
  extreme mutual love or mutual hatred depending on the initial feelings. What
  set of initial feelings lead to mutual love? What about mutual hatred? For
  each case, illustrate the behavior on a phase diagram.
\item Numerically simulate the model equation in \cref{eq:minilab-problem} and
  produce solution curves for various $(x_0,y_0)$. Although the results do not
  depend on specific magnitudes, use values of $a$, $b$, $c$ and $d$ from the
  set ${-2, -1, 0, 1, 2}$ to illustrate the different cases.
\end{enumerate}

\subsubsection*{Solution}
\begin{enumerate}
\item
  \label{itm:minilab-part-1-sol}
  In this case we may assume
  \begin{equation*}
    \begin{aligned}
      \dot{x} &= by \\
      \dot{y} &= cx \\
    \end{aligned}
  \end{equation*}
  This is a linear system, so we may place it into matrix form and analyze the
  eigenvalues in order to categorize the solution.
  \begin{equation*}
    0=\begin{vmatrix} -\lambda & b \\ c & -\lambda \end{vmatrix} = \lambda^2-bc
  \end{equation*}
  \begin{equation*}
    \implies \lambda = \sqrt{bc}.
  \end{equation*}
  Since $b<0$, $\lambda$ is complex. Thus we obtain a spiral solution.
  Furthermore, there is no real component of $\lambda$, so the solution remains
  in a never-ending cycle.
\item In this case we may assume $a=d$, $c=b$. Similar to the step in
  \cref{itm:minilab-part-1-sol}, we may analyze the eigenvalues of the linear
  system in order to categorize its solution.
  \begin{equation*}
    \begin{vmatrix} a-\lambda & b \\ c & d-\lambda \end{vmatrix} = {(a-\lambda)}^2-b^2
  \end{equation*}
  \begin{equation*}
    \begin{aligned}
      {(a-\lambda)}^2 &= b^2 \\
      a-\lambda &= \pm b
      \implies \lambda &= a\pm b \\
    \end{aligned}
  \end{equation*}
  From this result we have two independent eigenvalues $\lambda_1$, $\lambda_2$
  which correspond to the negative and positive case of the RHS, respectively.
  Since $a<0$, $\lambda_1=a-b$ is guaranteed negative, we'll analyze
  $\lambda_2=a+b$. When $\abs{a}>b$, $\lambda_2<0$, which leads the solution to
  be stable, resulting in an apathetic relationship. Contrary, when $\abs{a}<b$,
  the solution is unstable, resulting in an explosive reslationship. This occurs
  regardless of the initial condition.
\item The diagrams in \cref{fig:minilab-1-1,fig:minilab-2-1,fig:minilab-2-2}
  were computed with $[a, b, c, d]$=$[0, 3, -3, 0]$, $[-4, 3, 3, -4]$,
  $[-2, 3, 3, -2]$, in that order.

\end{enumerate}

\begin{figure}
  \centering
  \begin{subfigure}{0.9\textwidth}
    \begin{tikzpicture}
      \begin{axis}[width=12cm, height=12cm, domain=-3:3, xmin=-3, xmax=3,
        ymin=-3, ymax=3, % xtick={-1,-0.5,0,0.5,1}, ytick={-1,-0.5,0,0.5,1},
        view={0}{90}, xlabel=$x$, ylabel=$y$]
        \foreach \i in {1,...,4} {
          \addplot3[decoration={markings, mark=between positions 0.1 and 1 step 4em with
            {\arrow [scale=1.5]{stealth}} }, postaction=decorate] table {src-bin/minilab-1-1-u-\i.mat};
        }
        \addplot3[gray, quiver={u={3*y}, v={-4*x}, scale arrows=0.05}, -stealth,samples=15]{0};
        % \addplot3[gray, dashed, very thin] coordinates {(-1,0,0) (1,0,0)};
        % \addplot3[gray, dashed, very thin] coordinates {(0,-1,0) (0,1,0)};
        \addplot3[only marks] coordinates {(0,0,0)};
      \end{axis}
    \end{tikzpicture}
    \label{fig:minilab-1-1-phase-diagram}
    \caption{Phase Diagram}
  \end{subfigure} \\
  \begin{subfigure}{0.49\textwidth}
    \begin{tikzpicture}
      \begin{axis}[width=8cm, height=8cm, domain=0:6, xmin=0, xmax=6,
        ymin=-3, ymax=3, % xtick={-1,-0.5,0,0.5,1}, ytick={-1,-0.5,0,0.5,1},
        view={0}{90}, xlabel=$t$, ylabel=$x$]
        \foreach \i in {1,...,4} {
          \addplot3[decoration={markings, mark=between positions 0.1 and 0.9 step 4em with
            {\arrow [scale=1.5]{stealth}} }, postaction=decorate] table {src-bin/minilab-1-1-x-\i.mat};
        }
        \addplot3[gray, dashed, very thin] coordinates {(0,0,0) (8,0,0)};
      \end{axis}
    \end{tikzpicture}
    \label{fig:minilab-1-1-x}
    \caption{$x$ vs $t$}
  \end{subfigure}
  \begin{subfigure}{0.49\textwidth}
    \begin{tikzpicture}
      \begin{axis}[width=8cm, height=8cm, domain=0:6, xmin=0, xmax=6,
        ymin=-3, ymax=3, % xtick={-1,-0.5,0,0.5,1}, ytick={-1,-0.5,0,0.5,1},
        view={0}{90}, xlabel=$t$, ylabel=$y$]
        \foreach \i in {1,...,4} {
          \addplot3[decoration={markings, mark=between positions 0.1 and 0.9 step 4em with
            {\arrow [scale=1.5]{stealth}} }, postaction=decorate] table {src-bin/minilab-1-1-y-\i.mat};
        }
        \addplot3[gray, dashed, very thin] coordinates {(0,0,0) (8,0,0)};
      \end{axis}
    \end{tikzpicture}
    \caption{$y$ vs $t$}
    \label{fig:minilab-1-1-y}
  \end{subfigure}
  \label{fig:minilab-1-1}
  \caption{Minilab Part 1 Results}
\end{figure}

% 2-1
\begin{figure}
  \centering
  \begin{subfigure}{0.9\textwidth}
    \begin{tikzpicture}
      \begin{axis}[width=12cm, height=12cm,
        domain=-2:2,
        xmin=-2, xmax=2, restrict x to domain=-2:2,
        ymin=-2, ymax=2, restrict y to domain=-2:2,
        view={0}{90}, xlabel=$x$, ylabel=$y$]
        \foreach \i in {1,...,4} {
          \addplot3[decoration={markings, mark=between positions 0.1 and 0.9 step 4em with
            {\arrow [scale=1.5]{stealth}} }, postaction=decorate] table {src-bin/minilab-2-1-u-\i.mat};
        }
        \addplot3[gray, quiver={u={-4*x+3*y}, v={3*x-4*y}, scale arrows=0.05}, -stealth,samples=15]{0};
        % \addplot3[gray, dashed, very thin] coordinates {(-1,0,0) (1,0,0)};
        % \addplot3[gray, dashed, very thin] coordinates {(0,-1,0) (0,1,0)};
        \addplot3[only marks] coordinates {(0,0,0)};
      \end{axis}
    \end{tikzpicture}
    \caption{Phase Diagram}
    \label{fig:minilab-2-1-phase-diagram}
  \end{subfigure} \\
  \begin{subfigure}{0.49\textwidth}
    \begin{tikzpicture}
      \begin{axis}[width=8cm, height=8cm,
        domain=0:6,
        xmin=0, xmax=6, restrict x to domain=0:6,
        ymin=-2, ymax=2, restrict y to domain=-2:2,
        view={0}{90}, xlabel=$t$, ylabel=$x$]
        \foreach \i in {1,...,4} {
          \addplot3[decoration={markings, mark=between positions 0.1 and 0.9 step 4em with
            {\arrow [scale=1.5]{stealth}} }, postaction=decorate] table {src-bin/minilab-2-1-x-\i.mat};
        }
        \addplot3[gray, dashed, very thin] coordinates {(0,0,0) (8,0,0)};
      \end{axis}
    \end{tikzpicture}
    \label{fig:minilab-2-1-x}
    \caption{$x$ vs $t$}
  \end{subfigure}
  \begin{subfigure}{0.49\textwidth}
    \begin{tikzpicture}
      \begin{axis}[width=8cm, height=8cm,
        domain=0:6,
        xmin=0, xmax=6, restrict x to domain=0:6,
        ymin=-2, ymax=2, restrict y to domain=-2:2,
        view={0}{90}, xlabel=$t$, ylabel=$y$]
        \foreach \i in {1,...,4} {
          \addplot3[decoration={markings, mark=between positions 0.1 and 0.9 step 4em with
            {\arrow [scale=1.5]{stealth}} }, postaction=decorate] table {src-bin/minilab-2-1-y-\i.mat};
        }
        \addplot3[gray, dashed, very thin] coordinates {(0,0,0) (8,0,0)};
      \end{axis}
    \end{tikzpicture}
    \caption{$y$ vs $t$}\label{fig:minilab-2-1-y}
  \end{subfigure}
  \caption{Minilab Part 2 Results, $\abs{a}>b$}
  \label{fig:minilab-2-1}
\end{figure}

% 2-2
\begin{figure}
  \centering
  \begin{subfigure}{0.9\textwidth}
    \begin{tikzpicture}
      \begin{axis}[width=12cm, height=12cm,
        xmin=-200, xmax=200, domain=-200:200,
        ymin=-200, ymax=200, restrict y to domain=-200:200,
        view={0}{90}, xlabel=$x$, ylabel=$y$]
        \foreach \i in {1,...,4} {
          \addplot3[decoration={markings, mark=between positions 0.1 and 1 step 4em with
            {\arrow [scale=1.5]{stealth}} }, postaction=decorate] table {src-bin/minilab-2-2-u-\i.mat};
        }
        \addplot3[gray, quiver={u={-2*x + 3*y}, v={-2*x + 3*y}, scale arrows=0.05}, -stealth,samples=15]{0};
        \addplot3[only marks] coordinates {(0,0,0)};
      \end{axis}
    \end{tikzpicture}
    \label{fig:minilab-2-2-phase-diagram}
    \caption{Phase Diagram}
  \end{subfigure} \\
  \begin{subfigure}{0.49\textwidth}
    \begin{tikzpicture}
      \begin{axis}[restrict y to domain=-200:200, width=8cm, height=8cm, domain=0:6, xmin=0, xmax=6,
        ymin=-200, ymax=200, % xtick={-1,-0.5,0,0.5,1}, ytick={-1,-0.5,0,0.5,1},
        view={0}{90}, xlabel=$t$, ylabel=$x$]
        \foreach \i in {1,...,4} {
          \addplot3[decoration={markings, mark=between positions 0.1 and 0.9 step 4em with
            {\arrow [scale=1.5]{stealth}} }, postaction=decorate] table {src-bin/minilab-2-2-x-\i.mat};
        }
        \addplot3[gray, dashed, very thin] coordinates {(0,0,0) (10,0,0)};
      \end{axis}
    \end{tikzpicture}
    \label{fig:minilab-2-2-x}
    \caption{$x$ vs $t$}
  \end{subfigure}
  \begin{subfigure}{0.49\textwidth}
    \begin{tikzpicture}
      \begin{axis}[restrict y to domain=-200:200, width=8cm, height=8cm, domain=0:6, xmin=0, xmax=6,
        ymin=-200, ymax=200, % xtick={-1,-0.5,0,0.5,1}, ytick={-1,-0.5,0,0.5,1},
        view={0}{90}, xlabel=$t$, ylabel=$y$]
        \foreach \i in {1,...,4} {
          \addplot3[decoration={markings, mark=between positions 0.1 and 0.9 step 4em with
            {\arrow [scale=1.5]{stealth}} }, postaction=decorate] table {src-bin/minilab-2-2-y-\i.mat};
        }
        \addplot3[gray, dashed, very thin] coordinates {(0,0,0) (10,0,0)};
      \end{axis}
    \end{tikzpicture}
    \label{fig:minilab-2-2-y}
    \caption{$y$ vs $t$}
  \end{subfigure}
  \caption{Minilab Part 2 Results, $\abs{a}<b$}
  \label{fig:minilab-2-2}
\end{figure}
\end{document}
