\documentclass[12pt]{article}
\title{M374M Homework 7 \\
  \normalsize{\S~3.1 \#4$^1$, 7$^2$, 9$^3$, 20$^4$}}
\author{Hershal Bhave (hb6279)}
\date{Due 2016--03--25}

\usepackage{homework-macros}
\usepackage{xparse}
\tikzexternalize

\begin{document}
\maketitle

\section{\S~3.1}
\subsection{4$^1$}
\subsubsection*{Problem}
Let
\begin{equation*}
  f(y,\epsilon) = \frac{1}{{(1+\epsilon y)}^{3/2}},
  \quad y=y_0+y_1\epsilon+\cdots
\end{equation*}
Expand $f$ in powers of $\epsilon$ up to $\mathcal{O}(\epsilon^2)$

\subsubsection*{Remarks}
Given $f(y,\epsilon)$ and $y(\epsilon)=y_0+\epsilon y_1+\cdots$, let
$F(\epsilon)=f(y(\epsilon),\epsilon)$. Find the Taylor expansion for
$F(\epsilon)$ up to the $\mathcal{O}(\epsilon^2)$ term.

\subsubsection*{Solution}
The Taylor Series Expansion of $F(\epsilon)=f(y(\epsilon),\epsilon)$ is as
follows
\newcommand\nthdegree[1]{
  \foreach \deg in {1,...,#1} {
    \ifnum #1>0
    \ensuremath{{'}}
    \fi
  }
}
% epsilon*y as a function at nth deriviative or variable at nth degree
% [function/var][deriv degree/var number]
\DeclareDocumentCommand\epsy{s O{0}} {
  \IfBooleanTF #1
  {\ensuremath{\epsilon\,y_{#2}}}
  {\ensuremath{\epsilon\,y\nthdegree{#2}(\epsilon)}}
}
\begin{align*}
  F(\epsilon)
  &= f(0) + \epsilon f'(0) + \frac{\epsilon^2}{2} f''(0) \\
  &= 1 - \frac{3\epsilon}{2}{\left[(1+\epsy^{-5/2} (\epsy[1]))\right]}_{\epsilon=0} \\
  &\quad- \frac{3\epsilon^2}{4}
    {\left[-\frac{5}{2}(1+\epsy^{-7/2})(\epsy[1])
    + {(1+\epsy)}^{-5/2}(\epsy[2])\right]}_{\epsilon=0}
    + \mathcal{O}(\epsilon^3) \\
  &= 1 - \frac{3\epsilon}{2}{\left[{(1+\epsy*)}^{-5/2}(\epsy*[1])\right]} \\
  &\quad- \frac{3\epsilon^2}{4}{\left[-\frac{5}{2}{(1+\epsy*)}^{-7/2}(\epsy*[1])
    + {(1+\epsy*)}^{-5/2}(\epsy*[2])\right]} + \mathcal{O}(\epsilon^3). \\
\end{align*}

\newpage
\subsection{7$^2$}
\subsubsection*{Problem}
Find a three-term approximation to the real solution of ${(x+1)}^3=\epsilon x$.

\subsubsection*{Remarks}
Show standard series in powers of $\epsilon$ does not work; then consider a
generalized series in powers of $\epsilon^{1/3}$. It is helpful to introduce
$\delta=\epsilon^{1/3}$ or $\epsilon=\delta^3$ and work with $\delta$ instead of
$\epsilon$. Find approximation of each root up to
$\mathcal{O}(\delta)=\mathcal{O}(\epsilon^{1/3})$ term.

\subsubsection*{Solution}
Assume
\begin{equation}
  \label{eq:3.1.7-standard-relation}
  x=x_0+\epsilon x_1 + \epsilon^2x_2+\cdots.
\end{equation}
We'll slightly rewrite the problem equation to
\begin{equation}
  \label{eq:3.1.7-rewrite}
  {(x+1)}^3-\epsilon x=0.
\end{equation}
Substituting \cref{eq:3.1.7-standard-relation} into \cref{eq:3.1.7-rewrite} and expanding
yields
\begin{equation*}
  \label{eq:3.1.7-bad-eq}
  \begin{split}
    (x_2^3+\cdots) \epsilon ^6+3 x_1 x_2^2 \epsilon ^5+3 x_2 \left(x_1^2+\left(x_0+1\right)
      x_2\right) \epsilon ^4 \\
    +\left(x_1^3+6 \left(x_0+1\right) x_2 x_1-x_2\right) \epsilon ^3
    +\left(3 x_2 \left(x_0+1\right){}^2+3 x_1^2 \left(x_0+1\right)-x_1\right)
    \epsilon ^2 \\
    +\left(3 x_1 x_0^2+\left(6 x_1-1\right) x_0+3 x_1\right)
    \epsilon +\left(x_0+1\right){}^3&=0.
  \end{split}
\end{equation*}
Collecting the $\epsilon$ terms yields the system of equations
\begin{align*}
  \label{eq:3.1.7-bad-system}
  \epsilon^0 &:\quad \left(x_0+1\right){}^3=0, \\
  \epsilon^1 &:\quad 3 x_1 x_0^2+\left(6 x_1-1\right) x_0+3 x_1=0, \\
  \epsilon^2 &:\quad 3 x_2 \left(x_0+1\right){}^2+3 x_1^2 \left(x_0+1\right)-x_1=0, \\
  \epsilon^3 &:\quad x_1^3+6 \left(x_0+1\right) x_2 x_1-x_2=0, \\
  \epsilon^4 &:\quad 3 x_2 \left(x_1^2+\left(x_0+1\right) x_2\right)=0, \\
  \epsilon^5 &:\quad 3 x_1 x_2^2=0 \\
  \epsilon^6 &:\quad x_2^2+\cdots = 0. \\
\end{align*}
$\epsilon_1$ implies a contradiction in $1=0$; thus we may not use a standard
series in powers of $\epsilon$.

Instead of the standardized series, we will attempt to use the generalized
series in powers of $\epsilon^{1/3}$ from \cref{eq:3.1.7-generalized-relation}
where $\alpha=1/3$ to gain insight into the solution of \cref{eq:3.1.7-rewrite}.
\begin{equation}
  \label{eq:3.1.7-generalized-relation}
  x=x_0+\epsilon^{\alpha} x_1 + \epsilon^{2\alpha}x_2+\cdots
\end{equation}
Substituting \cref{eq:3.1.7-generalized-relation} into \cref{eq:3.1.7-rewrite}
and expanding yields
\begin{equation*}
  \begin{split}
    x_2^3 \epsilon ^2
    +x_2 \left(3 x_1 x_2-1\right) \epsilon ^{5/3}
    +\left(3 x_2 x_1^2-x_1+3 \left(x_0+1\right) x_2^2\right) \epsilon^{4/3} \\
    +\left(x_1^3+6 x_2 x_1+x_0\left(6 x_1 x_2-1\right)\right) \epsilon
    +3\left(x_0+1\right) \left(x_1^2+\left(x_0+1\right) x_2\right) \epsilon^{2/3} \\
    +3\left(x_0+1\right){}^2 x_1 \epsilon^{1/3}
    +\left(x_0+1\right){}^3&=0.
  \end{split}
\end{equation*}
Collecting the $\epsilon$ terms yields the system of equations
\begin{equation*}
  \begin{aligned}
    \epsilon^0 &:\quad \left(x_0+1\right){}^3=0 \\
    \epsilon^{1/3} &:\quad 3 \left(x_0+1\right){}^2 x_1=0 \\
    \epsilon^{2/3} &:\quad 3 \left(x_0+1\right) \left(x_1^2+\left(x_0+1\right) x_2\right)=0 \\
    \epsilon^{1} &:\quad x_1^3+6 x_2 x_1+x_0 \left(6 x_1 x_2-1\right)=0 \\
    \epsilon^{4/3} &:\quad 3 x_2 x_1^2-x_1+3 \left(x_0+1\right) x_2^2=0 \\
    \epsilon^{5/3} &:\quad x_2 \left(3 x_1 x_2-1\right)=0 \\
    \epsilon^{2} &:\quad x_2^3+\cdots=0 \\
  \end{aligned}
\end{equation*}
Solving this system by ``standard methods for solving nonlinear systems''
results in
\begin{equation*}
  (x_0,x_1,x_2) = {\left(-1, -1, -\frac{1}{3}\right),
    \left( -1,{(-1)}^{1/3},-\frac{1}{3}{(-1)}^{2/3} \right),
    \left( -1,-{(-1)}^{2/3},\frac{1}{3}{(-1)}^{1/3} \right)}
\end{equation*}
We may map the individual solutions of $x_i$ into
\cref{eq:3.1.7-generalized-relation} to approximate the roots to
\cref{eq:3.1.7-rewrite}.
\begin{equation*}
  \boxed{
    \begin{aligned}
      x &= -1 - \epsilon^{1/3} - \frac{1}{3}\epsilon^{2/3}, \\
      x &= -1 + {(-1)}^{1/3}\epsilon^{1/3} - \frac{1}{3}{(-1)}^{2/3}\epsilon^{2/3}, \\
      x &= -1 - {(-1)}^{2/3}\epsilon^{2/3} + \frac{1}{3}{(-1)}^{1/3}\epsilon^{2/3}. \\
    \end{aligned}
  }
\end{equation*}

\subsubsection{9$^3$}
\subsubsection*{Problem}
Given
\begin{equation}
  \label{eq:3.1.9-problem-roots}
  x^3-4.001x+0.002=0,
\end{equation}
\begin{equation}
  \label{eq:3.1.9-problem-examine}
  x^3-(4+\epsilon)x+2\epsilon=0,
\end{equation}
why is it easier to examine \cref{eq:3.1.9-problem-examine} to find
approximations to the roots of the cubic equation in
\cref{eq:3.1.9-problem-roots}? Find a two-term approximation to this equation.
\subsubsection*{Remarks}
Find approximation of each root up to $\mathcal{O}(\epsilon^2)$ term.

\subsubsection*{Solution}
Assume
\begin{equation}
  \label{eq:3.1.9-standard-relation}
  x=x_0+\epsilon x_1 + \epsilon^2x_2+\cdots.
\end{equation}
Substituting \cref{eq:3.1.9-standard-relation} into
\cref{eq:3.1.9-problem-examine} and expanding yields
\begin{equation}
  \label{eq:3.1.9-system}
  \begin{split}
    x_2^3 \epsilon ^6+3 x_1 x_2^2 \epsilon ^5
    +3 x_2 \left(x_1^2+x_0 x_2\right) \epsilon ^4 \\
    +\left(x_1^3+6 x_0 x_2 x_1-x_2\right) \epsilon ^3
    +\left(3 x_0 x_1^2-x_1+\left(3 x_0^2-4\right) x_2\right) \epsilon ^2 \\
    +\left(3 x_1 x_0^2-x_0-4 x_1+2\right) \epsilon
    +x_0 \left(x_0^2-4\right) &=0.
  \end{split}
\end{equation}
Collecting the $\epsilon$ terms yields the system of equations
\begin{equation}
  \label{eq:3.1.9-expansion}
  \begin{aligned}
    \epsilon_0 &:\quad x_0 \left(x_0^2-4\right)=0 \\
    \epsilon_1 &:\quad 3x_1 x_0^2-x_0-4 x_1+2=0 \\
    \epsilon_2 &:\quad 3x_0 x_1^2-x_1+\left(3 x_0^2-4\right) x_2=0 \\
    \epsilon_3 &:\quad x_1^3+6 x_0 x_2 x_1-x_2=0 \\
    \epsilon_4 &:\quad 3x_2 \left(x_1^2+x_0 x_2\right)=0 \\
    \epsilon_5 &:\quad 3x_1 x_2^2=0. \\
  \end{aligned}
\end{equation}
From $\epsilon_0$, $\epsilon_1$, $\epsilon_2$ terms we learn\footnote{However,
  when I involved the other powers $\epsilon$, I only get $(x_0,x_1,x_2)
  =(2,0,0).$} : $(x_0,x_1,x_2) =(-2, -1/2, 1/8),\; (0,1/2,-1/8), \;\text{and}\;
(2, 0, 0)$. We may map the individual solutions of $x_i$ into
\cref{eq:3.1.9-standard-relation} to approximate the roots to
\cref{eq:3.1.9-problem-examine}.
\begin{equation*}
  \begin{aligned}
    x &= -2 - \epsilon/2 + \epsilon^2/8 \\
    x &= \epsilon/2 - \epsilon^2/8 \\
    x &= 2.
  \end{aligned}
\end{equation*}

\newpage
\subsection{20$^4$}
\subsubsection*{Problem}
Find a two-term perturbation solution of \cref{eq:3.1.20-problem}.
\begin{equation}
  \label{eq:3.1.20-problem}
  u+u=\frac{1}{1+\epsilon u},\quad u(0)=u,\quad 0<\epsilon\ll1.
\end{equation}

\subsubsection*{Remarks}
Find approximation of solution up to $\mathcal{O}(\epsilon)$ term.

\subsubsection*{Solution}
Assume
\begin{equation}
  \label{eq:3.1.20-u}
  u = u_0 + \epsilon u_1 + \cdots
\end{equation}
We may substitute \cref{eq:3.1.20-u} in the differential equation
\begin{equation*}
  u_0'+\epsilon u_1'+\cdots = \frac{1}{1+\epsilon(u_0+\epsilon u_1+\cdots)}
  - u_0+\epsilon u_1+\cdots,
\end{equation*}
and collect powers of $\epsilon$ to obtain
\begin{equation}
  \begin{aligned}
    \label{eq:3.1.20-diff-equations}
    u_0' &= 1-u_0 \\
    u_1' &= -u_1. \\
  \end{aligned}
\end{equation}
In addition, substituting \cref{eq:3.1.20-u} into the initial condition of
\cref{eq:3.1.20-problem} yields
\begin{equation*}
  u_0(0)+\epsilon u_1(0) = 0,
\end{equation*}
which implies
\begin{equation}
  \label{eq:3.1.20-initial-conditions}
  \begin{aligned}
    u_0(0) &= 0 \\
    u_1(0) &= 0. \\
  \end{aligned}
\end{equation}
The sequence of conditions in
\cref{eq:3.1.20-diff-equations,eq:3.1.20-initial-conditions} represent a set of
linear initial value problems which may be solved individually to approximate
the solution to \cref{eq:3.1.20-problem}. Solving the set in
\cref{eq:3.1.20-diff-equations} with initial conditions in
\cref{eq:3.1.20-initial-conditions} yields
\begin{equation*}
  \begin{aligned}
    u_0 &= 1-e^{-t} \\
    u_1 &= 0 \\
  \end{aligned}
\end{equation*}
So that the final approximation is
\begin{equation*}
  \boxed{
    u = 1-e^{-t}
  }
\end{equation*}

\section{Programming Minilab}
A basic problem in ballistics is to determine how to aim a given weapon in order
for a bullet to strike a given target. Many factors are involved, such as
gravity, wind, Coriolis, and bullet shape and spin effects, and all are
important in long-range shots. Considering only gravity and air resistance, a
simple model for the near-horizontal motion of a bullet is in
\cref{eq:minilab-problem}.
\begin{equation}
  \label{eq:minilab-problem}
  \begin{aligned}
    m\ddot{x}&=-\alpha\dot{x}^2, &\quad \dot{x}(0)&=u_0, &\quad x(0)&=0 \\
    m\ddot{y}&=-\alpha\dot{x}\dot{y}-mg, &\quad \dot{y}(0)&=v_0, &\quad y(0)&=0.
  \end{aligned}
\end{equation}
Here $(x(t),\,y(t))$ is the bullet position, $t$ is time, $m$ is the bullet
mass, $g$ is gravitational acceleration, $\alpha$ is a resistance coefficient,
and $(u_0,v_0)$ is the bullet firing velocity; we assume $u_0^2+v_0^2=c^2$,
where $c>0$ is a constant that depends on the weapon. The problem is to
determine the aiming angle $\theta = \arctan(v_0/u_0)$ required for the bullet
path to intersect a fixed target at $(a,b)$, where $a>0$ and $b$ is arbitrary.
\subsubsection*{Problem}
\begin{enumerate}[(a)]
\item\label{itm:minilab-part-a} After dividing by $m$, consider the resulting
  system with small parameter $\epsilon=\alpha/m\ge0$. In the case of
  $\epsilon=0$, solve the system for $x(t)$ and $y(t)$ and show that the
  targeting problem has zero, one, or two solutions for $\theta$ depending on
  $a$, $b$, $c$, and $g$. If $c=500\,\text{m/s}$ and $g=10\,\text{m/s$^2$}$,
  what is the lowest aiming angle $\theta$ that can be used to strike a target
  at $(a,b)=(500,2)$ meters? How much time is required for the impact? If $s$ is
  the speed at impact, what is the ratio $s/c$?
\item\label{itm:minilab-part-b} For the case of small $\epsilon>0$, find a
  perturbation approximation for $x(t)$ and $y(t)$ up to $\mathcal{O}(\epsilon)$
  terms. Using this approximation, show that the targeting problem has zero,
  one, or two solutions for $\theta$ depending on $a$, $b$, $c$, $g$, and
  $\epsilon$. If $c=500\,\text{m/s}$ and $g=10\,\text{m/s$^2$}$, as before, and
  $\epsilon=0.0005\,\text{m$^{-1}$}$, what is the lowest aiming angle $\theta$
  that can be used to strike the target at $(a,b)=(500,2)$ meters? What is the
  corresponding impact time and speed ratio $s/c$ now?
\item Numerically simulate the model equations in \cref{eq:minilab-problem} and
  confirm your results in \cref{itm:minilab-part-a,itm:minilab-part-b}.
  Specifically, for the given values of the parameters and aiming angles, does
  the bullet path intersect (approximately) the target at the times and speeds
  as predicted?
\end{enumerate}
\newpage
\subsubsection*{Solution}
\begin{enumerate}[(a)]
\item Using \cref{eq:minilab-problem} and $\epsilon=\alpha/m$, Let
  \begin{equation}
    \label{eq:minilab-1-let}
    \begin{aligned}
      \ddot{x} &= -\epsilon\dot{x}^2, \\
      \ddot{y} &= -\epsilon\dot{x}\dot{y}-g. \\
    \end{aligned}
  \end{equation}
  In this case $\epsilon = 0$. We may solve for $x(t)$ and $y(t)$ directly.
  \begin{equation*}
    \begin{aligned}
      \ddot{x} &= 0, \\
      \ddot{y} &= -g.
    \end{aligned}
  \end{equation*}
  First we will solve for $\dot{x}$.
  \begin{equation*}
    \begin{aligned}
      \ddot{x} &= 0 \\
      \od[2]{x}{t} &= 0 \\
      \od{}{t}\int\dd{x} &= \int 0\, \dd{t} \\
      \od{x}{t} = C. \\
    \end{aligned}
  \end{equation*}
  We may invoke the initial condition so that
  \begin{equation*}
    \implies\od{x}{t} = u_0.
  \end{equation*}
  Now we will solve for $\dot{y}$.
  \begin{equation*}
    \begin{aligned}
      \ddot{y} &= -g \\
      \od[2]{y}{t} &= -g \\
      \od{}{t}\int\dd{y} &= -g\dd{t} \\
      \od{y}{t} &= -gt + C. \\
    \end{aligned}
  \end{equation*}
  We may invoke the initial condition so that
  \begin{equation*}
    \implies \od{y}{t} = -gt + v_0.
  \end{equation*}
  Now we will solve for $x$.
  \begin{equation*}
    \begin{aligned}
      \od{x}{t} &= u_0 \\
      \int \dd{x} &= \int u_0 \dd{t} \\
      x &= u_0t + C. \\
    \end{aligned}
  \end{equation*}
  We may invoke the initial condition so that
  \begin{equation*}
    \implies x(t) = u_0 t.
  \end{equation*}
  Finally, we will solve for $y$.
  \begin{equation*}
    \begin{aligned}
      \od{y}{t} &= -gt + v_0 \\
      \int\dd{y} &= \int(-gt + v_0) \dd{t} \\
      y &= -\frac{gt^2}{2} + v_0t + C
    \end{aligned}
  \end{equation*}
  We may invoke the initial condition so that
  \begin{equation*}
    \implies y(t) = -\frac{gt^2}{2} + v_0t.
  \end{equation*}
\end{enumerate}
\todo{}
\end{document}