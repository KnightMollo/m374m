\documentclass[12pt]{article}
\title{M374M Homework 7 \\
  \normalsize{\S~3.1 \#4$^1$, 7$^2$, 9$^3$, 20$^4$}}
\author{Hershal Bhave (hb6279)}
\date{Due 2016--03--25}

\usepackage{homework-macros}
\usepackage{xparse}
\tikzexternalize

\begin{document}
\maketitle

\section{\S~3.1}
\subsection{4$^1$}
\subsubsection*{Problem}
Let
\begin{equation*}
  f(y,\epsilon) = \frac{1}{{(1+\epsilon y)}^{3/2}},
  \quad y=y_0+y_1\epsilon+\cdots
\end{equation*}
Expand $f$ in powers of $\epsilon$ up to $\mathcal{O}(\epsilon^2)$

\subsubsection*{Remarks}
Given $f(y,\epsilon)$ and $y(\epsilon)=y_0+\epsilon y_1+\cdots$, let
$F(\epsilon)=f(y(\epsilon),\epsilon)$. Find the Taylor expansion for
$F(\epsilon)$ up to the $\mathcal{O}(\epsilon^2)$ term.

\subsubsection*{Solution}
The Taylor Series Expansion of $F(\epsilon)=f(y(\epsilon),\epsilon)$ is as
follows
\newcommand\nthdegree[1]{
  \foreach \deg in {1,...,#1} {
    \ifnum #1>0
    \ensuremath{{'}}
    \fi
  }
}
% epsilon*y as a function at nth deriviative or variable at nth degree
% [function/var][deriv degree/var number]
\DeclareDocumentCommand\epsy{s O{0}} {
  \IfBooleanTF #1
  {\ensuremath{\epsilon\,y_{#2}}}
  {\ensuremath{\epsilon\,y\nthdegree{#2}(\epsilon)}}
}
\begin{align*}
  F(\epsilon)
  &= f(0) + \epsilon f'(0) + \epsilon^2 f''(0) \\
  &= 1 - \frac{3}{2}{\left[(1+\epsy^{-5/2}
    (\epsy[1]))\right]}
    _{\epsilon=0}\epsilon \\
  &\quad- \frac{3}{2}
    {\left[-\frac{5}{2}(1+\epsy^{-7/2})(\epsy[1])
    + {(1+\epsy)}^{-5/2}(\epsy[2])\right]}_{\epsilon=0}\epsilon^2
    + \mathcal{O}(\epsilon^3) \\
  &= {\left[-\frac{3}{2}{(1+\epsy*)}^{-5/2}(\epsy*[1])\right]}\epsilon \\
  &\quad- \frac{3}{2}{\left[-\frac{5}{2}{(1+\epsy*)}^{-7/2}(\epsy*[1])
    + {(1+\epsy*)}^{5/2}(\epsy*[2])\right]}\epsilon^2 + \mathcal{O}(\epsilon^3) \\
\end{align*}

\newpage
\subsection{7$^2$}
\subsubsection*{Problem}
Find a three-term approximation to the real solution of ${(x+1)}^3=\epsilon x$.

\subsubsection*{Remarks}
Show standard series in powers of $\epsilon$ does not work; then consider a
generalized series in powers of $\epsilon^{1/3}$. It is helpful to introduce
$\delta=\epsilon^{1/3}$ or $\epsilon=\delta^3$ and work with $\delta$ instead of
$\epsilon$. Find approximation of each root up to
$\mathcal{O}(\delta)=\mathcal{O}(\epsilon^{1/3})$ term.

\subsubsection*{Solution}
\todo{}

\subsubsection{9$^3$}
\subsubsection*{Problem}
Given
\begin{equation}
  \label{eq:3.1.9-problem-roots}
  x^3-4.001x+0.002=0,
\end{equation}
and
\begin{equation}
  \label{eq:3.1.9-problem-examine}
  x^3-(4+\epsilon)x+2\epsilon=0
\end{equation}
Why is it easier to examine \cref{eq:3.1.9-problem-examine} to find
approximations to the roots of the cubic equation in
\cref{eq:3.1.9-problem-roots}? Find a two-term approximation to this equation.
\subsubsection*{Remarks}
Find approximation of each root up to $\mathcal{O}(\epsilon^2)$ term.

\subsection{20$^4$}
\subsubsection*{Problem}
Find a two-term perturbation solution of \cref{eq:3.1.20-problem}.
\begin{equation}
  \label{eq:3.1.20-problem}
  u'+u=\frac{1}{1+\epsilon u},\quad u(0)=u,\quad 0<\epsilon\ll1.
\end{equation}

\subsubsection*{Remarks}
Find approximation of solution up to $\mathcal{O}(\epsilon)$ term.

\subsubsection*{Solution}
\todo{}

\end{document}