\documentclass[12pt]{article}
\title{M374M Homework 7 \\
  \normalsize{\S~3.1 \#4$^1$, 7$^2$, 9$^3$, 20$^4$}}
\author{Hershal Bhave (hb6279)}
\date{Due 2016--03--25}

\usepackage{homework-macros}
\usepackage{xparse}
\tikzexternalize

\begin{document}
\maketitle

\section{\S~3.1}
\subsection{4$^1$}
\subsubsection*{Problem}
Let
\begin{equation*}
  f(y,\epsilon) = \frac{1}{{(1+\epsilon y)}^{3/2}},
  \quad y=y_0+y_1\epsilon+\cdots
\end{equation*}
Expand $f$ in powers of $\epsilon$ up to $\mathcal{O}(\epsilon^2)$

\subsubsection*{Remarks}
Given $f(y,\epsilon)$ and $y(\epsilon)=y_0+\epsilon y_1+\cdots$, let
$F(\epsilon)=f(y(\epsilon),\epsilon)$. Find the Taylor expansion for
$F(\epsilon)$ up to the $\mathcal{O}(\epsilon^2)$ term.

\subsubsection*{Solution}
The Taylor Series Expansion of $F(\epsilon)=f(y(\epsilon),\epsilon)$ is as
follows
\newcommand\nthdegree[1]{
  \foreach \deg in {1,...,#1} {
    \ifnum #1>0
    \ensuremath{{'}}
    \fi
  }
}
% epsilon*y as a function at nth deriviative or variable at nth degree
% [function/var][deriv degree/var number]
\DeclareDocumentCommand\epsy{s O{0}} {
  \IfBooleanTF #1
  {\ensuremath{\epsilon\,y_{#2}}}
  {\ensuremath{\epsilon\,y\nthdegree{#2}(\epsilon)}}
}
\begin{align*}
  F(\epsilon)
  &= f(0) + \epsilon f'(0) + \epsilon^2 f''(0) \\
  &= 1 - \frac{3}{2}{\left[(1+\epsy^{-5/2} (\epsy[1]))\right]}
    _{\epsilon=0}\epsilon \\
  &\quad- \frac{3\epsilon^2}{4}
    {\left[-\frac{5}{2}(1+\epsy^{-7/2})(\epsy[1])
    + {(1+\epsy)}^{-5/2}(\epsy[2])\right]}_{\epsilon=0}
    + \mathcal{O}(\epsilon^3) \\
  &= {\left[-\frac{3}{2}{(1+\epsy*)}^{-5/2}(\epsy*[1])\right]}\epsilon \\
  &\quad- \frac{3\epsilon^2}{4}{\left[-\frac{5}{2}{(1+\epsy*)}^{-7/2}(\epsy*[1])
    + {(1+\epsy*)}^{5/2}(\epsy*[2])\right]} + \mathcal{O}(\epsilon^3) \\
\end{align*}

\newpage
\subsection{7$^2$}
\subsubsection*{Problem}
Find a three-term approximation to the real solution of ${(x+1)}^3=\epsilon x$.

\subsubsection*{Remarks}
Show standard series in powers of $\epsilon$ does not work; then consider a
generalized series in powers of $\epsilon^{1/3}$. It is helpful to introduce
$\delta=\epsilon^{1/3}$ or $\epsilon=\delta^3$ and work with $\delta$ instead of
$\epsilon$. Find approximation of each root up to
$\mathcal{O}(\delta)=\mathcal{O}(\epsilon^{1/3})$ term.

\subsubsection*{Solution}
Assume
\begin{equation}
  \label{eq:3.1.7-standard-relation}
  x=x_0+\epsilon x_1 + \epsilon^2x_2+\cdots.
\end{equation}
We'll slightly rewrite the problem equation to
\begin{equation}
  \label{eq:3.1.7-rewrite}
  {(x+1)}^3-\epsilon x=0.
\end{equation}
Substituting \cref{eq:3.1.7-standard-relation} into \cref{eq:3.1.7-rewrite} and expanding
yields
\begin{equation*}
  \label{eq:3.1.7-bad-eq}
  \begin{split}
    (x_2^3+\cdots) \epsilon ^6+3 x_1 x_2^2 \epsilon ^5+3 x_2 \left(x_1^2+\left(x_0+1\right)
      x_2\right) \epsilon ^4 \\
    +\left(x_1^3+6 \left(x_0+1\right) x_2 x_1-x_2\right) \epsilon ^3
    +\left(3 x_2 \left(x_0+1\right){}^2+3 x_1^2 \left(x_0+1\right)-x_1\right)
    \epsilon ^2 \\
    +\left(3 x_1 x_0^2+\left(6 x_1-1\right) x_0+3 x_1\right)
    \epsilon +\left(x_0+1\right){}^3&=0.
  \end{split}
\end{equation*}
Collecting the $\epsilon$ terms yields the system of equations
\begin{align*}
  \label{eq:3.1.7-bad-system}
  \epsilon^0 &:\quad \left(x_0+1\right){}^3=0, \\
  \epsilon^1 &:\quad 3 x_1 x_0^2+\left(6 x_1-1\right) x_0+3 x_1=0, \\
  \epsilon^2 &:\quad 3 x_2 \left(x_0+1\right){}^2+3 x_1^2 \left(x_0+1\right)-x_1=0, \\
  \epsilon^3 &:\quad x_1^3+6 \left(x_0+1\right) x_2 x_1-x_2=0, \\
  \epsilon^4 &:\quad 3 x_2 \left(x_1^2+\left(x_0+1\right) x_2\right)=0, \\
  \epsilon^5 &:\quad 3 x_1 x_2^2=0 \\
  \epsilon^6 &:\quad x_2^2+\cdots = 0 \\
\end{align*}
$\epsilon_1$ implies a contradiction in $1=0$; thus we may not use a standard
series in powers of $\epsilon$.

Instead of the standardized series, we will attempt to use the generalized
series in powers of $\epsilon^{1/3}$ from \cref{eq:3.1.7-generalized-relation}
where $\alpha=1/3$ to gain insight into the solution of \cref{eq:3.1.7-rewrite}.
\begin{equation}
  \label{eq:3.1.7-generalized-relation}
  x=x_0+\epsilon^{\alpha} x_1 + \epsilon^{2\alpha}x_2+\cdots
\end{equation}
Substituting \cref{eq:3.1.7-generalized-relation} into \cref{eq:3.1.7-rewrite}
and expanding yields
\begin{equation*}
  \begin{split}
    x_2^3 \epsilon ^2
    +x_2 \left(3 x_1 x_2-1\right) \epsilon ^{5/3}
    +\left(3 x_2 x_1^2-x_1+3 \left(x_0+1\right) x_2^2\right) \epsilon^{4/3} \\
    +\left(x_1^3+6 x_2 x_1+x_0\left(6 x_1 x_2-1\right)\right) \epsilon
    +3\left(x_0+1\right) \left(x_1^2+\left(x_0+1\right) x_2\right) \epsilon^{2/3} \\
    +3\left(x_0+1\right){}^2 x_1 \epsilon^{1/3}
    +\left(x_0+1\right){}^3&=0.
  \end{split}
\end{equation*}
Collecting the $\epsilon$ terms yields the system of equations
\begin{equation*}
  \begin{aligned}
    \epsilon^0 &:\quad \left(x_0+1\right){}^3=0 \\
    \epsilon^{1/3} &:\quad 3 \left(x_0+1\right){}^2 x_1=0 \\
    \epsilon^{2/3} &:\quad 3 \left(x_0+1\right) \left(x_1^2+\left(x_0+1\right) x_2\right)=0 \\
    \epsilon^{1} &:\quad x_1^3+6 x_2 x_1+x_0 \left(6 x_1 x_2-1\right)=0 \\
    \epsilon^{4/3} &:\quad 3 x_2 x_1^2-x_1+3 \left(x_0+1\right) x_2^2=0 \\
    \epsilon^{5/3} &:\quad x_2 \left(3 x_1 x_2-1\right)=0 \\
    \epsilon^{2} &:\quad x_2^3+\cdots=0 \\
  \end{aligned}
\end{equation*}
Solving this system by ``standard methods for solving nonlinear systems''
results in
\begin{equation*}
  (x_0,x_1,x_2) = {\left(-1, -1, -\frac{1}{3}\right),
    \left( -1,{(-1)}^{1/3},-\frac{1}{3}{(-1)}^{2/3} \right),
    \left( -1,-{(-1)}^{2/3},\frac{1}{3}{(-1)}^{1/3} \right)}
\end{equation*}
We may map the individual solutions of $x_n$ into
\cref{eq:3.1.7-generalized-relation} to approximate the roots to
\cref{eq:3.1.7-rewrite}.
\begin{equation*}
  \boxed{
    \begin{aligned}
      x &= -1 - \epsilon^{1/3} - \frac{1}{3}\epsilon^{2/3}, \\
      x &= -1 + {(-1)}^{1/3}\epsilon^{1/3} - \frac{1}{3}{(-1)}^{2/3}\epsilon^{2/3}, \\
      x &= -1 - {(-1)}^{2/3}\epsilon^{2/3} + \frac{1}{3}{(-1)}^{1/3}\epsilon^{2/3}. \\
    \end{aligned}
  }
\end{equation*}

\subsubsection{9$^3$}
\subsubsection*{Problem}
Given
\begin{equation}
  \label{eq:3.1.9-problem-roots}
  x^3-4.001x+0.002=0,
\end{equation}
and
\begin{equation}
  \label{eq:3.1.9-problem-examine}
  x^3-(4+\epsilon)x+2\epsilon=0
\end{equation}
Why is it easier to examine \cref{eq:3.1.9-problem-examine} to find
approximations to the roots of the cubic equation in
\cref{eq:3.1.9-problem-roots}? Find a two-term approximation to this equation.
\subsubsection*{Remarks}
Find approximation of each root up to $\mathcal{O}(\epsilon^2)$ term.

\subsection{20$^4$}
\subsubsection*{Problem}
Find a two-term perturbation solution of \cref{eq:3.1.20-problem}.
\begin{equation}
  \label{eq:3.1.20-problem}
  u'+u=\frac{1}{1+\epsilon u},\quad u(0)=u,\quad 0<\epsilon\ll1.
\end{equation}

\subsubsection*{Remarks}
Find approximation of solution up to $\mathcal{O}(\epsilon)$ term.

\subsubsection*{Solution}
\todo{}

\end{document}