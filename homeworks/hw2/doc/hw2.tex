\documentclass[12pt]{article}
\title{M374M Homework 2 \\
  \normalsize{\S~1.2 \#1$^1$, 3$^2$, 4$^3$, 9}}
\author{Hershal Bhave (hb6279)}
\date{Due 2016-02-08}

\usepackage{macros}

\begin{document}
\maketitle

\section{\S~1.2}
\subsection{1$^1$}

\subsubsection*{Problem}
Let $u=u(t),\;0 \le t \le b$, be a given smooth function. If
$M=\text{max}\abs{u(t)}$, then $u$ can be scales by $M$ to obtain the
dimensionless dependent $U-u/M$. A time scale can be taken as
$c_c=M/\text{max}\abs{u'(t)}$, the ratio of the maximum slope. Find $M$ and $_c$
for the following functions:
\begin{enumerate}
\item $u(t)=A\sin\omega t, \quad t>0$.
\item $u(t)=Ae^{-\lambda t},\quad t>0$.
\item $u(t)=Ate^{-\lambda t},\quad 0 \le t \le 2/\lambda.$
\end{enumerate}

\subsubsection*{Remarks}
This problem illustrates an alternative way to identify scales. Assume $A$,
$\omega$, and $\lambda$ are positive constants.

\subsubsection*{Solution}
\begin{enumerate}
\item
  \begin{equation} \boxed{
    \begin{aligned}
      M &= \text{max}\abs{A\sin\omega t} \\
      \implies M &= A \\
    \end{aligned}
    }
  \end{equation}

  \begin{equation} \boxed{
    \begin{aligned}
      t_c &= M/\text{max}\abs{u'(t)} \\
      &= A/\text{max}\abs{\omega A \cos \omega t} \\
      \implies t_c &= 1/\omega \\
    \end{aligned}
    }
  \end{equation}
\item
  \begin{equation} \boxed{
    \begin{aligned}
      M &= \text{max}\abs{Ae^{-\lambda t}} \\
      \implies M &= A \\
    \end{aligned}
    }
  \end{equation}

  \begin{equation} \boxed{
    \begin{aligned}
      t_c &= M/\text{max}\abs{u'(t)} \\
      &= A/\text{max}\abs{-A\lambda e^{-\lambda t}} \\
      \implies t_c &= 1/\lambda \\
    \end{aligned}
    }
  \end{equation}

\item
\todo
\end{enumerate}

\subsection{3$^2$}
\subsubsection*{Problem}
The growth rate of an organism is often measured using carbon biomass as the
``currency.'' The \textbf{von Bertalanffy growth model} is $$m' = ax^2 - bx^3,$$
where $m$ is its biomass, $x$ is some characteristic length of the organism, $a$
is its biomass assimilation rate, and $b$ is its biomass use rate. Thus, it
assimilates nutrients proportional to its area, and it uses nutrients
proportional to its volume. Assume $m=\rho x^2$ and rewrite the model in terms
of the length $x$. Determine the dimensions of the constants $a$, $b$, and
$\rho$. Select time and length scales $\rho/b$ and $a/b$, repsectively, and
reduce the problem to dimensionless form. If $x(0)=0$, find the length $x$ at
time $t$. Does this seem like a reasonable model?

\subsubsection*{Remarks}
Assume $m$, $x$ are functions of $t$. The notation $m'$ means $\od{m}{t}$. As
time $t$ increases, show that the size $x$ of an organism does not grow without
bound, but instead approaches some maximum value.

\subsubsection*{Solution}
\todo
\subsubsection*{Check}
\todo

\subsection{4$^3$}
\subsubsection*{Problem}
A mass hanging on a spring is given a positive initial velocity $V$ from
equilibrium. The ensuing displacement $x=x(t)$ from equilibrium ($x=0$) is
governed by $$mx'' = -ax\abs{x'}-kx, \qquad x(0) = 0,\quad x'(0) = V$$ where
$-ax\abs{x'}$ is a nonlinear damping force and $-kx$ is a linear rebound force
of the spring. What are possible length scales? If the restoring force is small
compared to the damping force, choose appropriate time and length scales and
non-dimensionalize the model so that the small term appears in the damping
force.

\subsubsection*{Remarks}
Find dimensions of constants $m$, $a$, $k$, $V$. For the small rebound force
scenario, find the time and length scales from $m$, $a$, $V$, and
non-dimensionalize the equation. Under what condition on $k$ would we expect the
rebound force term to be negligible?

\subsubsection*{Solution}
\todo
\subsubsection*{Check}
\todo

\subsection{9}
\subsubsection*{Problem}

The temperature $T=T(t)$ of a chemical sample in a furnace at time $t$ is
governed by the initial value problem
$$\od{T}{t}=qe^{-A/T}-k(T-T_f),\qquad T(0)=T_0,$$ where $T_0$ is the initial
temperature of the sample, $T_f$ is the temperature in the furnace, and $q$,
$k$, and $A$ are positive constants. The first term on the right side is the
heat generation term, and the seond is th heat loss term given by Newton's law
of cooling.
\begin{enumerate}
\item What are the dimensions of the constants $q$, $k$, and $A$?
\item Reduce the problem to dimensionless form using $T_f$ as the temperature
  scale and choose an appropriate time scale when the heat loss is large
  compared to the heat generated by the reaction.
\end{enumerate}

\subsubsection*{Solution}
\todo
\subsubsection*{Check}
\todo

\section{Programming Minilab}
\todo

\end{document}
