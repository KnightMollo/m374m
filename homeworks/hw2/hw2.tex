\documentclass[12pt]{article}
\title{M374M Homework 2 \\
  \normalsize{\S~1.2 \#1$^1$, 3$^2$, 4$^3$, 9}}
\author{Hershal Bhave (hb6279)}
\date{Due 2016-02-08}

\usepackage{homework-macros}
\tikzexternalize

\begin{document}
\maketitle
\section{\S~1.2}
\subsection{1$^1$}
\subsubsection*{Problem}
Let $u=u(t),\;0 \le t \le b$, be a given smooth function. If
$M=\text{max}\abs{u(t)}$, then $u$ can be scales by $M$ to obtain the
dimensionless dependent $U=u/M$. A time scale can be taken as
$t_c=M/\text{max}\abs{u'(t)}$, the ratio of the maximum slope. Find $M$ and
$t_c$ for the following functions:
\begin{enumerate}
\item $u(t)=A\sin\omega t, \quad t>0$.
\item $u(t)=Ae^{-\lambda t},\quad t>0$.
\item $u(t)=Ate^{-\lambda t},\quad 0 \le t \le 2/\lambda.$
\end{enumerate}

\subsubsection*{Remarks}
This problem illustrates an alternative way to identify scales. Assume $A$,
$\omega$, and $\lambda$ are positive constants.

\subsubsection*{Solution}
\begin{enumerate}
\item
  \begin{equation*} \boxed{
    \begin{aligned}
      M &= \text{max}\abs{A\sin\omega t} \\
      \implies M &= A \\
    \end{aligned}
    }
  \end{equation*}

  \begin{equation*} \boxed{
    \begin{aligned}
      t_c &= M/\text{max}\abs{u'(t)} \\
      &= A/\text{max}\abs{\omega A \cos \omega t} \\
      \implies t_c &= 1/\omega \\
    \end{aligned}
    }
  \end{equation*}
\item
  \begin{equation*} \boxed{
    \begin{aligned}
      M &= \text{max}\abs{Ae^{-\lambda t}} \\
      \implies M &= A \\
    \end{aligned}
    }
  \end{equation*}

  \begin{equation*} \boxed{
    \begin{aligned}
      t_c &= M/\text{max}\abs{u'(t)} \\
      &= A/\text{max}\abs{-A\lambda e^{-\lambda t}} \\
      \implies t_c &= 1/\lambda \\
    \end{aligned}
    }
  \end{equation*}

\item
  \begin{equation*} \boxed{
      \begin{aligned}
        M &= \text{max}\abs{Ate^{-\lambda t}} \\
        \implies M &= \frac{A}{\lambda e} \\
      \end{aligned}
    }
  \end{equation*}

  \begin{equation*} \boxed{
      \begin{aligned}
        t_c &= M/\text{max}\abs{u'(t)} \\
        &= M/\text{max}\abs{(1-\lambda t)Ae^{-\lambda t}} \\
        &= \frac{A}{\lambda e}\frac{1}{A} \\
        \implies t_c &= \frac{1}{\lambda e}
      \end{aligned}
    }
  \end{equation*}
\end{enumerate}

\newpage
\subsection{3$^2$}
\subsubsection*{Problem}
The growth rate of an organism is often measured using carbon biomass as the
``currency.'' The \textbf{von Bertalanffy growth model} is
\begin{equation}
  \label{eq:3-growth-model}
  m' = ax^2 - bx^3,
\end{equation}
where $m$ is its biomass, $x$ is some characteristic length of the organism, $a$
is its biomass assimilation rate, and $b$ is its biomass use rate. Thus, it
assimilates nutrients proportional to its area, and it uses nutrients
proportional to its volume. Assume
\begin{equation}
  \label{eq:3-biomass-model}
  m=\rho x^2
\end{equation}
and rewrite the model in terms
of the length $x$. Determine the dimensions of the constants $a$, $b$, and
$\rho$. Select time and length scales $\rho/b$ and $a/b$, repsectively, and
reduce the problem to dimensionless form. If $x(0)=0$, find the length $x$ at
time $t$. Does this seem like a reasonable model?

\subsubsection*{Remarks}
Assume $m$, $x$ are functions of $t$. The notation $m'$ means $\od{m}{t}$. As
time $t$ increases, show that the size $x$ of an organism does not grow without
bound, but instead approaches some maximum value.

\subsubsection*{Solution}
First we will obtain the dimensions of each quantity. It is assumed that $[x]=L$
and $[\od{x}{t}]=LT^{-1}$. We will obtain the dimensions of $a$ and $b$ from
\cref{eq:3-growth-model}. We will separate the terms in \cref{eq:3-growth-model} to
more clearly demonstrate the dimension breakdown.
\vspace{-2em}
\begin{multicols}{2}
  \begin{equation*}
    \begin{aligned}
      MT^{-1} &= [a]L^2 \\
      \implies [a] &= ML^{-2}T^{-1} \\
    \end{aligned}
  \end{equation*} \\
  \begin{equation*}
    \begin{aligned}
      MT^{-1} &= [b]L^3 \\
      \implies [b] &= ML^{-3}T^{-1} \\
    \end{aligned}
  \end{equation*}
\end{multicols}
\noindent
Similarly, we will obtain the dimensions of $\rho$ from \cref{eq:3-biomass-model}.
\begin{equation*}
  \begin{aligned}
    m &= \rho x^3 \\
    [m] &= [\rho] [x]^3 \\
    M &= [\rho] L^3 \\
    \implies [\rho] &= ML^{-3} \\
  \end{aligned}
\end{equation*}
Now we will rewrite the model in terms of length $x$ by differentiating \cref{eq:3-biomass-model} and
setting it equal to the \cref{eq:3-growth-model}.
\begin{equation}
  \begin{aligned}
    \label{eq:model-in-terms-of-x}
    \od{}{t}\rho x^3 &= ax^2-bx^3 \\
    3\rho x^2\od{x}{t} &= ax^2-bx^3 \\
    \od{x}{t} &= \frac{a}{3\rho} - \frac{bx}{3\rho} \\
  \end{aligned}
\end{equation}
We are given scales $t_c=\rho/b$ and $x_c=a/b$. From this we may construct a
dimensionless model. Assuming $\bar{t}=t/t_c=bt/\rho$ and $\bar{x}=x/x_c=bx/a$, we may use
the chain rule.
\begin{equation}
  \label{eq:chain-rule-setup}
  \od{\bar{x}}{\bar{t}} = \od{\bar{x}}{t} \cdot \od{t}{\bar{t}}
\end{equation}
Differentiating $\bar{x}$ in terms of $t$ and differentiating $t$ in terms of
$\bar{t}$ to yields the first and second terms, respectively so that
\cref{eq:chain-rule-setup} becomes
\begin{equation}
  \label{eq:dxbar-in-terms-of-dx}
  \begin{aligned}
    \od{\bar{x}}{\bar{t}} &= \left(\frac{b}{a}\od{x}{t}\right)\cdot\left(\frac{\rho}{b}\right) \\
    &= \frac{\rho}{a}\od{x}{t} \\
  \end{aligned}
\end{equation}
We may substitute in $\od{x}{t}$ from \cref{eq:model-in-terms-of-x} into
\cref{eq:dxbar-in-terms-of-dx}.
\begin{equation}
  \label{eq:dimensionless-model-in-terms-of-x}
  \begin{aligned}
    \od{\bar{x}}{\bar{t}} &= \frac{\rho}{a}\left(\frac{a}{3\rho} -
    \frac{bx}{3\rho}\right) \\ &= \left(\frac{a\rho}{3a\rho} -
    \frac{bx\rho}{3a\rho}\right) \\
    &= \left(\frac{\cancel{a}\cancel{\rho}}{3\cancel{a}\cancel{\rho}} -
    \frac{bx\cancel{\rho}}{3a\cancel{\rho}}\right) \\
    &= \frac{1}{3}\left(1-\frac{bx}{a}\right) \\
  \end{aligned}
\end{equation}
Substituting $x$ in \cref{eq:dimensionless-model-in-terms-of-x}
\begin{equation}
  \label{eq:dimensionless-model-in-terms-of-x-bar}
  \begin{aligned}
    \od{\bar{x}}{\bar{t}} &= \frac{1}{3}(1-\frac{b}{a}\frac{a\bar{x}}{b}) \\
    &=
    \frac{1}{3}(1-\frac{\cancel{b}}{\cancel{a}}\frac{\cancel{a}\bar{x}}{\cancel{b}})
    \\
    \implies \od{\bar{x}}{\bar{t}} &= \frac{1}{3}(1-\bar{x}) \\
  \end{aligned}
\end{equation}
We may now solve the IVP from \cref{eq:dimensionless-model-in-terms-of-x-bar} to
obtain the dimensionless version of \cref{eq:model-in-terms-of-x}.
\begin{equation*}
  \begin{aligned}
    \od{\bar{x}}{\bar{t}} &= \frac{1}{3}(1-\bar{x}) \\
    \frac{\dd{\bar{x}}}{1-\bar{x}} &= \frac{1}{3} \dd{\bar{t}} \\
    \int \frac{\dd{\bar{x}}}{1-\bar{x}} &= \int \frac{1}{3} \dd{\bar{t}} \\
    -\ln (1-\bar{x}) &= \frac{\bar{t}}{3} + C \\
    \ln (1-\bar{x}) &= -\frac{\bar{t}}{3} + C \\
    1-\bar{x} &= e^{-\bar{t}/3+C} \\
    &= Ce^{-\bar{t}/3} \\
    \implies \bar{x} &= 1-Ce^{-\bar{t}/3}
  \end{aligned}
\end{equation*}
The initial condition $x(0)=0$ is equivalent to the dimensionless version
$\bar{x}(0)=0$.
\begin{equation*}
  \begin{aligned}
    \bar{x}(0) &= 1-Ce^{0} \\
    0 &= 1-C \\
    \implies C &= 1
  \end{aligned}
\end{equation*}
From these results we may conclude that the dimensionless form of the organism's
characteristic length $x$ in terms of time $t$ is
\begin{equation*} \boxed{
    \bar{x} = 1-e^{-\bar{t}/3}
  }
\end{equation*}
The model seems reasonable since this equation is bounded, i.e. the organism
does not grow infinitely.

%% \subsubsection*{Check}
%% We will verify that the obtained dimensions are accurate.

%% \begin{equation*}
%%   \begin{aligned}
%%     \od{m}{t} &= ax^2 - bx^3 \\
%%     \left[\od{m}{t}\right] &= [a][x]^2 - [b][x]^3 \\
%%     MT^{-1} &= (ML^{-2}T^{-1})(L^2) - (ML^{-3}T^{-1})(L^3) \\
%%     &= (M\cancel{L^{-2}}T^{-1})\cancel{(L^2)} - (M\cancel{L^{-3}}T^{-1})\cancel{(L^3)} \\
%%     &= MT^{-1} - MT^{-1} \\
%%     \implies MT^{-1} &= MT^{-1} \quad\checkmark
%%   \end{aligned}
%% \end{equation*}

%% \begin{equation*}
%%   \begin{aligned}
%%     m &= \rho x^3 \\
%%     [m] &= [\rho] [x]^3 \\
%%     M &= ML^{-3} L^{3} \\
%%     &= M\cancel{L^{-3}} \cancel{L^{3}} \\
%%     \implies M &= M \quad\checkmark
%%   \end{aligned}
%% \end{equation*}

\newpage
\subsection{4$^3$}
\subsubsection*{Problem}
A mass hanging on a spring is given a positive initial velocity $V$ from
equilibrium. The ensuing displacement $x=x(t)$ from equilibrium ($x=0$) is
governed by
\begin{equation}
  \label{eq:4-problem}
  mx'' = -ax\abs{x'}-kx, \quad x(0) = 0,\quad x'(0) = V
\end{equation}
where $-ax\abs{x'}$ is a nonlinear damping force and $-kx$ is a linear rebound
force of the spring. What are possible length scales? If the rebound force is
small compared to the damping force, choose appropriate time and length scales
and non-dimensionalize the model so that the small term appears in the damping
force.

\subsubsection*{Remarks}
Find dimensions of constants $m$, $a$, $k$, $V$. For the small rebound force
scenario, find the time and length scales from $m$, $a$, $V$, and
non-dimensionalize the equation. Under what condition on $k$ would we expect the
rebound force term to be negligible?

\subsubsection*{Solution}
The dimensions of each variable are determined by inspection and are represented
in \cref{fig:4-var-mappings}.
  \begin{figure}
    \centering
    \begin{tabularx}{0.5\textwidth}{XXX}
      Variable & Dimension & Exponent \\ \hline
      $m$ & $M$ & $a$ \\
      $V$ & $L^{-1}T^{-1}$ & $b$ \\
      $a$ & $ML^{-1}T^{-2}$ & $c$ \\
      $k$ & $MT^{-2}$ & $d$ \\
      $x$ & $L$ &  \\
      $\od{x}{t}$ & $ML^{-3}$ &  \\
      $\od[2]{x}{t}$ & $ML^{2}T^{-2}$ & \\
    \end{tabularx}
    \caption{Variable mappings for \#4}
    \label{fig:4-var-mappings}
  \end{figure}
We will first determine appropriate time and length scales. The length scales
are found by solving $L=m^aV^ba^ck^d$.
\begin{equation*}
  \begin{aligned}
    L &= m^aV^ba^ck^d \\
    &= M^{a}(LT^{-1})^b(ML^{-1}T^{-1})^c(MT^{-2})^d \\
    &= M^{a+c+d}L^{b-c}T^{-b-c-2d} \\
  \end{aligned}
\end{equation*}
This yields the following linear system
\begin{equation*}
  \begin{aligned}
    0 &= a + c + d \\
    1 &= b - c \\
    0 &= -b  - c - 2d \\
  \end{aligned}
\end{equation*}
Written in augmented matrix form
\begin{equation*}
  \begin{pmatrix}[cccc|c]
    1 & 0 & 1 & 1 & 0 \\
    0 & 1 & -1 & 0 & 1 \\
    0 & -1 & -1 & -2 & 0 \\
  \end{pmatrix}
\end{equation*}
Which may be reduced to echelon form
\begin{equation*}
  \begin{pmatrix}[cccc|c]
    1 & 0 & 0 & 0 & 1/2 \\
    0 & 1 & 0 & 1 & 1/2 \\
    0 & 0 & 1 & 1 & -1/2 \\
  \end{pmatrix}
\end{equation*}
Solving this system yields
\begin{equation*}
  \begin{pmatrix}
    a \\ b \\ c \\ d
  \end{pmatrix} =
  \begin{pmatrix}
    1/2 \\ 1/2 \\ -1/2 \\ 0
  \end{pmatrix} +
  \begin{pmatrix}
    0 \\ -1 \\ -1 \\ 1
  \end{pmatrix}
  d
\end{equation*}
Since the rebound force is small compared to the damping force, we'd like to
avoid terms involving $k$, so we pick the case when $d=0$. Our length scale
turns into
\begin{equation}
  \label{eq:4-length-scale}
  \implies x_c = \sqrt{\frac{mV}{a}}
\end{equation}
We do the same approach to obtain the time scales, this time solving for
$T=m^aV^ba^ck^d$.
\begin{equation*}
  \begin{aligned}
    T &= m^aV^ba^ck^d \\
    &= M^{a}(LT^{-1})^b(ML^{-1}T^{-1})^c(MT^{-2})^d \\
    &= M^{a+c+d}L^{b-c}T^{-b-c-2d} \\
  \end{aligned}
\end{equation*}
This yields the following system
\begin{equation*}
  \begin{aligned}
    0 &= a + c + d \\
    0 &= b - c \\
    1 &= -b  - c - 2d \\
  \end{aligned}
\end{equation*}
Written in augmented matrix form
\begin{equation*}
  \begin{pmatrix}[cccc|c]
    1 & 0 & 1 & 1 & 0 \\
    0 & 1 & -1 & 0 & 0 \\
    0 & -1 & -1 & -2 & 1 \\
  \end{pmatrix}
\end{equation*}
Which may be reduced to echelon form
\begin{equation*}
  \begin{pmatrix}[cccc|c]
    1 & 0 & 0 & 0 & 1/2 \\
    0 & 1 & 0 & 1 & -1/2 \\
    0 & 0 & 1 & 1 & -1/2 \\
  \end{pmatrix}
\end{equation*}
Solving this system yields
\begin{equation*}
  \begin{pmatrix}
    a \\ b \\ c \\ d
  \end{pmatrix} =
  \begin{pmatrix}
    1/2 \\ -1/2 \\ -1/2 \\ 0
  \end{pmatrix} +
  \begin{pmatrix}
    0 \\ -1 \\ -1 \\ 1
  \end{pmatrix}
  d
\end{equation*}
Similar to the length scales, we pick the case when $d=0$ so that our time scale
turns into
\begin{equation}
  \label{eq:4-time-scale}
  \implies t_c = \sqrt{\frac{m}{aV}}
\end{equation}
Thus the dimensionless forms of the dependent and independent forms of $x$ and
$t$ become
\begin{equation*}
  \bar{x} = \frac{x}{\sqrt{\frac{mV}{a}}}, \quad
  \bar{t} = \frac{t}{\sqrt{\frac{m}{Va}}}
\end{equation*}
We'll use the scales from \cref{eq:4-length-scale,eq:4-time-scale} to construct
a dimensionless version of \cref{eq:4-problem}. We begin by obtaining the first
derivitive of $\bar{x}$ in terms of $\bar{t}$.
\begin{equation}
  \label{eq:4-dxbar-in-terms-of-dx}
  \begin{aligned}
    \od{\bar{x}}{\bar{t}} &= \od{\bar{x}}{t}\od{t}{\bar{t}} \\
    &= \frac{1}{\sqrt{\frac{mV}{a}}}\od{x}{t}\od{t}{\bar{t}} \\
    &= \frac{1}{\sqrt{\frac{mV}{a}}}\sqrt{\frac{m}{vA}}\od{x}{t} \\
    &= \sqrt{\frac{a}{mV}\frac{m}{Va}} \od{x}{t} \\
    &= \sqrt{\frac{\cancel{a}}{\cancel{m}V}\frac{\cancel{m}}{V\cancel{a}}}
    \od{x}{t} \\
    \od{\bar{x}}{\bar{t}} &= \frac{1}{V}\od{x}{t} \\
  \end{aligned}
\end{equation}
Deriving \cref{eq:4-dxbar-in-terms-of-dx}
\begin{equation}
  \label{eq:4-d2xbar-in-terms-of-d2x}
  \begin{aligned}
    \od{}{\bar{t}}\od{\bar{x}}{\bar{t}} &= \od{}{\bar{t}} \left(\frac{1}{V}\od{x}{t}\right) \\
    \od[2]{\bar{x}}{\bar{t}} &= \od{}{t}
    \left(\frac{1}{V}\od{x}{t}\right) \od{t}{\bar{t}} \\
    &= \frac{1}{V} \left(\od[2]{x}{t}\right) \sqrt{\frac{m}{Va}} \\
    \od[2]{\bar{x}}{\bar{t}} &= \frac{1}{V}\sqrt{\frac{m}{Va}} \left(\od[2]{x}{t}\right) \\
  \end{aligned}
\end{equation}
We may non-dimensionalize the variables within \cref{eq:4-problem}
\begin{equation}
  \label{eq:4-equation}
  \begin{aligned}
    \od[2]{x}{t} &= -\frac{ax}{m} \abs{\od{x}{t}} - \frac{kx}{m} \\
    \od[2]{x}{t} &= -\frac{a}{m}\left( \sqrt{\frac{mV}{a}}\bar{x}\right)
    \left( \abs{V\od{\bar{x}}{\bar{t}}}\right) -
    \frac{k\bar{x}}{m}\sqrt{\frac{mV}{a}} \\
  \end{aligned}
\end{equation}
and insert it into \cref{eq:4-d2xbar-in-terms-of-d2x}
\begin{equation*}
  \begin{aligned}
    \od[2]{\bar{x}}{\bar{t}} &= \frac{1}{V}\sqrt{\frac{m}{Va}} \left(
    -\frac{a}{m}\left( \sqrt{\frac{mV}{a}}\bar{x}\right) \left(
    \abs{V\od{\bar{x}}{\bar{t}}}\right) -
    \frac{k\bar{x}}{m}\sqrt{\frac{mV}{a}}\right) \\
    &= \left(\frac{1}{V}\sqrt{\frac{m}{Va}}\right)\frac{-a}{m}\sqrt{\frac{mV}{a}}\bar{x}
    \abs{V\od{\bar{x}}{\bar{t}}} -
    \left(\frac{1}{V}\sqrt{\frac{m}{Va}}\right)\frac{k\bar{x}}{m}\sqrt{\frac{mV}{a}} \\
    &= \left(\cancel{\frac{1}{V}}\sqrt{\frac{m}{\cancel{V}a}}\right)\frac{-a}{m}\sqrt{\frac{m\cancel{V}}{a}}\bar{x}
    \abs{\cancel{V}\od{\bar{x}}{\bar{t}}} -
    \left(\frac{1}{V}\sqrt{\frac{m}{\cancel{V}a}}\right)\frac{k\bar{x}}{m}\sqrt{\frac{m\cancel{V}}{a}} \\
    &= -\cancel{\frac{a}{m}}\cancel{\frac{m}{a}}\bar{x}\od{\bar{x}}{\bar{t}} -
    \frac{\cancel{m}}{Va}\frac{k\bar{x}}{\cancel{m}} \\
    \implies \od[2]{\bar{x}}{\bar{t}} &= -\bar{x}\od{\bar{x}}{\bar{t}} - \frac{k\bar{x}}{Va} \\
  \end{aligned}
\end{equation*}
We may scale the initial conditions based on \cref{eq:4-dxbar-in-terms-of-dx}.
\begin{equation*}
  \begin{aligned}
    \od{\bar{x}}{\bar{t}}(0) &= \frac{1}{V}\od{x}{t}(0) \\
     &= \frac{1}{V}V \\
    \implies \od{\bar{x}}{\bar{t}}(0) &= 1 \\
  \end{aligned}
\end{equation*}
\begin{equation}
  \begin{aligned}
    \bar{x}(0) &= \frac{0}{\sqrt{\frac{mV}{a}}} \\
    \implies \bar{x}(0) &= 0 \\
  \end{aligned}
\end{equation}
\begin{equation*} \boxed{
    \od[2]{\bar{x}}{\bar{t}} = -\bar{x}\od{\bar{x}}{\bar{t}} - \beta\bar{x},
    \quad\beta=\frac{k}{Va},\quad \bar{x}(0)=0,
    \quad\od{\bar{x}}{\bar{t}}(0) = 1
  }
\end{equation*}
We expect the rebound force term to be small when $\beta\ll 1$.

\newpage
\subsection{9}
\subsubsection*{Problem}

The temperature $T=T(t)$ of a chemical sample in a furnace at time $t$ is
governed by the initial value problem
\begin{equation}
  \label{eq:9-problem}
  \od{T}{t}=qe^{-A/T}-k(T-T_f),\quad T(0)=T_0,
\end{equation}
where $T_0$ is the initial temperature of the sample, $T_f$ is the temperature
in the furnace, and $q$, $k$, and $A$ are positive constants. The first term on
the right side is the heat generation term, and the seond is the heat loss term
given by Newton's law of cooling.
\begin{enumerate}
\item What are the dimensions of the constants $q$, $k$, and $A$?
\item Reduce the problem to dimensionless form using $T_f$ as the temperature
  scale and choose an appropriate time scale when the heat loss is large
  compared to the heat generated by the reaction.
\end{enumerate}

\subsubsection*{Solution}
\begin{enumerate}
\item The variable mappings have been determined by inspection and are shown in
  \cref{fig:9-var-mappings}.

  \begin{figure}
    \centering
    \begin{tabularx}{0.5\textwidth}{XXX}
      Variable & Dimension & Exponent \\ \hline
      $k$ & $T^{-1}$ & $a$ \\
      $q$ & $\Theta T^{-1}$ & $b$ \\
      $A$ & $\Theta$ & $c$ \\
      $T_f$ & $\Theta$ & $d$ \\
    \end{tabularx}
    \caption{Variable mappings for \#9}
    \label{fig:9-var-mappings}
  \end{figure}

\item
  We are given the temperature scale so that $\bar{T}=T/T_f$. We will determine
  appropriate length scales when the heat loss is large compared to the heat
  generated.
  \begin{equation*}
    \begin{aligned}
      T &= k^aq^bA^cT_f^d \\
      &= (T^{-1})^a(\Theta T^{-1})^b(\Theta)^c(\Theta)^d \\
      &= T^{-a-b}\Theta^{b+c+d}
    \end{aligned}
  \end{equation*}
  This yields the linear system
  \begin{equation*}
    \begin{aligned}
      1 &= -a - b \\
      0 &= b + c + d \\
    \end{aligned}
  \end{equation*}
  which may be written in augmented matrix form
  \begin{equation*}
    \begin{pmatrix}[cccc|c]
      -1 & -1 & 0 & 0 & 1 \\
      0 & 1 & 1 & 1 & 0 \\
    \end{pmatrix}
  \end{equation*}
  and reduced into echelon form
  \begin{equation*}
    \begin{pmatrix}[cccc|c]
      1 & 0 & -1 & -1 & -1 \\
      0 & 1 & 1 & 1 & 0 \\
    \end{pmatrix}.
  \end{equation*}
  Solving this system yields
  \begin{equation*}
    \begin{pmatrix}
      a \\ b \\ c \\ d \\
    \end{pmatrix} =
    \begin{pmatrix}
      -1 \\ 0 \\ 0 \\ 0 \\
    \end{pmatrix} +
    \begin{pmatrix}
      1 \\ -1 \\ 1 \\ 0 \\
    \end{pmatrix}c +
    \begin{pmatrix}
      1 \\ -1 \\ 0 \\ 1 \\
    \end{pmatrix}d.
  \end{equation*}
  If we assume the heat loss is large compared to the heat generated, then we
  would like to choose scales which involve $k$ and $T_f$. We will pick the case
  when $(c,d)=(0,0)$.
  \begin{equation}
    \label{eq:9-t-scale}
    \implies t_c  = \frac{1}{k}
  \end{equation}
  Additionally, we are given
  \begin{equation}
    \label{eq:9-T-scale}
    \implies T_c  = T_f.
  \end{equation}
  Now we may use the scales obtained in \cref{eq:9-t-scale,eq:9-T-scale} to
  non-dimensionalize \cref{eq:9-problem}.
  \begin{equation*}
    \begin{aligned}
      \od{\bar{T}}{\bar{t}} &= \frac{t_c}{T_c}\od{T}{t} \\
      &= \frac{1}{kT_f}(qe^{-A/\bar{T}T_f} - k(\bar{T}T_f-T_f)) \\
      &= \frac{1}{kT_f}(qe^{-A/\bar{T}T_f} - kT_f(\bar{T}-1)) \\
      &= \frac{q}{kT_f}e^{-A/\bar{T}T_f} - (\bar{T}-1)
    \end{aligned}
  \end{equation*}
  \begin{equation*} \boxed{
      \implies \od{\bar{T}}{\bar{t}} =
      e^{-\alpha/(\bar{T})}-\beta(\bar{T}-1), \quad \alpha = \frac{A}{T_f},
      \quad \beta = \frac{kT_f}{q}, \quad T(0) = T_0/T_f.
    }
  \end{equation*}
\end{enumerate}

\skiptooddpage{}
\section{Programming Minilab}
A model for the mass concentration of a chemical \textbf{C} in a tank reactor is
\begin{equation}
  \label{eq:minilab-tank}
  \dot{c} = \frac{\eta}{V}(c_{\text{in}}-c)-\gamma c^2,\quad c(0)=0.
\end{equation}
Here $c(t)$ is the average concentration of \textbf{C} in the reactor and its
output, $c_{\text{in}}$ is the concentration of the input, $V$ is the volume of
the reactor, $\eta$ is the volume flow rate, and $\gamma$ is a constant that
describes the reaction rate. When $\gamma$ is ``small'', we expect the nonlinear
term to be negligible, so that \cref{eq:minilab-tank} may be approximated by
\begin{equation}
  \label{eq:minilab-tank-approx}
  \dot{c} = \frac{\eta}{V}(c_{\text{in}}-c),\quad c(0)=0.
\end{equation}
Here we derive a condition to determine what it means for $\gamma$ to be
``small'' in this sense. Relevant physical dimensions are indicated in
\cref{fig:minilab-var-mappings}. Use the units of kilograms, meters, and
seconds.

\begin{figure}
  \centering
  \begin{tabularx}{0.5\textwidth}{XXX}
    Variable & Dimension & Exponent \\ \hline
    $\eta$ & $L^3T^{-1}$ & $a$ \\
    $V$ & $L^3$ & $b$ \\
    $c_{\text{in}}$ & $ML^{-3}$ & $c$ \\
    $\gamma$ & $L^3M^{-1}T^{-1}$ & $d$ \\
  \end{tabularx}
  \caption{Variable mappings for \#9}
  \label{fig:minilab-var-mappings}
\end{figure}

\subsection{}
\label{sec:minilab-first-problem}
\subsubsection*{Problem}
In the case when $\gamma$ is ``small'', or actually zero, the dominant effects
in \cref{eq:minilab-tank} are due to $\eta$, $V$, and $c_{\text{in}}$. Use these
quantities to find a characteristic time scale $t_c$ and concentration scale
$c_c$. Rewrite \cref{eq:minilab-tank} in dimensionless form.

\subsubsection*{Solution}
We'll first find a possible concentration scale using the mappings from
\cref{fig:minilab-var-mappings}.
\begin{equation*}
  \begin{aligned}
    ML^{-3} &= \gamma^aV^bc_{\text{in}}^c\gamma^d \\
    &= M^{c-d}L^{3a+3b-3c+3d}T^{-a-d} \\
  \end{aligned}
\end{equation*}
This yeilds the linear system in augmented matrix form
\begin{equation*}
  \begin{pmatrix}[cccc|c]
    1 & 0 & 0 & 1 & 0 \\
    3 & 3 & -3 & 3 & -3 \\
    0 & 0 & 1 & -1 & 1 \\
  \end{pmatrix}
\end{equation*}
which may be rewritten in echelon form
\begin{equation*}
  \begin{pmatrix}
    1 & 0 & 0 & 1 & 0 \\
    0 & 1 & 0 & -1 & 0 \\
    0 & 0 & 1 & -1 & 1 \\
  \end{pmatrix}.
\end{equation*}
Solving this system results in
\begin{equation*}
  \begin{pmatrix}
    a \\ b \\ c \\ d \\
  \end{pmatrix} =
  \begin{pmatrix}
    0 \\ 0 \\ 1 \\ 0 \\
  \end{pmatrix} +
  \begin{pmatrix}
    -1 \\ 1 \\ 1 \\ 1 \\
  \end{pmatrix}d
\end{equation*}
Since we do not want to involve $\gamma$ in our scales, we'll use the case when
$d=0$.
\begin{equation*}
  \implies c_c = c_{\text{in}}
\end{equation*}
We'll do the same to obtain the scales for time.
\begin{equation*}
  \begin{aligned}
    T &= \gamma^aV^bc_{\text{in}}^c\gamma^d \\
    &= M^{c-d}L^{3a+3b-3c+3d}T^{-a-d} \\
  \end{aligned}
\end{equation*}
This yeilds the linear system in augmented matrix form
\begin{equation*}
  \begin{pmatrix}[cccc|c]
    1 & 0 & 0 & 1 & 1 \\
    3 & 3 & -3 & 3 & 0 \\
    0 & 0 & 1 & -1 & 0 \\
  \end{pmatrix}
\end{equation*}
which may be rewritten in echelon form
\begin{equation*}
  \begin{pmatrix}
    1 & 0 & 0 & 1 & 1 \\
    0 & 1 & 0 & -1 & -1 \\
    0 & 0 & 1 & -1 & 0 \\
  \end{pmatrix}.
\end{equation*}
Solving this system results in
\begin{equation*}
  \begin{pmatrix}
    a \\ b \\ c \\ d \\
  \end{pmatrix} =
  \begin{pmatrix}
    1 \\ -1 \\ 0 \\ 0 \\
  \end{pmatrix} +
  \begin{pmatrix}
    -1 \\ 1 \\ 1 \\ 1 \\
  \end{pmatrix}d
\end{equation*}
Again, since we do not want to involve $\gamma$ in scales, we'll use the case
when $d=0$.
\begin{equation*}
  \implies t_c = \frac{V}{\eta}
\end{equation*}
Now we may rewrite \cref{eq:minilab-tank} in dimensionless form
\begin{equation*}
  \begin{aligned}
    \od{\bar{c}}{\bar{t}} &=
    \frac{t_c}{c_c}\od{c}{t} \\
    %% &= \frac{V}{\eta c_{\text{in}}} \left(\frac{\eta}{V}(c_{\text{in}}-c)-\gamma
    %% c^2\right) \\
    &= \frac{V}{\eta c_{\text{in}}}
    \left(\frac{\eta}{V}(c_{\text{in}}-\bar{c}c_{\text{in}})-\gamma
    (\bar{c}c_{\text{in}})^2\right) \\
    &= \frac{V}{\eta c_{\text{in}}}\frac{\eta c_{\text{in}}}{V} (1-\bar{c}) -
    \frac{V\gamma c_{\text{in}}^2}{\eta c_{\text{in}}}\bar{c}^2 \\
    &= \cancel{\frac{V}{\eta c_{\text{in}}}} \cancel{\frac{\eta c_{\text{in}}}{V}}
    (1-\bar{c}) -
    \frac{V \gamma c_{\text{in}}^{\cancel{2}}}{\eta \cancel{c_{\text{in}}}}\bar{c}^2 \\
    \implies \od{\bar{c}}{\bar{t}}
    &= 1-\bar{c} - \frac{V \gamma c_{\text{in}}}{\eta}\bar{c}^2 \\
  \end{aligned}
\end{equation*}
Or more concisely,
\begin{equation*} \boxed{
    \od{\bar{c}}{\bar{t}} = (1 - \bar{c}) - \beta\bar{c}^2,
    \quad \beta = \frac{V \gamma c_{\text{in}}}{\eta} \quad c(0)=0.
    }
\end{equation*}

\subsection{}
\label{sec:minilab-second-problem}
\subsubsection*{Problem}
Using the result from \cref{sec:minilab-first-problem}, identify a condition on
$\gamma$ under which the nonlinear term is expected to be negligible. In which
of the two operating scenarios
$(c_{\text{in}}\,,\eta\,,V\,,\gamma) = (10^{-2}\,,10^{-2}\,,10\,,10^{-1})$ or
$(c_{\text{in}}\,,\eta\,,V\,,\gamma) = (10^{-2}\,,10^{-1}\,,1\,,10^{-1})$
is this condition reasonably met? That is, in which scenario can $\gamma$ be
regarded as ``small''?

\subsubsection*{Solution}
When the reaction rate is small compared to the flow rate, we may regard $\beta
\ll 1$. This implies that $\gamma \ll \frac{\eta}{Vc_\text{in}}$. The second
operating scenario allows this condition to be met, while the first scenario
does not. Therefore, $\gamma$ may be considered ``small'' under the second
scenario.

\subsection{}
\subsubsection*{Problem}
For the first operating scenario in \cref{sec:minilab-second-problem}, simulate
the systems in \cref{eq:minilab-tank,eq:minilab-tank-approx} for
$t\in[0,10\,V/\eta]$ and superimose plots of $c$ versus $t$ for the two systems.
Repeat for the second scenario. Do the simulations agree with the analysis in
\cref{sec:minilab-second-problem}? Specifically, does the nonlinear term seem
to have a negligible or small effect in one of the scenarios as predicted?

\subsubsection*{Solution}
The simulations agree with the analysis in \cref{sec:minilab-second-problem}, as
the nonlinear term seems to have negligible effects on the solution of the
second scenario, as predicted. Reference
\cref{fig:minilab-scenario-one,fig:minilab-scenario-two}.
\end{document}
