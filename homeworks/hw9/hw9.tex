\documentclass[12pt,twoside]{article}
\title{M374M Homework 9 \\
  \normalsize{\S~3.3 \#1$^1$g, \S~3.4 3$^1$} \\
  Revision: \input{revision}}
\author{Hershal Bhave (hb6279)}
\date{Due 2016--04--15}

\usepackage{homework-macros}
\tikzexternalize%

\begin{document}
\maketitle

\section{\S~3.3}
\subsection{1$^1$g}
\subsubsection*{Problem}
Use singular perturbation methods to obtain a uniform approximate solution to
\begin{equation}
  \label{eq:3.3.1g-problem}
  \epsilon y''+2y'+e^y=0,\quad y(0)=y(1)=0.
\end{equation}
Assume that $0<\epsilon\ll1$ and $0<x<1$.
\subsubsection*{Remarks}
Assume boundary layer is on left side of interval and find leading-order
composite approximation.
\subsubsection*{Solution}
\todo{}

\section{\S~3.4}
\subsection{3$^1$}
\subsubsection*{Problem}
Find a uniformly valid approximation to the problem
\begin{equation}
  \label{eq:3.4.3-problem}
  \begin{aligned}
    \epsilon y''+{(t+1)}^2y'&=1, &\quad y>0,\;0<\epsilon\ll1,\\
    y(y)&=1, &\quad \epsilon y'(0)=1 \\
  \end{aligned}
\end{equation}
\subsubsection*{Remarks}
Assume boundary layer is on left side of interval and find leading-order
composite approximation.
\subsubsection*{Solution}
\todo{}

\section{Programming Minilab}
Due to surface tension, a liquid-gas interface will rise up or dip down at a
solid boundary to form a meniscus, which is responsible for the so-called
capillary effect in narrow tubes. In the planar case, the shape of the interface
or meniscus curve $y(x)$ is described by
\begin{equation}
  \label{eq:minilab-problem}
  \begin{aligned}
    \frac{\sigma}{\rho g}\od[2]{y}{x} = y {\left[ 1 + {\left( \od{y}{x}
          \right)}^2 \right]}^{3/2},\quad 0\le x\le L \\
    \od{y}{x}(0) = 0,\quad \od{y}{x}(L) = \tan\gamma.
  \end{aligned}
\end{equation}
where $\rho$ is the density of the liquid, $g$ is the gravitational
acceleration, and $\rho$ and $\gamma$ are the surface tension and wetting angle
constants associated with the interface. Here we develop an approximate solution
of \cref{eq:minilab-problem} under the assumption that $0<\frac{\sigma}{\rho
  g}\ll L^2$. For this problem we expect a boundary layer at $x=L$, and we will
see that $\od{y}{x}$ is regular, by $\od[2]{y}{x}$ behaves singularly.

\subsection{a}
\label{sec:minilab-part-a}
\subsubsection*{Problem}
introduce the dimensionless variables $h=y/L$ and $s=x/L$ and show that
\cref{eq:minilab-problem} can be written in the following form for an
appropriate parameter $0<\epsilon\ll1$.
\begin{equation}
  \label{eq:minilab-problem-part-a}
  \begin{aligned}
    \epsilon\od[2]{h}{s} = h{\left[ 1+{\left( \od{h}{s} \right)}^2
      \right]}^{3/2},\quad 0\le s\le1, \\
    \od{h}{s}(0) = 0,\quad \od{h}{s}(1) = \tan\gamma
  \end{aligned}
\end{equation}
\subsubsection*{Solution}
\todo{}

\subsection{b}
\label{sec:minilab-part-b}
\subsubsection*{Problem}
Find the leading-order approximation.
\subsubsection*{Solution}
\todo{}

\subsection{c}
\label{sec:minilab-part-c}
\subsubsection*{Problem}
For the inner approximation, consider the change of variables
$\tau=\epsilon^{-\alpha}(s-a)$ and $u=\epsilon^{-\beta}h$. Show that
$\alpha=1/2$ and $\beta=1/2$ leads to the regular system
\begin{equation}
  \label{eq:minilab-problem-part-b}
  \od[2]{u}{\tau} = u{\left[ 1+{\left( \od{u}{\tau} \right)}^2 \right]}^{3/2},
  \quad \tau\le0,\quad \od{u}{\tau}=\tan\gamma.
\end{equation}

\subsubsection*{Solution}
\todo{}

\subsection{d}
\label{sec:minilab-part-d}
\subsubsection*{Problem}
By matching the results from \cref{sec:minilab-part-b,sec:minilab-part-c},
construct a leading-order composite approximation for $h(s)$, and hence fro the
meniscus curve $y(x)$. Use the approximation to find an expression for the
meniscus height $\ell=y(L)$. Show that $\od{y}{x}$ remains bounded at all
points as $\epsilon\downarrow0$, but that $\od[2]{y}{x}$ grows unbounded at
$x=L$ as $\epsilon\downarrow0$.
\subsubsection*{Solution}
\todo{}

\subsection{e}
\subsubsection*{Problem}
Simulate the equations in \cref{eq:minilab-problem} for $\sigma=0.05$,
$\rho=1000$, $g=10$, $l=0.07$, and $\gamma=\pi/3$ in kg-m-s units. Superimpose
plots of the curve $y(x)$ produced by the simulation and by your approximation
from \cref{sec:minilab-part-d}. Is your prediction of the meniscus height $\ell$
in (approximate) agreement with the simulation?
\subsubsection*{Solution}
\todo{}

\end{document}