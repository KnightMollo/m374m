\documentclass[12pt,twoside]{article}
\title{M374M Homework 9 \\
  \normalsize{\S~3.3 \#1$^1$g, \S~3.4 \#3$^1$} \\
  Revision: \input{revision}}
\author{Hershal Bhave (hb6279)}
\date{Due 2016--04--15}

\usepackage{homework-macros}
\tikzexternalize%

\begin{document}
\maketitle

\section{\S~3.3}
\subsection{1$^1$g}
\subsubsection*{Problem}
Use singular perturbation methods to obtain a uniform approximate solution to
\begin{equation}
  \label{eq:3.3.1g-problem}
  \epsilon y''+2y'+e^y=0,\quad y(0)=y(1)=0.
\end{equation}
Assume that $0<\epsilon\ll1$ and $0<x<1$.
\subsubsection*{Remarks}
Assume boundary layer is on left side of interval and find leading-order
composite approximation.
\subsubsection*{Solution}
First we'll obtain the outer approximation. We'll collect coefficients of
$\epsilon^0$ and solve the resulting ODE.\@
\begin{equation*}
  \begin{aligned}
    2y'+e^y&=0 \\
    2\od{y}{x}+e^y&=0 \\
    2\od{y}{x} &= -e^y \\
    \int -e^{-y} \dd{y} &= \int\frac{1}{2} \dd{t} \\
    e^{-y} &= \frac{t}{2}+c \\
    \ln e^{-y} &= \ln\left( \frac{t}{2}+c \right) \\
    -y &= \ln\left( \frac{t}{2}+c \right) \\
    \implies y &= -\ln\left( \frac{t}{2}+c \right).
  \end{aligned}
\end{equation*}
Invoking the initial condition on the outer boundary resolves $c$:
\begin{equation*}
  \begin{aligned}
    y(1) &= \ln\left( \frac{1}{2}+c \right) \\
    0 &= \ln\left( \frac{1}{2}+c \right) \\
    \implies c &= \frac{1}{2}.
  \end{aligned}
\end{equation*}
So the outer boundary approximation is
\begin{equation*}
  \boxed{y_0(x) = -\ln\left( \frac{t+1}{2} \right).}
\end{equation*}
To resolve the inner (boundary) layer we will introduce a change of variables.
Let
\begin{equation*}
  x=\epsilon^{\alpha}s,\quad s=\epsilon^{-\alpha}x
\end{equation*}
so that \cref{eq:3.3.1g-problem} becomes
\begin{equation*}
  \epsilon^{-2\alpha+1}y''(s) + 2\epsilon^{-\alpha}y'(s) + e^y = 0.
\end{equation*}
Multiplying both sides by $\epsilon^{2\alpha-1}$ yields
\begin{equation}
  \label{eq:3.3.1g-ys-alpha}
  y''(s) + 2\epsilon^{\alpha-1}y'(s) + \epsilon^{2\alpha-1}e^y=0
\end{equation}
so that $\alpha$ must conform to
\begin{equation}
  \label{eq:3.3.1g-alpha-conditions}
  \begin{aligned}
    \alpha-1&\ge0, &\quad 2\alpha-1&\ge0 \\
    \implies \alpha&\ge1, &\quad \implies \alpha&\ge1/2. \\
  \end{aligned}
\end{equation}
The minimum $\alpha$ value which conforms to the
\cref{eq:3.3.1g-alpha-conditions} is $\alpha=1$. Thus \cref{eq:3.3.1g-ys-alpha}
becomes
\begin{equation*}
  \label{eq:3.3.1g-ys}
  y''(s) + 2y + \epsilon e^y=0.
\end{equation*}
Collecting coefficients of $\epsilon^0$ yeilds a linear second-order ODE\@
\begin{equation*}
  y''(s) + 2y'(s) = 0
\end{equation*}
whose characteristic equation of
\begin{equation*}
  r^2+2r=0
\end{equation*}
yields a solution in the form
\begin{equation*}
  y(s) = c_1 + c_2e^{-2s}
\end{equation*}
Invoking the initial condition in the boundary (inner) layer resolves $c_2$:
\begin{equation*}
  \begin{aligned}
    y(0) &= c_1+c_2 \\
    \implies c_2 &= -c_1.
  \end{aligned}
\end{equation*}
So that the boundary (inner) approximation becomes
\begin{equation*}
  \boxed{y_i(s) = c_1(1-e^{-2s})}
\end{equation*}
Now we'll perform the matching step to resolve $c_1$.
\begin{equation*}
  \begin{aligned}
    \lim_{x\rightarrow0}-\ln\left(\frac{x+1}{2}\right) &= \lim_{s\rightarrow\infty}c_1(1-e^{-2s}) \\
    -\ln\left(\frac{1}{2}\right) &= c_1
    \implies c_1 &= -\ln\left(\frac{1}{2}\right)
  \end{aligned}
\end{equation*}
Thus the uniform approximation in the form of $y(x)=y_0(x)+y_i(x)-C$, where $C$
is the common limit, is
\begin{equation*}
  \boxed{y(x)=-\ln\left( \frac{x+1}{2} \right) -
    \ln\left(\frac{1}{2}\right)(1-e^{-2x/\epsilon}) -
    \ln\left( \frac{1}{2} \right).}
\end{equation*}

\section{\S~3.4}
\subsection{3$^1$}
\subsubsection*{Problem}
Find a uniformly valid approximation to the problem
\begin{equation}
  \label{eq:3.4.3-problem}
  \begin{aligned}
    \epsilon y''+{(t+1)}^2y'&=1, \quad y>0,\quad 0<\epsilon\ll1,\\
    y(0)&=1, \quad \epsilon y'(0)=1 \\
  \end{aligned}
\end{equation}
\subsubsection*{Remarks}
Assume boundary layer is on left side of interval and find leading-order
composite approximation.
\subsubsection*{Solution}
First we'll obtain the outer approximation. We'll collect coefficients of
$\epsilon^0$ and solve the resulting ODE.\@
\begin{equation*}
  {(t+1)}^2y'(t) = 1
\end{equation*}
so that
\begin{equation*}
  \boxed{\implies y(t) = -\frac{1}{1+t}+c_1}
\end{equation*}
\Cref{eq:3.4.3-problem} is an initial value problem, so we cannot obtain $c_1$
at the moment; We'll find $c_1$ during the matching step. To resolve the inner
(boundary) layer we will introduce a change of variables. Let
\begin{equation*}
  t=\epsilon^{\alpha}s,\quad s=\epsilon^{-\alpha}t
\end{equation*}
so that \cref{eq:3.4.3-problem} becomes
\begin{equation*}
  \epsilon^{-2\alpha+1}y''(s) + {(t+1)}^2\epsilon^{-\alpha}y'(s) = 1.
\end{equation*}
Multiplying both sides by $\epsilon^{2\alpha-1}$ yields
\begin{equation}
  \label{eq:3.4.3-ys-alpha}
  \begin{aligned}
    y''(s) + {(t+1)}^2\epsilon^{\alpha-1}y'(s) &= \epsilon^{2\alpha-1} \\
    y''(s) + {(\epsilon^{\alpha}s+1)}^2\epsilon^{\alpha-1}y'(s) &= \epsilon^{2\alpha-1} \\
    y''(s) + (\epsilon^{2\alpha}s^2 + 2\epsilon^{\alpha}s + 1) \epsilon^{\alpha-1} y'(s) &= \epsilon^{2\alpha-1} \\
    y''(s) + (\epsilon^{3\alpha-1}s^2 + 2\epsilon^{2\alpha-1}s + \epsilon^{\alpha-1}) y'(s) &= \epsilon^{2\alpha-1} \\
  \end{aligned}
\end{equation}
so that $\alpha$ must conform to
\begin{equation}
  \label{eq:3.4.3-alpha-conditions}
  \begin{aligned}
    \alpha-1&\ge0, &\quad 2\alpha-1&\ge0, &\quad 3\alpha-1&\ge0 \\
    \implies \alpha&\ge1, &\quad \implies \alpha&\ge1/2, &\quad \implies \alpha&\ge1/3 \\
  \end{aligned}
\end{equation}
The minimum $\alpha$ value which conforms to the
\cref{eq:3.4.3-alpha-conditions} is $\alpha=1$. Thus \cref{eq:3.4.3-ys-alpha}
becomes a second-order ODE.\@
\begin{equation}
  \label{eq:3.4.3-ys}
  \begin{aligned}
    y''(s) + (\epsilon^{2}+2\epsilon+1)y'(s) &= \epsilon^{2\alpha-1} \\
    \implies y''(s) + y'(s) &= 0 \\
  \end{aligned}
\end{equation}
We'll refactor the $\epsilon y'(0)=1$ initial condition to apply for our
substitution so that we may solve for $y(s)$.
\begin{equation*}
  \begin{aligned}
    y(s) &= y(\epsilon^{-\alpha}t) \\
    y'(s) &= \epsilon^{-\alpha}y(\epsilon^{-\alpha}t) \\
    \implies \epsilon^{1-\alpha}y'(0) &= 1.
  \end{aligned}
\end{equation*}
Since $\alpha=1$, the initial conditions for $y(s)$ become
\begin{equation}
  \label{eq:3.4.3-initial-conditions}
  y(0) = 1,\quad y'(0) = 1.
\end{equation}
Thus, solving \cref{eq:3.4.3-ys} with initial conditions found in
\cref{eq:3.4.3-initial-conditions} yields
\begin{equation*}
  \boxed{y(s) = 2-e^{-s}}
\end{equation*}
Now we'll perform the matching step to resolve $c_1$.
\begin{equation*}
  \begin{aligned}
    \lim_{t\rightarrow0}-\frac{1}{1+t}+c_1 &= \lim_{s\rightarrow\infty}2-e^{-s} \\
    -1+c_1 &= 2 \\
    \implies c_1 &= 3.
  \end{aligned}
\end{equation*}
Thus the uniform approximation in the form of $y(x)=y_0(x)+y_i(x)-C$, where $C$
is the common limit, is
\begin{equation*}
  \boxed{y(x)=-\frac{1}{1+t}+2-e^{-s}.}
\end{equation*}

\section{Programming Minilab}
Due to surface tension, a liquid-gas interface will rise up or dip down at a
solid boundary to form a meniscus, which is responsible for the so-called
capillary effect in narrow tubes. In the planar case, the shape of the interface
or meniscus curve $y(x)$ is described by
\begin{equation}
  \label{eq:minilab-problem}
  \begin{aligned}
    \frac{\sigma}{\rho g}\od[2]{y}{x} = y {\left[ 1 + {\left( \od{y}{x}
          \right)}^2 \right]}^{3/2},\quad 0\le x\le L \\
    \od{y}{x}(0) = 0,\quad \od{y}{x}(L) = \tan\gamma.
  \end{aligned}
\end{equation}
where $\rho$ is the density of the liquid, $g$ is the gravitational
acceleration, and $\rho$ and $\gamma$ are the surface tension and wetting angle
constants associated with the interface. Here we develop an approximate solution
of \cref{eq:minilab-problem} under the assumption that $0<\frac{\sigma}{\rho
  g}\ll L^2$. For this problem we expect a boundary layer at $x=L$, and we will
see that $\od{y}{x}$ is regular, by $\od[2]{y}{x}$ behaves singularly.

\subsection{a}
\label{sec:minilab-part-a}
\subsubsection*{Problem}
Introduce the dimensionless variables $h=y/L$ and $s=x/L$ and show that
\cref{eq:minilab-problem} can be written in the following form for an
appropriate parameter $0<\epsilon\ll1$.
\begin{equation}
  \label{eq:minilab-problem-part-a}
  \begin{aligned}
    \epsilon\od[2]{h}{s} = h{\left[ 1+{\left( \od{h}{s} \right)}^2
      \right]}^{3/2},\quad 0\le s\le1, \\
    \od{h}{s}(0) = 0,\quad \od{h}{s}(1) = \tan\gamma
  \end{aligned}
\end{equation}
\subsubsection*{Solution}
\todo{}

\subsection{b}
\label{sec:minilab-part-b}
\subsubsection*{Problem}
Find the leading-order approximation.
\subsubsection*{Solution}
\todo{}

\subsection{c}
\label{sec:minilab-part-c}
\subsubsection*{Problem}
\begin{enumerate}[(i)]
\item For the inner approximation, consider the change of variables
  $\tau=\epsilon^{-\alpha}(s-a)$ and $u=\epsilon^{-\beta}h$. Show that
  $\alpha=1/2$ and $\beta=1/2$ leads to the regular system
  \begin{equation}
    \label{eq:minilab-problem-part-c}
    \od[2]{u}{\tau} = u{\left[ 1+{\left( \od{u}{\tau} \right)}^2 \right]}^{3/2},
    \quad \tau\le0,\quad \od{u}{\tau}=\tan\gamma.
  \end{equation}

\item This system is difficult to solve. To simplify it, we expect that
  $\od{u}{\tau}$ varies between zero and $\tan\gamma$ in the boundary layer;
  hence using an average of $\frac{1}{2}\tan\gamma$, we replace
  $1+{\left(\od{u}{\tau}\right)}^2$ by $1+\frac{1}{4}\tan^2\gamma$ in the
  differential equation. Use this simplification to find an inner approximation.
\end{enumerate}
\subsubsection*{Solution}
\begin{enumerate}[(i)]
\item We may utilize the change of variables to obtain $\od{h}{s}$ in terms of
  $\od{u}{\tau}$
  \begin{equation*}
    \begin{aligned}
      \od{u}{\tau} &= \od{u}{h}\od{h}{s}\od{s}{\tau} \\
      &= \epsilon^{-\beta}\od{h}{s}\epsilon^{\alpha} \\
      &= \epsilon^{\alpha-\beta}\od{h}{s} \\
      \implies \od{h}{s} &= \epsilon^{\beta-\alpha}\od{u}{\tau} \\
    \end{aligned}
  \end{equation*}
  Similarly, we may obtain $\od[2]{h}{s}$ in terms of $\od[2]{u}{\tau}$.
  \begin{equation*}
    \begin{aligned}
      \od{}{\tau}\od{u}{\tau} &= \od{u}{\tau}\od{h}{s}\od{}{s}\od{s}{\tau} \\
      \od[2]{u}{\tau} &= \epsilon^{\alpha-\beta}\od[2]{h}{s}\epsilon^{\alpha} \\
      &= \epsilon^{2\alpha-\beta}od[2]{h}{s} \\
      \implies \od[2]{h}{s} &= \epsilon^{\beta-2\alpha}\od[2]{u}{\tau}
    \end{aligned}
  \end{equation*}
  Substituting $\od{h}{s}$ and $\od[2]{h}{s}$ into \cref{eq:minilab-problem-part-c}
  \begin{equation*}
    \epsilon\left(\epsilon^{\beta-2\alpha}\od[2]{u}{\tau}\right) =
    h{\left(1+{\left(\epsilon^{\beta-\alpha}\od{u}{\tau}\right)}^2\right)}^{3/2}
  \end{equation*}
\end{enumerate}

\subsection{d}
\label{sec:minilab-part-d}
\subsubsection*{Problem}
By matching the results from \cref{sec:minilab-part-b,sec:minilab-part-c},
construct a leading-order composite approximation for $h(s)$, and hence fro the
meniscus curve $y(x)$. Use the approximation to find an expression for the
meniscus height $\ell=y(L)$. Show that $\od{y}{x}$ remains bounded at all
points as $\epsilon\downarrow0$, but that $\od[2]{y}{x}$ grows unbounded at
$x=L$ as $\epsilon\downarrow0$.
\subsubsection*{Solution}
\todo{}

\subsection{e}
\subsubsection*{Problem}
Simulate the equations in \cref{eq:minilab-problem} for $\sigma=0.05$,
$\rho=1000$, $g=10$, $l=0.07$, and $\gamma=\pi/3$ in kg-m-s units. Superimpose
plots of the curve $y(x)$ produced by the simulation and by your approximation
from \cref{sec:minilab-part-d}. Is your prediction of the meniscus height $\ell$
in (approximate) agreement with the simulation?
\subsubsection*{Solution}
\todo{}

\end{document}