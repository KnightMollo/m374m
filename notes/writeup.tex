\documentclass[12pt]{article}
\title{M374M Notes}
\author{Hershal Bhave (hb6279)}
\date{Updated \today}

\usepackage{macros}

\begin{document}
\maketitle
\tableofcontents

\section{Dimensional Analysis}
\subsection{Example}
Let $q = \alpha p^2$, where $[\alpha] = \frac{L}{T^2}$, $[p] = T$. Find $[q]$,
$\left[\frac{dq}{dp}\right]$.

\subsection{Definition}
$q$ is \emph{dimensionless} if $[q] = 1$. Angles are dimensionless, for example.

\subsection{Examples}
\subsubsection{}
Let $y = f(x)$, where $[y] = [x]$. Then $\frac{dy}{dx}$ is dimensionless.

\subsubsection{}
By definition, pure numbers are dimensionless

\begin{equation}
  \begin{aligned}
    g &= 10 \text{m}/\text{s}^2 \\
    g + g &= 2g = 20 \text{m}/\text{s}^2 \\
    g \cdot g \cdot g & = g^3 = 1000 \text{m}^3/\text{s}^6
  \end{aligned}
\end{equation}

\subsection{Remark}
Some functions make dimensional sense only if inputs are dimensionless.

\begin{equation}
  \begin{aligned}
    e^q &= 1 + q + \frac{q^2}{2!} + \frac{q^3}{3!} + \cdots \\
    [q] &= 1 \implies [e^q] = 1 \\
    [q] &= L \implies [e^q] = ?
  \end{aligned}
\end{equation}

\subsection{Definition}
An equation in ``standard form'' $F(q, p, \ldots) = 0$ is called unit-free if
each term has the same dimenison.

\subsection{Example}
Galileo's Law.

\begin{equation}
  \begin{aligned}
    x &= \frac{gt^2}{2} \\
    x - \frac{gt^2}{2} &= 0 \\
    F(x,t,g) &= 0
  \end{aligned}
\end{equation}

Where $[x] = L$, $[\frac{gt^2}{2}] = [gt^2] = L$.

$\therefore F(x,t,g)=0$ is unit-free.

\subsection{Remarks}
\begin{enumerate}
\item All equations we consider are unit-free
\item Differentiation, integration preserves unit-free propery of an equation
\end{enumerate}

\begin{equation}
  \begin{aligned}
    \frac{d^2x}{dt^2} &= g \text{unit-free} \\
    \frac{dx}{dt} &= gt + v_0 \\
    x &= \frac{1}{2}gt^2 + v_0 t + x_0 \\
  \end{aligned}
  %% $\therefore$ unit-free ODE $\implies$ unit-free solution.
\end{equation}

\subsection{Definition}
Let $Q = {q_1, ..., q_m}$ be a set of quantities involving a set $D =
{D_1,...,D_n}$ of dimensions so that

\begin{equation}
  \begin{aligned}
    [q_1] &= D_1^{a_{11}} D_2^{a_{21}}\cdots D_n^{a_{n1}} \\
    \vdots & \\
    [q_2] &= D_1^{a_{21}} D_2^{a_{22}}\cdots D_n^{a_{n2}} \\
  \end{aligned}
\end{equation}
for some exponents $a_{11}, \ldots, a_{nm}$.

By the dimension matrix for $Q$, $D$, we mean
\begin{equation}
  A =
  \begin{pmatrix}
    a11 & a12 & \cdots & a1m \\
    a21 & a22 & \cdots & a2m \\
    \vdots & & & \\
    an1 & an2 & \cdots & anm \\
  \end{pmatrix}
  \in \mathbb{R}^{n\times m}
\end{equation}

\subsection{Example}
$m = 3 | Q = {x, t, g}$, $n = 2 | D = {L, T}$.
\begin{equation}
  \begin{aligned}
    [x] &= L^1 T^0 \\
    [t] &= L^0 T^1 \\
    [g] &= L^1 T^{-2}
  \end{aligned}
\end{equation}

\begin{equation}
  A =
  \begin{pmatrix}
    1 & 0 & 1 \\
    0 & 1 & -2 \\
  \end{pmatrix}
  \in \mathbb{R}^{2\times3}
\end{equation}

\subsection{Remarks}
All dimensionless combinations of $q_1,\ldots,q_m$ can be found from the null
vectors of $A$. That is,

$$Z = q_1^{v_1}, q_2^{v_2}, \cdots, q_n^{v_n}$$ where $v_1, \ldots, v_m$
arbitrary.

Then
\begin{equation}
  \begin{aligned}
    [Z] &= 1 \rightleftarrows Av=0, \\
    v &= \begin{pmatrix} v_1 \\ \vdots \\ v_m \end{pmatrix}
  \end{aligned}
\end{equation}

\subsection{Example}
Find all dimensionless combinations of $x, t, g$.

\begin{equation}
  \begin{aligned}
    Z &= x^{v_1}t^{v_2}g^{v_3} \\
    [Z] &= [x]^{v_1}[t]^{v_2}[g]^{v_3} \\
    &= L^{v_1}T^{v_2}(LT^{-2})^{v_3} \\
    &= L^{v_1+v_3} T^{v_2-2v_3} \\
    \implies [Z] = 1 & \rightleftarrows  v_1+v_3=0, v_2-2v^3=0
  \end{aligned}
\end{equation}

\begin{equation}
  \begin{aligned}
    \begin{pmatrix}
      1 & 0 & 1 \\
      0 & 1 & -2
    \end{pmatrix}
    \begin{pmatrix}
      v_1 \\ v_2 \\ v_3 \\
    \end{pmatrix}
    &=
    \begin{pmatrix}
      0 \\ 0
    \end{pmatrix}
    \implies Av &= 0
  \end{aligned}
\end{equation}

\begin{equation}
  v = c
  \begin{pmatrix}
    -1 & 2 & 1
  \end{pmatrix}
\end{equation}

Where $c$ is free.

\begin{equation}
  \begin{aligned}
    c &= 1 \rightarrow v =
    \begin{pmatrix}
      -1 \\ 2 \\ 1
    \end{pmatrix} \rightarrow
    Z = x^{-1} t^2 g^1 = \frac{t^2g}{x} \\
    c &= 2 \rightarrow v=
    \begin{pmatrix}
      -2 \\ 4 \\ 2
    \end{pmatrix} \rightarrow
    Z = \frac{t^4g^1}{x^2}
  \end{aligned}
\end{equation}

The cases where $c=1$ and $c=2$ are the same combination, so we don't have
independent results.

\newpage
\section{Characteristic Scales}
\subsection{Definition}
Non-zero constants $t_c$, $y_c$ are called characteristic scales for a function
$y=f(t)$ if the following is true:

\begin{enumerate}
\item \begin{itemize}
\item $[t_c] = [t]$
\item $[y_c] = [t]$
\end{itemize}

\item Main features of $y=f(t)$ graph are clearly visible in a $mt_c \times ny_c$
   window for moderate values of $m,n$ ($1 \le m, \; n \le 10$).
\end{enumerate}

\subsection{Examples}
\subsubsection{Example 1}

$y=A\sin(\pi t/b), t \ge 0$, $A$, $b$ constants, $A = 1 \text{ meter}$, $b=1
\text{ second}$. Reference fig. 1.1.

Characteristic scales are $t_c=b, y_c=A$ since $2b \times 2A$ window captures
main features of graph.

Can understand the features of the graph within this window. Reference fig. 1.2.

Windows of significantly different sizes would give poor representations of
functions. Reference fig. 1.3.

\subsubsection{Example 2}

$y = Ae^{-t/b}, t\ge0$ where $[y]=L$, $[t] = T$, $A$, $b$ constants.

Reference fig. 1.4. Characteristic scales are $t_c=b$, $y_c=A$ since window
$10b \times A$ captures main features.

\subsubsection{Example 3}

Fig. 1.5 has two sets of characteristic scales; two window sizes are needed to
represent the features, as shown in figs. 1.6 and 1.7.

\subsection{Procedure}
Given some (potentially indirect) definition of a graph, how do we indentify the
scales?

If $y=f(t)$ is defined by an equation involving constants $g_1, \ldots, g_m$,
then characteristic scales $t_c$, $y_c$ can be found by solving:

\begin{equation}
  \begin{aligned}
    t_c &= g_1^{\alpha_1}, \ldots, g_m^{\alpha_m} \\
    y_c &= g_1^{\beta_1}, \ldots, g_m^{\beta_m}
  \end{aligned}
\end{equation}

for $\alpha_1, \ldots, \alpha_m$ and $\beta_1, \ldots, \beta_m$.

\subsection{Examples}
\subsubsection{Example 1}

Find characteristic scales $t_c$, $y_c$ for $y=f(t)$ defined by

\begin{equation}
  \left\{
  \begin{aligned}
    \frac{dy}{dt} &= \eta y, \quad t\ge0 \\
    y_{(0)} &= \lambda \\
  \end{aligned} \right.
\end{equation}

$[y] = \Eta , [t]=T$.

Dimensions of $\eta$, $\lambda$:

\begin{equation}
  \begin{aligned}
    \frac{\Eta}{T} &= [\eta]\Eta \rightarrow [\eta]=\frac{1}{T} \\
    \Eta &= [\lambda]
  \end{aligned}
\end{equation}

Scales for $t$, $y$:

\begin{equation}
  \begin{aligned}
    t_c &= \eta^{\alpha_1} \lambda^{\alpha_2} \\
    \implies T &= T^{-\alpha_1}\Eta^{\alpha_2} \\
    \implies \alpha_1 &= -1,\quad \alpha_2 = 0 \\
    \implies t_c &= \frac{1}{\eta}\\
  \end{aligned}
\end{equation}

\begin{equation}
  \begin{aligned}
    y_c &= \eta^{\beta_1} \lambda^{\beta_2} \\
    \implies \Eta &= T^{-\beta_1}\Eta^{\beta_2} \\
    \implies \beta_1 &= 0,\quad \beta_2 = 1 \\
    \implies y_c &= \lambda
  \end{aligned}
\end{equation}

\subsubsection{Example 2}

Do the same for
\begin{equation}
  \left\{
  \begin{aligned}
    \frac{d^2y}{dt^2} &= \eta \frac{dy}{dt} + \mu y^2,\quad t \ge 0 \\
    y_{(0)} &= \lambda, \quad \frac{dy}{dt}(0) = 0
  \end{aligned} \right.
\end{equation}
$[y]=L$, $[t]=T$, $\eta$, $\mu$, $\lambda$ constants.

Dimensions of $\eta$, $\mu$, $\lambda$:
\begin{equation}
  \begin{aligned}
    [\lambda] &= L \\
    [\eta]\frac{L}{T} &= \frac{L}{T^2} \quad\longrightarrow\quad [\eta] = \frac{1}{T} \\
    [\mu]L^2 &= \frac{L}{T^2} \quad\longrightarrow\quad [\mu] = \frac{1}{LT^2} \\
  \end{aligned}
\end{equation}

Scale for $t$:
\begin{equation}
  \begin{aligned}
    t_c &= \eta^{\alpha_1}\mu^{\alpha_2}\lambda^{\alpha_3} \\
    &= T^{-\alpha_1}(L^{-1}T^{-2})^{\alpha_2}L^{\alpha_3} \\
    \implies L^0T^1 &= L^{-\alpha_2+\alpha_3}T^{\alpha_1-2\alpha_2} \\
  \end{aligned}
\end{equation}

\begin{equation}
  \begin{pmatrix}
    -1 & -2 & 0 \\
    0 & -1 & 1 \\
  \end{pmatrix}
\begin{pmatrix}
  \alpha_1 \\
  \alpha_2 \\
  \alpha_3 \\
\end{pmatrix}
=
\begin{pmatrix}
  1\\0
\end{pmatrix}
\end{equation}

Where $\alpha_3$ is free.

\begin{equation}
  \begin{aligned}
    -\alpha_1-2\alpha_2 &= 1 \rightarrow \alpha_1=-2\alpha_2-1=-2\alpha_3-1\\
    -\alpha_2 + \alpha_3 = 0 \rightarrow \alpha_2&=\alpha_3
  \end{aligned}
\end{equation}

\begin{equation}
  \begin{pmatrix}
    \alpha_1 \\ \alpha_2 \\\alpha_3 \\
  \end{pmatrix}
  =
  \begin{pmatrix}
    -2\alpha_3-1 \\ \alpha_3 \\\alpha_3 \\
  \end{pmatrix}
  = \alpha_3
  \begin{pmatrix}
    -2 \\ 1 \\ 1
  \end{pmatrix}
  +
  \begin{pmatrix}
    -1 \\ 0 \\ 0
  \end{pmatrix}
\end{equation}

Where $\alpha_3$ is free. $\alpha_3=0, \quad \alpha_3\ne0$ gives two scales for
t\_c:

\begin{equation}
  \alpha_3=0, \quad
  \begin{pmatrix}
    -1 \\ 0 \\ 0
  \end{pmatrix}
\end{equation}

\begin{equation}
  \implies t_c = \frac{1}{\eta}.
\end{equation}

And

\begin{equation}
  \alpha_3 = \frac{-1}{2}, \quad
  \begin{pmatrix}
    \alpha_1 \\ \alpha_2 \\ \alpha_3 \\
  \end{pmatrix}
  =
  \begin{pmatrix}
    0 \\ -1/2 \\ -1/2
  \end{pmatrix}
\end{equation}

\begin{equation}
  \implies t_c = \frac{1}{\sqrt{\mu\lambda}}
\end{equation}

Same for $y_c$

\begin{equation}
  \begin{aligned}
    y_c &= \frac{\eta^2}{\mu} \\
    y_c &= \lambda
  \end{aligned}
\end{equation}

\newpage
\section{Dimensionless Forms}
\subsection{Definition}
Consider an equation involving vars $v_1,\ldots,v_k$. By the dimensionless form
of an equation with respect to scales $v_1^c,\ldots,v_k^c$ we mean the equation
obtained by the substitution $\overline{v_i}=v_i/v_i^c,\quad i=1,\ldots,k$ where
$\overline{v_i}$ is a dimensionless variable.

\subsection{Example}
\begin{equation}
  \left\{
  \begin{aligned}
    \frac{dy}{dt} &= \eta y+r, \quad t\ge0 \\
    y_{(0)} &= \lambda \\
  \end{aligned}
  \right.
\end{equation}

\begin{equation}
  [y] = \Theta,\quad[t]=T,\quad \eta,\lambda,r > 0 \text{ constants}
\end{equation}

Find dimensionless equation with respect to scales
$t_c=\frac{1}{\eta},\quad y_c=\lambda$.

Dimensionless vars
\begin{equation}
  \begin{aligned}
    \bar{y} &= y/y_c = y/\lambda,\quad & \bar{t} &= t/t_c = \eta t \\
    y &= \lambda \bar{y}, \quad & t&=\bar{t}/\eta \\
  \end{aligned}
\end{equation}

Derivitive relation
\begin{equation}
  \begin{aligned}
    \frac{d\bar{y}}{d\bar{t}} &= \frac{d\bar{y}}{dt} \cdot \frac{dt}{d\bar{t}}
  \end{aligned}
\end{equation}

Where $\frac{d\bar{y}}{dt} = \frac{d}{dt}(y/\lambda) = \frac{1}{\lambda}\frac{dy}{dt}$ and
$\frac{dt}{d\bar{t}} = \frac{1}{\eta}$. Thus

\begin{equation}
  \begin{aligned}
    \frac{d\bar{y}}{d\bar{t}} &= \frac{1}{\eta\lambda}\frac{dy}{dt} \\
    \therefore \frac{dy}{dt} &= \eta\lambda\frac{d\bar{y}}{d\bar{t}} \\
  \end{aligned}
\end{equation}

In general, $\frac{dy}{dt} = \frac{y_c}{t_c} \cdot \frac{d\bar{y}}{d\bar{t}}$.
Substituted into equations,

\begin{equation}
  \begin{aligned}
    \frac{dy}{dt} &= \eta y+r, &\quad t&\ge0 \\
    \eta\lambda\frac{d\bar{y}}{d\bar{t}} &= \eta\lambda\bar{y}+r, &\quad \bar{t}/\eta&\ge0 \\
    \frac{d\bar{y}}{d\bar{t}} &= \bar{y}+\frac{r}{\eta\lambda}, &\quad \bar{t}&\ge0 \\
    y &= \lambda, &\quad t&=0 \\
    \lambda\bar{y} &= \lambda, &\quad \bar{t}/\eta &= 0 \\
    \bar{y} &= 1, &\quad\bar{t}&=0. \\
  \end{aligned}
\end{equation}

We end up with dimensionless equations
\begin{equation} \left\{
  \begin{aligned}
    \frac{d\bar{y}}{d\bar{t}} &= \bar{y} + \bar{r}, &\quad \bar{t}&\ge0 \\
    \bar{y} &= 1, &\quad \bar{t} &= 0
  \end{aligned} \right.
\end{equation}

Where $\bar{r} = \frac{r}{\eta\lambda}$ is a dimensionless constant.

\subsection{Remarks}
\begin{enumerate}
\item Different choices of scales lead to different dimensionless equations
  \begin{equation}
    \begin{aligned}
      t_c &= \lambda/r \\
      y_c &= \lambda
    \end{aligned}
  \end{equation}
  Which leads to

  \begin{equation}
    \begin{aligned}
      \bar{t} &= t/t_c = rt/\lambda \\
      \bar{y} &= y/y_c = y/\lambda \\
    \end{aligned}
  \end{equation}


  Which leads to
  \begin{equation}
    \begin{aligned}
      \frac{d\bar{y}}{d\bar{t}} &= \bar{\eta}\bar{y} + 1, &\quad \bar{t}&\ge0 \\
      \bar{y} &= 1 &\quad \bar{t} &= 0 \\
      \bar{\eta} &= \frac{\eta\lambda}{r} \\
    \end{aligned}
  \end{equation}

  Where $\bar{y}$ and $\bar{\eta}$ are dimensionless.

\item Solutions of dimensionful and dimensionless equations are equivalent

  \begin{equation}
    \text{dimensioned solution}\quad\xleftrightarrow[\bar{t} = t/t_c]{\bar{y} = y/y_c}\quad
    \text{dimensionless solution}
  \end{equation}

\item Dimensionless equations allow comparisons to be made
\end{enumerate}

For example, we may not compare $a$, $b$, $c$, $d$ in \cref{eq:dim-full-eq}
since dimensions are different. By comparison, we may compare $\bar{a}$,
$\bar{b}$, $\bar{c}$, $\bar{d}$ in \cref{eq:dim-less-eq} since all variables are
dimensionless.

\begin{equation}
  \label{eq:dim-full-eq}
  \frac{d^2u}{dt^2} = au + bu^2 + ce^{-u/d}
\end{equation}

\begin{equation}
  \label{eq:dim-less-eq}
  \frac{d^2\bar{u}}{d\bar{t^2}} = \bar{a} \bar{u} + \bar{b} \bar{u^2} + \bar{c}e^{-\bar{u}/\bar{d}}
\end{equation}

\subsection{Example}
Vertical motion of a ball can be described using the variables in
\cref{fig:ball-model} and the equation

\begin{figure}
  \centering
  \begin{tabularx}{0.5\textwidth}{XX}
    variable & quantity \\ \hline
    $g$ & gravity \\
    $\eta$ & air resistance \\
  \end{tabularx}
  \caption{vertical motion of a ball model}
  \label{fig:ball-model}
\end{figure}

\begin{equation}
  \frac{d^2x}{dt^2} = -g - \eta\frac{dx}{dt}, \quad t\ge0
\end{equation}

Where
\begin{equation}
  x=0,\quad \frac{dx}{dt} = v_0, \quad t=0
\end{equation}

Where $\eta$, $g$, $v_0$ are constants greater than zero.

Under what condition on $\eta$ do we expect air effects to be small?

First, choose scales. In the limiting case of no air resistance ($\eta=0$), the
solution is determined by $g$, $v_0$. Using these constants, we get
$t_c=v_0/g, x_c=v_0^2/g$. Define dimensionless variables
$$\bar{x}=x/x_c,\quad \bar{t}=t/t_c$$ and dimensionless
equations
\begin{equation}
  \frac{d^2\bar{x}}{d\bar{t^2}} = -\bar{g} -
  \bar{\eta}\frac{d\bar{x}}{d\bar{t}}, \quad \bar{t}\ge0
\end{equation}

Where dimensionless variables
\begin{equation}
  \begin{aligned}
    \bar{x} &= 0,\quad \frac{d\bar{x}}{d\bar{t}}&=1 \text{ at } \bar{t}=0 \\
    \bar{g} &= 1, \quad \bar{\eta} &= \frac{\eta v_0}{g} \\
  \end{aligned}
\end{equation}

Expect air effects to be small (when solution is viewed in window with sales
$t_c$, $x_c$) where $\bar{\eta} << \bar{g}$, $\frac{\eta v_0}{g} << 1$, and
$\eta << \frac{g}{v_0}$.

\section{Case Study: Chemical Reactor}
\subsection{Problem}
Use scaling to study a model for a pair of reactions.

Chemical $A$, $B$ in fluid

\begin{equation}
  \begin{aligned}
    A + 2B &\xrightarrow{k_1} C \\
    2A &\xrightarrow{k_2} B \\
  \end{aligned}
\end{equation}

Where the flow rate is $\eta$ in gallons/second, tank volume is $V$, Chemical
$A$, $B$, $C$ in fluid. $a_{in}$, $b_{in}$ is the concentration of $A$, $B$ at
the tank inlet, $a_{out}$, $b_{out}$ is the concentration of $A$, $B$ at the
tank outlet, $k_1$, $k_2$ are the rate constants

\begin{equation}
  \begin{aligned}
    [a]=[b]&=\frac{1}{\text{vol}}, &\quad [k_1]&=\frac{\text{vol}^3}{\text{time}}, \\
    [k_2]&=\frac{\text{vol}^2}{\text{time}}, &\quad [\eta]&=\frac{\text{vol}}{\text{time}} \\
  \end{aligned}
\end{equation}

Under what conditions on $k_2$ would we expect the second reaction to be insignificant?

\subsubsection{Derivation of Model}

In a reaction $pA+qB \xrightarrow{k} \text{product}$, one product requires a
successful pairing of $pA$'s and $qB$'s. We assume

\begin{equation}
  \left(\frac{\text{number of successful pairings}}{time}\right) = \left(text{reaction const}\right)\cdot
  \left(\text{number of possible pairings of pA's and qB'mas}\right)
\end{equation}

The second parameter turns out to be
\begin{equation}
  \begin{aligned}
    &\left(text{total number of A's}\right)^p \cdot \left(text{total number of Bmq's}\right)^q \\
    = &(Va)^p(Vb)^q \\
    = & \left(text{reaction const}\right)V^{p+q} a^p b^q \\
  \end{aligned}
\end{equation}

Where $\left(text{reaction const}\right)V^{p+q} = k$.

From this we get

\begin{equation}
  \begin{aligned}
    \left(text{consumption rate of }A\right) &= \left(frac{\text{\# }A}{\text{successful pairing}}\right)
    \left(frac{\text{\# successful pairing}}{\text{time}}\right) \\
    &= (p) (ka^pb^q) \\
  \end{aligned}
\end{equation}

Similarly,
\begin{equation}
  \left(text{consumption rate of }A\right) = (q)(ka^pb^q)
\end{equation}

This is the Law of Mass Action

Now, conservation o fmass is applied to chemical $A$ in tank states

\begin{equation}
  \begin{aligned}
    \od{}{t}\text{\# of A in tank} = \text{rate A enters tank} - \text{rate A leaves tank} - \text{rate of consumption of A} - \text{rate of consumption of A}
  \end{aligned}
\end{equation}

\todo clean up
Where the first ``rate of consumption of A'' is $A + 2B \xrightarrow{k_1} C$,
and the second ``rate of consumption of A'' is $2A \xrightarrow{k_2} B$.

\begin{equation}
  \begin{aligned}
    \od{}{t}(Va) = \eta a_{in} - \eta a - 1\cdot k_1ab^2 - 2\cdot k_2a^2
  \end{aligned}
\end{equation}

Conservation of mass for chemical $B$ states

\begin{equation}
  \od{}{t}\text{\# B in tank} = \text{rate B enters tank} - \text{rate B leaves
    tank} - \text{rate of consumption of B} + \text{rate of production}
\end{equation}

Where the rate of consumption is $A + 2B \xrightarrow{k_1} C$, and the rate of
production is $2A \xrightarrow{k_2} B$.

\todo clean up

\begin{equation}
  \od{}{t}Vb = \eta b_{\text{in}} - \eta b - 2\cdot k_2a^2
\end{equation}

Now, combining all gives
\begin{equation}
  \begin{aligned}
    \od{a}{t} &= \frac{\eta}{V}(a_{in}-a) - \frac{k_1}{V}ab^2 - \frac{2k_2}{V}a^2, &\quad t&\ge0 \\
    \od{b}{t} &= \frac{\eta}{V}(b_{in}-b) - \frac{2k_1}{V}ab^2 + \frac{k_2}{V}a^2, &\quad t&\ge0 \\
  \end{aligned}
\end{equation}

Where
\begin{equation}
  a_{(0)}=a_0,\qquad b_{(0)}=b_0,\qquad \text{ at } t=0
\end{equation}

$\eta$, $V$, $k_1$, $k_2$, $a_{in}$, $b_{in}$, $a_0$, $b_0$ consts $>0$.

\subsubsection{Scales for $t$, $a$, $b$}
In limiting case of no reaction \#2, ($k_2=0$), the remaining consts are
\begin{equation}
  \frac{\eta}{V},\quad \frac{k_1}{V}, \quad a_{in}, \quad b_{in}, \quad a_0, \quad b_0
\end{equation}

Where
\begin{figure}
  \centering
  \begin{tabularx}{0.5\textwidth}{XXX}
    Variable & Dimension \\ \hline
    $\frac{\eta}{V}$ & $T^{-1}$  \\
    $\frac{k_1}{V}$ & $LT^{-1}$ \\
  \end{tabularx}
  \caption{Variable mappings for \#14}
  \label{fig:14-var-mappings}
\end{figure}

We choose scales
\begin{equation}
  \begin{aligned}
    a_c &= a_{in}, Y\quad b_c &= b_{in} \\
    t_c &= \frac{V}{k_1}\frac{1}{a_{in}b_{in}}
  \end{aligned}
\end{equation}

\subsubsection{Dimensionless Form of Equations}
\begin{equation}
  \begin{aligned}
    \bar{a} &= \frac{a}{a_c}, &\quad a &= a_c\bar{a} \\
    \bar{b} &= \frac{b}{b_c}, &\quad b &= b_c\bar{b} \\
    \bar{t} &= \frac{t}{t_c}, &\quad t &= t_c\bar{t} \\
  \end{aligned}
\end{equation}

Derivatives
\begin{equation}
  \begin{aligned}
    \od{\bar{a}}{\bar{t}} &= \od{\bar{a}}{t} \cdot \od{t}{\bar{t}} &= \frac{t_c}{a_c}\cdot\od{a}{t} \\
    &= \frac{1}{a_c}\od{a}{t} \cdot t_c \\
    \od{\bar{b}}{\bar{t}} &= \od{\bar{b}}{t} \cdot \od{t}{\bar{t}} &= \frac{t_c}{b_c}\cdot\od{b}{t} \\
  \end{aligned}
\end{equation}

Equations
\begin{equation}
  \begin{aligned}
    \frac{a_c}{t_c}\cdot\od{\bar{a}}{\bar{t}} &= \frac{\eta}{V}(a_{in}-a_c\bar{a})-\frac{k_1}{V}a_c\bar{a}b_c^2\bar{b}^2 \\
    &- \frac{2k_2}{V}a_c^2a^{-2},\qquad t,\bar{t}\ge 0 \\
  \end{aligned}
\end{equation}

\begin{equation}
  \begin{aligned}
    \frac{b_c}{t_c}\cdot\od{\bar{b}}{\bar{t}} &= \frac{\eta}{V}(b_{in}-b_c\bar{b})-\frac{2k_1}{V}a_c\bar{a}b_c^2\bar{b}^2 \\
    &- \frac{k_2}{V}a_c^2a^{-2},\qquad t,\bar{t}\ge 0 \\
  \end{aligned}
\end{equation}

Where

\begin{equation}
  a_c\bar{a} = a_0, \qquad b_c\bar{b} = b_0, \quad\text{ at } t_c\bar{t} = 0
\end{equation}

Simplifying gives

\begin{equation}
  \begin{aligned}
    \od{\bar{a}}{\bar{t}} &= \mu(1-\bar{a}) - \sigma\bar{a}\bar{b}^2 - 2\lambda\bar{a}^2, &\quad t&\ge0 \\
    \od{\bar{b}}{\bar{t}} &= \mu(1-\bar{b}) - 2\bar{a}\bar{b}^2 - 2\frac{\lambda}{\sigma}\bar{a}^2, &\quad t&\ge0 \\
  \end{aligned}
\end{equation}

Where
\begin{equation}
  \bar{a} = \frac{a_0}{a_{in}}, \quad \bar{b} = \frac{b_0}{b_{in}} \quad \text{ at } t=0
\end{equation}


\subsubsection{Observation}
We expect reaction \#2 terms to have small influence on solution (solution
viewed in scales $t_c, a_c, b_c$) when $\bar{a}$ equation
$$2\lambda << \sigma, \mu \rightarrow \lambda << \frac{\sigma}{2},\frac{\mu}{2}$$,
$\bar{b}$ equation $$\frac{\lambda}{\sigma} << 2, \mu \rightarrow \lambda << 2\sigma, \mu\sigma$$.

All conditions met when
$$\lambda << \frac{\mu}{2}, \frac{\sigma}{2},\mu\sigma$$ which implies $$k_2 << \frac{\eta}{2a_{in}}, \frac{k_1b_{in}^2}{2a_{in}}, \frac{\eta b_{in}}{a_{in}^2}$$

\section{Dynamical Systems in 1 Dimension}
\subsection{Setup}
A dynamical system for a variable $u=u(t)$ is a system in the form
\begin{equation} \text{IVP}\left\{
  \begin{aligned}
    \od{u}{t} &= f(u),&\qquad t\ge t_0 \\
    u(t_0) &= u_0 \\
  \end{aligned}\right.
\end{equation}
We seek to understand behavior of solutions for different $u_0$.

Solutions can be viewed in two ways
\begin{enumerate}
\item $u$ vs $t$ graph. Reference fig. 5.1
\item motion of point $u(t)$ on $u$-axis. Reference fig. 5.2
\end{enumerate}

\subsection{Solvability Theorem}
Consider (IVP) and assume $f(u)$, $f'(u)$ are continuous for all $u_0$.
Then:

\begin{enumerate}
\item $\exists$ a unique solution $u(t)$ for any given $t_0$, $u_0$.
\item Solution $u(t)$ is defined for $t$ in some interval $[t_0,t_0+T]$
  where $T$ depends on $u_0$.
\item Either $T=\infty$ or $T$ is finite; if finite, then
  $|u(t)| \rightarrow \infty$ as $t\rightarrow (t_0+T)$.
\item For a given $t_0$, solutions with different $u_0$ cannot intersect or touch.
\end{enumerate}

\subsection{Examples}
\subsubsection{Example 1}

\begin{equation} \left\{
  \begin{aligned}
    \od{u}{t} &= 2u,&\qquad t\ge0 \\
    u(0) &= u_0 \\
  \end{aligned}\right.
\end{equation}

\begin{equation}
  \begin{aligned}
    f(u) &= 2u \\
    f'(u) &= 2 \\
  \end{aligned}
\end{equation}
This implies $f(u)$ and $f'(u)$ is continuous.

Therefore the solution is defined for all $t\in[0,\infty)$ so $T=\infty$ in this
case.
\begin{equation}
  u(t) = u_0e^{2t}
\end{equation}
Reference figs. 5.3, 5.4.

\subsubsection{Example 2}
\begin{equation} \left\{
  \begin{aligned}
    \od{u}{t} &= u^2, &\qquad t\ge0 \\
    u(0) &= u_0 \\
  \end{aligned} \right.
\end{equation}

\begin{equation}
  \begin{aligned}
    f(u) &= u^2 \\
    f'(u) &= 2u \\
  \end{aligned}
\end{equation}
This implies $f(u)$ and $f'(u)$ is continuous. The solution is defined for
$t\in[0,\frac{1}{u_0})$ if $u_0>0$ and for $t\in[0,\infty)$ if $u_0\le0$.
\begin{equation}
  \begin{aligned}
    u^{-2} \dd{u} &= \dd{t} \\
    \int u^{-2} \dd{u} &= \int \dd{t} \\
    -u^{-1} &= t+C \\
    \implies u(t) &= \frac{-1}{t+C} \\
  \end{aligned}
\end{equation}

$$u(0) = u_0 \rightarrow u_0 = \frac{-1}{C}$$
$$\therefore u(t) = \frac{-1}{t-\frac{1}{u_0}} = \frac{u_0}{1-u_0t}$$

\section{Definition}
An equilibrium or steady-state solutoin of (IVP) is a solution of the form

$$u(t) \equiv u_*,\qquad t\in[t_0,\infty)$$
Since $\od{u}{t}=f(u)$ we have \todo
  %% \begin{equation}
  %%   \left[
  %%     \parbox{2cm}{u_* \text{is an equilibrium solution}}
  %%     \right]
  %%   \Leftrightarrow
  %%   \left[
  %%     f(u_*)=0
  %%   \right]e
  %% \end{equation}

\subsection{Examples}
\subsubsection{Example 1}
\begin{equation} \left\{
  \begin{aligned}
    \od{u}{t} &= 2u, &\qquad t\ge0 \\
    u(0) &= u_0 \\
  \end{aligned} \right.
\end{equation}
\begin{equation}
  \begin{aligned}
    f(u) &= 2u \\
    f(u_*) &= 0 \\
    \implies u_* &= 0 \\
  \end{aligned}
\end{equation}

\subsubsection{Example 2}
\begin{equation} \left\{
  \begin{aligned}
    \od{u}{t} &= u^2-3u &\qquad t\ge0 \\
    u(0) &= u_0 \\
  \end{aligned} \right.
\end{equation}
\begin{equation}
  \begin{aligned}
    f(u) &= u^2-3u \\
    f(u_*) &= 0 \rightarrow u^2-3u_* = 0 \\
    \implies u_* &= 0,3 \\
  \end{aligned}
\end{equation}

This has two constant equilibrium solutions which exist for all time.
Reference fig. 5.7.

\subsection{Characterization Theorem}
Consider (IVP) with $f(u)$, $f'(u)$ continuous and define the sets
\begin{equation}
  \begin{aligned}
    \text{Equil} &= {u | f(u) = 0} \\
    I^+ &= {u | f(u) > 0} \\
    I^- &= {u | f(u) < 0} \\
  \end{aligned}
\end{equation}
Then $u\in\text{Equil} \implies u(t)\equiv u_0$ is an equal solution.
$u_0\in I^+ \implies u(t)$ increases as it is defined. $u_0\in I^-
\implies u(t)$ decreases while it is defined. Since solutions with
different $u_0$ cannot interset, this gives a complete portrait of
behavior.

\subsubsection{Example 1}
Sketch time and phase portait of solutions for
\begin{equation} \left\{
  \begin{aligned}
    \od{u}{t} &= u^3-8u^2,&\qquad t\ge0 \\
    u(0) &= u_0 \\
  \end{aligned} \right.
\end{equation}

$$f(u)=u^3-8u^2 = u^2(u-8)$$ Reference figs. 5.8--5.10.

\subsubsection{Example 2}
\begin{equation} \left\{
  \begin{aligned}
    \od{u}{t} &= u-e^{-u},&\qquad t\ge0 \\
    u(0) &= u_0 \\
  \end{aligned} \right.
\end{equation}
Does $f(u)=u-e^{-u}$ have any roots? Reference figs. 5.11, 5.12.

\end{document}
