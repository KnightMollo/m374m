\documentclass[12pt]{article}
\title{M374M Homework 1 \\
  \normalsize{\S~1.1 \#4$^1$, 5, 9$^2$, 13, 14$^3$}}
\author{Hershal Bhave (hb6279)}
\date{Due 2016-02-01}

\usepackage{macros}

\begin{document}

\tableofcontents

\section{Characteristic Scales}
\subsection{Definition}
Non-zero constants $t_c$, $y_c$ are called characteristic scales for a function
$y=f(t)$ if the following is true:

\begin{enumerate}
\item \begin{itemize}
\item $[t_c] = [t]$
\item $[y_c] = [t]$
\end{itemize}

\item Main features of $y=f(t)$ graph are clearly visible in a $mt_c \times ny_c$
   window for moderate values of $m,n$ ($1 \le m, \quad n \le 10$).
\end{enumerate}

\subsection{Examples}
\subsubsection{Example 1}

$y=A\sin(\pi t/b), t \ge 0$, $A$, $b$ constants, $A = 1 \text{ meter}$, $b=1
\text{ second}$. Reference fig. 1.1.

Characteristic scales are $t_c=b, y_c=A$ since $2b \times 2A$ window captures
main features of graph.

Can understand the features of the graph within this window. Reference fig. 1.2.

Windows of significantly different sizes would give poor representations of
functions. Reference fig. 1.3.

\subsubsection{Example 2}

$y = Ae^{-t/b}, t\ge0$ where $[y]=L$, $[t] = T$, $A$, $b$ constants.

Reference fig. 1.4. Characteristic scales are $t_c=b$, $y_c=A$ since window
$10b \times A$ captures main features.

\subsubsection{Example 3}

Fig. 1.5 has two sets of characteristic scales; two window sizes are needed to
represent the features, as shown in figs. 1.6 and 1.7.

\subsection{Procedure}
Given some (potentially indirect) definition of a graph, how do we indentify the
scales?

If $y=f(t)$ is defined by an equation involving constants $g_1, \ldots, g_m$,
then characteristic scales $t_c$, $y_c$ can be found by solving:

\begin{equation}
  \begin{aligned}
    t_c &= g_1^{\alpha_1}, \ldots, g_m^{\alpha_m} \\
    y_c &= g_1^{\beta_1}, \ldots, g_m^{\beta_m}
  \end{aligned}
\end{equation}

for $\alpha_1, \ldots, \alpha_m$ and $\beta_1, \ldots, \beta_m$.

\subsection{Examples}
\subsubsection{Example 1}

Find characteristic scales $t_c$, $y_c$ for $y=f(t)$ defined by

\begin{equation}
  \left\{
  \begin{aligned}
    \frac{dy}{dt} &= \eta y, \quad t\ge0 \\
    y_{(0)} &= \lambda \\
  \end{aligned} \right.
\end{equation}

$[y] = \Eta , [t]=T$.

Dimensions of $\eta$, $\lambda$:

\begin{equation}
  \begin{aligned}
    \frac{\Eta}{T} &= [\eta]\Eta \rightarrow [\eta]=\frac{1}{T} \\
    \Eta &= [\lambda]
  \end{aligned}
\end{equation}

Scales for $t$, $y$:

\begin{equation}
  \begin{aligned}
    t_c &= \eta^{\alpha_1} \lambda^{\alpha_2} \\
    \implies T &= T^{-\alpha_1}\Eta^{\alpha_2} \\
    \implies \alpha_1 &= -1,\quad \alpha_2 = 0 \\
    \implies t_c &= \frac{1}{\eta}\\
  \end{aligned}
\end{equation}

\begin{equation}
  \begin{aligned}
    y_c &= \eta^{\beta_1} \lambda^{\beta_2} \\
    \implies \Eta &= T^{-\beta_1}\Eta^{\beta_2} \\
    \implies \beta_1 &= 0,\quad \beta_2 = 1 \\
    \implies y_c &= \lambda
  \end{aligned}
\end{equation}

\subsubsection{Example 2}

Do the same for
\begin{equation}
  \left\{
  \begin{aligned}
    \frac{d^2y}{dt^2} &= \eta \frac{dy}{dt} + \mu y^2,\quad t \ge 0 \\
    y_{(0)} &= \lambda, \quad \frac{dy}{dt}(0) = 0
  \end{aligned} \right.
\end{equation}
$[y]=L$, $[t]=T$, $\eta$, $\mu$, $\lambda$ constants.

Dimensions of $\eta$, $\mu$, $\lambda$:
\begin{equation}
  \begin{aligned}
    [\lambda] &= L \\
    [\eta]\frac{L}{T} &= \frac{L}{T^2} \quad\longrightarrow\quad [\eta] = \frac{1}{T} \\
    [\mu]L^2 &= \frac{L}{T^2} \quad\longrightarrow\quad [\mu] = \frac{1}{LT^2} \\
  \end{aligned}
\end{equation}

Scale for $t$:
\begin{equation}
  \begin{aligned}
    t_c &= \eta^{\alpha_1}\mu^{\alpha_2}\lambda^{\alpha_3} \\
    &= T^{-\alpha_1}(L^{-1}T^{-2})^{\alpha_2}L^{\alpha_3} \\
    \implies L^0T^1 &= L^{-\alpha_2+\alpha_3}T^{\alpha_1-2\alpha_2} \\
  \end{aligned}
\end{equation}

\begin{equation}
  \begin{pmatrix}
    -1 & -2 & 0 \\
    0 & -1 & 1 \\
  \end{pmatrix}
\begin{pmatrix}
  \alpha_1 \\
  \alpha_2 \\
  \alpha_3 \\
\end{pmatrix}
=
\begin{pmatrix}
  1\\0
\end{pmatrix}
\end{equation}

Where $\alpha_3$ is free.

\begin{equation}
  \begin{aligned}
    -\alpha_1-2\alpha_2 &= 1 \rightarrow \alpha_1=-2\alpha_2-1=-2\alpha_3-1\\
    -\alpha_2 + \alpha_3 = 0 \rightarrow \alpha_2&=\alpha_3
  \end{aligned}
\end{equation}

\begin{equation}
  \begin{pmatrix}
    \alpha_1 \\ \alpha_2 \\\alpha_3 \\
  \end{pmatrix}
  =
  \begin{pmatrix}
    -2\alpha_3-1 \\ \alpha_3 \\\alpha_3 \\
  \end{pmatrix}
  = \alpha_3
  \begin{pmatrix}
    -2 \\ 1 \\ 1
  \end{pmatrix}
  +
  \begin{pmatrix}
    -1 \\ 0 \\ 0
  \end{pmatrix}
\end{equation}

Where $\alpha_3$ is free. $\alpha_3=0, \quad \alpha_3\ne0$ gives two scales for
t\_c:

\begin{equation}
  \alpha_3=0, \quad
  \begin{pmatrix}
    -1 \\ 0 \\ 0
  \end{pmatrix}
\end{equation}

\begin{equation}
  \implies t_c = \frac{1}{\eta}.
\end{equation}

And

\begin{equation}
  \alpha_3 = \frac{-1}{2}, \quad
  \begin{pmatrix}
    \alpha_1 \\ \alpha_2 \\ \alpha_3 \\
  \end{pmatrix}
  =
  \begin{pmatrix}
    0 \\ -1/2 \\ -1/2
  \end{pmatrix}
\end{equation}

\begin{equation}
  \implies t_c = \frac{1}{\sqrt{\mu\lambda}}
\end{equation}

Same for $y_c$

\begin{equation}
  \begin{aligned}
    y_c &= \frac{\eta^2}{\mu} \\
    y_c &= \lambda
  \end{aligned}
\end{equation}

\end{document}
