\documentclass[12pt]{article}
\title{M374M Notes}
\author{Hershal Bhave (hb6279)}
\date{Updated \today}

\usepackage{homework-macros}

\begin{document}
\maketitle
\tableofcontents

\newpage
\section{Dimensional Analysis}
\subsection{Example}
Let $q = \alpha p^2$, where $[\alpha] = \frac{L}{T^2}$, $[p] = T$. Find $[q]$,
$\left[\frac{dq}{dp}\right]$.

\subsection{Definition}
$q$ is \emph{dimensionless} if $[q] = 1$. Angles are dimensionless, for example.

\subsection{Examples}
\subsubsection{}
Let $y = f(x)$, where $[y] = [x]$. Then $\frac{dy}{dx}$ is dimensionless.

\subsubsection{}
By definition, pure numbers are dimensionless

\begin{equation}
  \begin{aligned}
    g &= 10 \text{m}/\text{s}^2 \\
    g + g &= 2g = 20 \text{m}/\text{s}^2 \\
    g \cdot g \cdot g & = g^3 = 1000 \text{m}^3/\text{s}^6
  \end{aligned}
\end{equation}

\subsection{Remark}
Some functions make dimensional sense only if inputs are dimensionless.

\begin{equation}
  \begin{aligned}
    e^q &= 1 + q + \frac{q^2}{2!} + \frac{q^3}{3!} + \cdots \\
    [q] &= 1 \implies [e^q] = 1 \\
    [q] &= L \implies [e^q] = ?
  \end{aligned}
\end{equation}

\subsection{Definition}
An equation in ``standard form'' $F(q, p, \ldots) = 0$ is called unit-free if
each term has the same dimenison.

\subsection{Example}
Galileo's Law.

\begin{equation}
  \begin{aligned}
    x &= \frac{gt^2}{2} \\
    x - \frac{gt^2}{2} &= 0 \\
    F(x,t,g) &= 0
  \end{aligned}
\end{equation}

Where $[x] = L$, $[\frac{gt^2}{2}] = [gt^2] = L$.
$\therefore F(x,t,g)=0$ is unit-free.

\subsection{Remarks}
\begin{enumerate}
\item All equations we consider are unit-free
\item Differentiation, integration preserves unit-free propery of an equation
\end{enumerate}

\begin{equation}
  \begin{aligned}
    \frac{d^2x}{dt^2} &= g \text{unit-free} \\
    \frac{dx}{dt} &= gt + v_0 \\
    x &= \frac{1}{2}gt^2 + v_0 t + x_0 \\
  \end{aligned}
  %% $\therefore$ unit-free ODE $\implies$ unit-free solution.
\end{equation}

\subsection{Definition}
Let $Q = {q_1, \ldots, q_m}$ be a set of quantities involving a set $D =
{D_1,\ldots,D_n}$ of dimensions so that

\begin{equation}
  \begin{aligned}
    [q_1] &= D_1^{a_{11}} D_2^{a_{21}}\cdots D_n^{a_{n1}} \\
    \vdots & \\
    [q_2] &= D_1^{a_{21}} D_2^{a_{22}}\cdots D_n^{a_{n2}} \\
  \end{aligned}
\end{equation}
for some exponents $a_{11}, \ldots, a_{nm}$.

By the dimension matrix for $Q$, $D$, we mean
\begin{equation}
  A =
  \begin{pmatrix}
    a11 & a12 & \cdots & a1m \\
    a21 & a22 & \cdots & a2m \\
    \vdots & & & \\
    an1 & an2 & \cdots & anm \\
  \end{pmatrix}
  \in \mathbb{R}^{n\times m}
\end{equation}

\subsection{Example}
$m = 3 | Q = {x, t, g}$, $n = 2 | D = {L, T}$.
\begin{equation}
  \begin{aligned}
    [x] &= L^1 T^0 \\
    [t] &= L^0 T^1 \\
    [g] &= L^1 T^{-2}
  \end{aligned}
\end{equation}

\begin{equation}
  A =
  \begin{pmatrix}
    1 & 0 & 1 \\
    0 & 1 & -2 \\
  \end{pmatrix}
  \in \mathbb{R}^{2\times3}
\end{equation}

\subsection{Remarks}
All dimensionless combinations of $q_1,\ldots,q_m$ can be found from the null
vectors of $A$. That is,

$$Z = q_1^{v_1}, q_2^{v_2}, \ldots, q_n^{v_n}$$ where $v_1, \ldots, v_m$
arbitrary. Then
\begin{equation}
  \begin{aligned}
    [Z] &= 1 \rightleftarrows Av=0, \\
    v &= \begin{pmatrix} v_1 \\ \vdots \\ v_m \end{pmatrix}
  \end{aligned}
\end{equation}

\subsection{Example}
Find all dimensionless combinations of $x, t, g$.

\begin{equation}
  \begin{aligned}
    Z &= x^{v_1}t^{v_2}g^{v_3} \\
    [Z] &= {[x]}^{v_1}{[t]}^{v_2}{[g]}^{v_3} \\
    &= L^{v_1}T^{v_2}{(LT^{-2})}^{v_3} \\
    &= L^{v_1+v_3} T^{v_2-2v_3} \\
    \implies [Z] = 1 & \rightleftarrows  v_1+v_3=0, v_2-2v^3=0
  \end{aligned}
\end{equation}

\begin{equation}
  \begin{aligned}
    \begin{pmatrix}
      1 & 0 & 1 \\
      0 & 1 & -2
    \end{pmatrix}
    \begin{pmatrix}
      v_1 \\ v_2 \\ v_3 \\
    \end{pmatrix}
    &=
    \begin{pmatrix}
      0 \\ 0
    \end{pmatrix}
    \implies Av &= 0
  \end{aligned}
\end{equation}

\begin{equation}
  v = c
  \begin{pmatrix}
    -1 & 2 & 1
  \end{pmatrix}
\end{equation}
Where $c$ is free.
\begin{equation}
  \begin{aligned}
    c &= 1 \rightarrow v =
    \begin{pmatrix}
      -1 \\ 2 \\ 1
    \end{pmatrix} \rightarrow
    Z = x^{-1} t^2 g^1 = \frac{t^2g}{x} \\
    c &= 2 \rightarrow v=
    \begin{pmatrix}
      -2 \\ 4 \\ 2
    \end{pmatrix} \rightarrow
    Z = \frac{t^4g^1}{x^2}
  \end{aligned}
\end{equation}
The cases where $c=1$ and $c=2$ are the same combination, so we don't have
independent results.

\newpage
\section{Characteristic Scales}
\subsection{Definition}
Non-zero constants $t_c$, $y_c$ are called characteristic scales for a function
$y=f(t)$ if the following is true:

\begin{enumerate}
\item \begin{itemize}
\item $[t_c] = [t]$
\item $[y_c] = [t]$
\end{itemize}

\item Main features of $y=f(t)$ graph are clearly visible in a $mt_c \times ny_c$
   window for moderate values of $m,n$ ($1 \le m, \; n \le 10$).
\end{enumerate}

\subsection{Examples}
\subsubsection{Example 1}

$y=A\sin(\pi t/b), t \ge 0$, $A$, $b$ constants, $A = 1 \text{ meter}$, $b=1
\text{ second}$. Reference fig. 1.1.

Characteristic scales are $t_c=b, y_c=A$ since $2b \times 2A$ window captures
main features of graph.

Can understand the features of the graph within this window. Reference fig. 1.2.

Windows of significantly different sizes would give poor representations of
functions. Reference fig. 1.3.

\subsubsection{Example 2}

$y = Ae^{-t/b}, t\ge0$ where $[y]=L$, $[t] = T$, $A$, $b$ constants.

Reference fig. 1.4. Characteristic scales are $t_c=b$, $y_c=A$ since window
$10b \times A$ captures main features.

\subsubsection{Example 3}

Fig. 1.5 has two sets of characteristic scales; two window sizes are needed to
represent the features, as shown in figs. 1.6 and 1.7.

\subsection{Procedure}
Given some (potentially indirect) definition of a graph, how do we indentify the
scales?

If $y=f(t)$ is defined by an equation involving constants $g_1, \ldots, g_m$,
then characteristic scales $t_c$, $y_c$ can be found by solving:

\begin{equation}
  \begin{aligned}
    t_c &= g_1^{\alpha_1}, \ldots, g_m^{\alpha_m} \\
    y_c &= g_1^{\beta_1}, \ldots, g_m^{\beta_m}
  \end{aligned}
\end{equation}

for $\alpha_1, \ldots, \alpha_m$ and $\beta_1, \ldots, \beta_m$.

\subsection{Examples}
\subsubsection{Example 1}

Find characteristic scales $t_c$, $y_c$ for $y=f(t)$ defined by

\begin{equation} \text{(IVP)}\quad
  \left\{
  \begin{aligned}
    \frac{dy}{dt} &= \eta y, \quad t\ge0 \\
    y_{(0)} &= \lambda \\
  \end{aligned} \right.
\end{equation}
$[y] = \Eta , [t]=T$. Dimensions of $\eta$, $\lambda$:
\begin{equation}
  \begin{aligned}
    \frac{\Eta}{T} &= [\eta]\Eta \rightarrow [\eta]=\frac{1}{T} \\
    \Eta &= [\lambda]
  \end{aligned}
\end{equation}
Scales for $t$, $y$:
\begin{multicols}{2}
  \begin{equation}
    \begin{aligned}
      t_c &= \eta^{\alpha_1} \lambda^{\alpha_2} \\
      \implies T &= T^{-\alpha_1}\Eta^{\alpha_2} \\
      \implies \alpha_1 &= -1,\quad \alpha_2 = 0 \\
      \implies t_c &= \frac{1}{\eta}\\
    \end{aligned}
  \end{equation}
\\
  \begin{equation}
    \begin{aligned}
      y_c &= \eta^{\beta_1} \lambda^{\beta_2} \\
      \implies \Eta &= T^{-\beta_1}\Eta^{\beta_2} \\
      \implies \beta_1 &= 0,\quad \beta_2 = 1 \\
      \implies y_c &= \lambda
    \end{aligned}
  \end{equation}
\end{multicols}
\subsubsection{Example 2}

Do the same for
\begin{equation}
  \left\{
  \begin{aligned}
    \frac{d^2y}{dt^2} &= \eta \frac{dy}{dt} + \mu y^2,\quad t \ge 0 \\
    y_{(0)} &= \lambda, \quad \frac{dy}{dt}(0) = 0
  \end{aligned} \right.
\end{equation}
Where $[y]=L$, $[t]=T$, $\eta$, $\mu$, $\lambda$ are constants.

Dimensions of $\eta$, $\mu$, $\lambda$:
\begin{equation}
  \begin{aligned}
    [\lambda] &= L \\
    [\eta]\frac{L}{T} &= \frac{L}{T^2} \quad\longrightarrow\quad [\eta] = \frac{1}{T} \\
    [\mu]L^2 &= \frac{L}{T^2} \quad\longrightarrow\quad [\mu] = \frac{1}{LT^2} \\
  \end{aligned}
\end{equation}

Scale for $t$:
\begin{equation}
  \begin{aligned}
    t_c &= \eta^{\alpha_1}\mu^{\alpha_2}\lambda^{\alpha_3} \\
    &= T^{-\alpha_1}{(L^{-1}T^{-2})}^{\alpha_2}L^{\alpha_3} \\
    \implies L^0T^1 &= L^{-\alpha_2+\alpha_3}T^{\alpha_1-2\alpha_2} \\
  \end{aligned}
\end{equation}

\begin{equation}
  \begin{pmatrix}
    -1 & -2 & 0 \\
    0 & -1 & 1 \\
  \end{pmatrix}
\begin{pmatrix}
  \alpha_1 \\
  \alpha_2 \\
  \alpha_3 \\
\end{pmatrix}
=
\begin{pmatrix}
  1\\0
\end{pmatrix}
\end{equation}

Where $\alpha_3$ is free.

\begin{equation}
  \begin{aligned}
    -\alpha_1-2\alpha_2 &= 1 \rightarrow \alpha_1=-2\alpha_2-1=-2\alpha_3-1\\
    -\alpha_2 + \alpha_3 = 0 \rightarrow \alpha_2&=\alpha_3
  \end{aligned}
\end{equation}

\begin{equation}
  \begin{pmatrix}
    \alpha_1 \\ \alpha_2 \\\alpha_3 \\
  \end{pmatrix}
  =
  \begin{pmatrix}
    -2\alpha_3-1 \\ \alpha_3 \\\alpha_3 \\
  \end{pmatrix}
  = \alpha_3
  \begin{pmatrix}
    -2 \\ 1 \\ 1
  \end{pmatrix}
  +
  \begin{pmatrix}
    -1 \\ 0 \\ 0
  \end{pmatrix}
\end{equation}

Where $\alpha_3$ is free. $\alpha_3=0, \quad \alpha_3\ne0$ gives two scales for
t\_c:

\begin{equation}
  \alpha_3=0, \quad
  \begin{pmatrix}
    -1 \\ 0 \\ 0
  \end{pmatrix}
\end{equation}

\begin{equation}
  \implies t_c = \frac{1}{\eta}.
\end{equation}

And

\begin{equation}
  \alpha_3 = \frac{-1}{2}, \quad
  \begin{pmatrix}
    \alpha_1 \\ \alpha_2 \\ \alpha_3 \\
  \end{pmatrix}
  =
  \begin{pmatrix}
    0 \\ -1/2 \\ -1/2
  \end{pmatrix}
\end{equation}

\begin{equation}
  \implies t_c = \frac{1}{\sqrt{\mu\lambda}}
\end{equation}

Same for $y_c$

\begin{equation}
  \begin{aligned}
    y_c &= \frac{\eta^2}{\mu} \\
    y_c &= \lambda
  \end{aligned}
\end{equation}

\newpage
\section{Dimensionless Forms}
\subsection{Definition}
Consider an equation involving vars $v_1,\ldots,v_k$. By the dimensionless form
of an equation with respect to scales $v_1^c,\ldots,v_k^c$ we mean the equation
obtained by the substitution $\overline{v_i}=v_i/v_i^c,\quad i=1,\ldots,k$ where
$\overline{v_i}$ is a dimensionless variable.

\subsection{Example}
\begin{equation}
  \left\{
  \begin{aligned}
    \frac{dy}{dt} &= \eta y+r, \quad t\ge0 \\
    y_{(0)} &= \lambda \\
  \end{aligned}
  \right.
\end{equation}

\begin{equation}
  [y] = \Theta,\quad[t]=T,\quad \eta,\lambda,r > 0 \text{ constants}
\end{equation}

Find dimensionless equation with respect to scales
$t_c=\frac{1}{\eta},\quad y_c=\lambda$.

Dimensionless vars
\begin{equation}
  \begin{aligned}
    \bar{y} &= y/y_c = y/\lambda,\quad & \bar{t} &= t/t_c = \eta t \\
    y &= \lambda \bar{y}, \quad & t&=\bar{t}/\eta \\
  \end{aligned}
\end{equation}

Derivitive relation
\begin{equation}
  \begin{aligned}
    \frac{d\bar{y}}{d\bar{t}} &= \frac{d\bar{y}}{dt} \cdot \frac{dt}{d\bar{t}}
  \end{aligned}
\end{equation}

Where $\frac{d\bar{y}}{dt} = \frac{d}{dt}(y/\lambda) = \frac{1}{\lambda}\frac{dy}{dt}$ and
$\frac{dt}{d\bar{t}} = \frac{1}{\eta}$. Thus

\begin{equation}
  \begin{aligned}
    \frac{d\bar{y}}{d\bar{t}} &= \frac{1}{\eta\lambda}\frac{dy}{dt} \\
    \therefore \frac{dy}{dt} &= \eta\lambda\frac{d\bar{y}}{d\bar{t}} \\
  \end{aligned}
\end{equation}

In general, $\frac{dy}{dt} = \frac{y_c}{t_c} \cdot \frac{d\bar{y}}{d\bar{t}}$.
Substituted into equations,

\begin{equation}
  \begin{aligned}
    \frac{dy}{dt} &= \eta y+r, &\quad t&\ge0 \\
    \eta\lambda\frac{d\bar{y}}{d\bar{t}} &= \eta\lambda\bar{y}+r, &\quad \bar{t}/\eta&\ge0 \\
    \frac{d\bar{y}}{d\bar{t}} &= \bar{y}+\frac{r}{\eta\lambda}, &\quad \bar{t}&\ge0 \\
    y &= \lambda, &\quad t&=0 \\
    \lambda\bar{y} &= \lambda, &\quad \bar{t}/\eta &= 0 \\
    \bar{y} &= 1, &\quad\bar{t}&=0. \\
  \end{aligned}
\end{equation}

We end up with dimensionless equations
\begin{equation} \left\{
  \begin{aligned}
    \frac{d\bar{y}}{d\bar{t}} &= \bar{y} + \bar{r}, &\quad \bar{t}&\ge0 \\
    \bar{y} &= 1, &\quad \bar{t} &= 0
  \end{aligned} \right.
\end{equation}

Where $\bar{r} = \frac{r}{\eta\lambda}$ is a dimensionless constant.

\subsection{Remarks}
\begin{enumerate}
\item Different choices of scales lead to different dimensionless equations
  \begin{equation}
    \begin{aligned}
      t_c &= \lambda/r \\
      y_c &= \lambda
    \end{aligned}
  \end{equation}
  Which leads to

  \begin{equation}
    \begin{aligned}
      \bar{t} &= t/t_c = rt/\lambda \\
      \bar{y} &= y/y_c = y/\lambda \\
    \end{aligned}
  \end{equation}


  Which leads to
  \begin{equation}
    \begin{aligned}
      \frac{d\bar{y}}{d\bar{t}} &= \bar{\eta}\bar{y} + 1, &\quad \bar{t}&\ge0 \\
      \bar{y} &= 1 &\quad \bar{t} &= 0 \\
      \bar{\eta} &= \frac{\eta\lambda}{r} \\
    \end{aligned}
  \end{equation}

  Where $\bar{y}$ and $\bar{\eta}$ are dimensionless.

\item Solutions of dimensionful and dimensionless equations are equivalent

  \begin{equation}
    \text{dimensioned solution}\quad\xleftrightarrow[\bar{t} = t/t_c]{\bar{y} = y/y_c}\quad
    \text{dimensionless solution}
  \end{equation}

\item Dimensionless equations allow comparisons to be made
\end{enumerate}

For example, we may not compare $a$, $b$, $c$, $d$ in \cref{eq:dim-full-eq}
since dimensions are different. By comparison, we may compare $\bar{a}$,
$\bar{b}$, $\bar{c}$, $\bar{d}$ in \cref{eq:dim-less-eq} since all variables are
dimensionless.

\begin{equation}
  \label{eq:dim-full-eq}
  \frac{d^2u}{dt^2} = au + bu^2 + ce^{-u/d}
\end{equation}

\begin{equation}
  \label{eq:dim-less-eq}
  \frac{d^2\bar{u}}{d\bar{t^2}} = \bar{a} \bar{u} + \bar{b} \bar{u^2} + \bar{c}e^{-\bar{u}/\bar{d}}
\end{equation}

\subsection{Example}
Vertical motion of a ball can be described using the variables in
\cref{fig:ball-model} and the equation

\begin{figure}
  \centering
  \begin{tabularx}{0.5\textwidth}{XX}
    variable & quantity \\ \hline
    $g$ & gravity \\
    $\eta$ & air resistance \\
  \end{tabularx}
  \caption{vertical motion of a ball model}
  \label{fig:ball-model}
\end{figure}

\begin{equation}
  \frac{d^2x}{dt^2} = -g - \eta\frac{dx}{dt}, \quad t\ge0
\end{equation}
\begin{equation}
  x=0,\quad \frac{dx}{dt} = v_0, \quad t=0
\end{equation}
$\eta$, $g$, $v_0$ are constants greater than zero.

\subsubsection*{Problem}
Under what condition on $\eta$ do we expect air effects to be small?

\subsubsection*{Solution}
First, choose scales. In the limiting case of no air resistance ($\eta=0$), the
solution is determined by $g$, $v_0$. Using these constants, we get
$t_c=v_0/g, x_c=v_0^2/g$. Define dimensionless variables
$$\bar{x}=x/x_c,\quad \bar{t}=t/t_c$$ and dimensionless
equations
\begin{equation}
  \frac{d^2\bar{x}}{d\bar{t^2}} = -\bar{g} -
  \bar{\eta}\frac{d\bar{x}}{d\bar{t}}, \quad \bar{t}\ge0
\end{equation}
Where dimensionless variables
\begin{equation}
  \begin{aligned}
    \bar{x} &= 0,\quad \frac{d\bar{x}}{d\bar{t}} = 1 \text{ at } \bar{t} = 0 \\
    \bar{g} &= 1, \quad \bar{\eta} = \frac{\eta v_0}{g} \\
  \end{aligned}
\end{equation}

Expect air effects to be small (when solution is viewed in window with sales
$t_c$, $x_c$) where $\bar{\eta} << \bar{g}$, $\frac{\eta v_0}{g} << 1$, and
$\eta << \frac{g}{v_0}$.

\newpage
\section{Case Study: Chemical Reactor}
\subsection{Problem}
Use scaling to study a model for a pair of reactions.

Chemical $A$, $B$ in fluid

\begin{equation}
  \begin{aligned}
    A + 2B &\xrightarrow{k_1} C \\
    2A &\xrightarrow{k_2} B \\
  \end{aligned}
\end{equation}

Where the flow rate is $\eta$ in gallons/second, tank volume is $V$, Chemical
$A$, $B$, $C$ in fluid. $a_{in}$, $b_{in}$ is the concentration of $A$, $B$ at
the tank inlet, $a_{out}$, $b_{out}$ is the concentration of $A$, $B$ at the
tank outlet, $k_1$, $k_2$ are the rate constants

\begin{equation}
  \begin{aligned}
    [a]=[b]&=\frac{1}{\text{vol}}, &\quad [k_1]&=\frac{\text{vol}^3}{\text{time}}, \\
    [k_2]&=\frac{\text{vol}^2}{\text{time}}, &\quad [\eta]&=\frac{\text{vol}}{\text{time}} \\
  \end{aligned}
\end{equation}

Under what conditions on $k_2$ would we expect the second reaction to be insignificant?

\subsection{Derivation of Model}

In a reaction $pA+qB \xrightarrow{k} \text{product}$, one product requires a
successful pairing of $pA$'s and $qB$'s. We assume

\begin{equation}
  \left(\frac{\text{\# successful pairings}}{\text{time}}\right) = \left(\text{reaction const}\right)\cdot
  \left(\text{\parbox{10em}{\centering \# possible pairings of pA's and qB's}}\right)
\end{equation}

The second parameter turns out to be
\begin{equation}
  \begin{aligned}
    &{\left(\text{total \# A's}\right)}^p \cdot {\left(\text{total \# B's}\right)}^q \\
    = &{(Va)}^p{(Vb)}^q \\
    = & \left(\text{reaction const}\right)V^{p+q} a^p b^q \\
  \end{aligned}
\end{equation}

Where $\left(\text{reaction const}\right)V^{p+q} = k$.

From this we get

\begin{equation}
  \begin{aligned}
    \left(\text{consumption rate of }A\right) &= \left(\frac{\text{\# }A}{\text{successful pairing}}\right)
    \left(\frac{\text{\# successful pairing}}{\text{time}}\right) \\
    &= (p) (ka^{p} b^q) \\
  \end{aligned}
\end{equation}

Similarly,
\begin{equation}
  \left(\text{consumption rate of }A\right) = (q)(ka^p b^q)
\end{equation}
This is the Law of Mass Action. Now, conservation of mass is applied to chemical
$A$ in tank states
\begin{equation}
  \begin{aligned}
    \od{}{t}\left(\text{A in tank}\right) = \left(\text{\parbox{5em}{\centering rate A enters tank}}\right) -
    \left(\text{\parbox{5em}{\centering rate A leaves tank}}\right) -
    \left(\text{\parbox{5em}{\centering rate of consumption of A}}\right) -
    \left(\text{\parbox{5em}{\centering rate of consumption of A}}\right)
  \end{aligned}
\end{equation}
Where the first ``rate of consumption of A'' is $A + 2B \xrightarrow{k_1} C$,
and the second ``rate of consumption of A'' is $2A \xrightarrow{k_2} B$.
\begin{equation}
  \begin{aligned}
    \od{}{t}(Va) = \eta a_{in} - \eta a - 1\cdot k_1ab^2 - 2\cdot k_2a^2
  \end{aligned}
\end{equation}
Conservation of mass for chemical $B$ states
\begin{equation}
  \od{}{t}\left(\text{B in tank}\right) = \left(\text{\parbox{5em}{\centering rate B enters tank}}\right) -
  \left(\text{\parbox{5em}{\centering rate B leaves tank}}\right) -
  \left(\text{\parbox{5em}{\centering rate of consumption of B}}\right) +
  \left(\text{\parbox{5em}{\centering rate of production}}\right)
\end{equation}
Where the rate of consumption is $A + 2B \xrightarrow{k_1} C$, and the rate of
production is $2A \xrightarrow{k_2} B$.
\begin{equation}
  \od{}{t}Vb = \eta b_{\text{in}} - \eta b - 2\cdot k_2a^2
\end{equation}
Now, combining all gives
\begin{equation}
  \begin{aligned}
    \od{a}{t} &= \frac{\eta}{V}(a_{in}-a) - \frac{k_1}{V}ab^2 - \frac{2k_2}{V}a^2, &\quad t&\ge0 \\
    \od{b}{t} &= \frac{\eta}{V}(b_{in}-b) - \frac{2k_1}{V}ab^2 + \frac{k_2}{V}a^2, &\quad t&\ge0 \\
  \end{aligned}
\end{equation}
Where
\begin{equation}
  a_{(0)}=a_0,\qquad b_{(0)}=b_0,\qquad \text{ at } t=0
\end{equation}
$\eta$, $V$, $k_1$, $k_2$, $a_{in}$, $b_{in}$, $a_0$, $b_0$ are constants $>0$.

\subsection{Scales for $t$, $a$, $b$}
In limiting case of no reaction \#2, ($k_2=0$), the remaining consts are
\begin{equation}
  \frac{\eta}{V},\quad \frac{k_1}{V}, \quad a_{in}, \quad b_{in}, \quad a_0, \quad b_0
\end{equation}
Where
\begin{figure}
  \centering
  \begin{tabularx}{0.5\textwidth}{XXX}
    Variable & Dimension \\ \hline
    $\eta/V$ & $T^{-1}$  \\
    $k_1/V$ & $LT^{-1}$ \\
  \end{tabularx}
  \caption{Variable mappings for \#14}
  \label{fig:14-var-mappings}
\end{figure}
We choose scales
\begin{equation}
  \begin{aligned}
    a_c &= a_{in}, Y\quad b_c &= b_{in} \\
    t_c &= \frac{V}{k_1}\frac{1}{a_{in}b_{in}}
  \end{aligned}
\end{equation}

\subsection{Dimensionless Form of Equations}
\begin{equation}
  \begin{aligned}
    \bar{a} &= \frac{a}{a_c}, &\quad a &= a_c\bar{a} \\
    \bar{b} &= \frac{b}{b_c}, &\quad b &= b_c\bar{b} \\
    \bar{t} &= \frac{t}{t_c}, &\quad t &= t_c\bar{t} \\
  \end{aligned}
\end{equation}

Derivatives
\begin{equation}
  \begin{aligned}
    \od{\bar{a}}{\bar{t}} &= \od{\bar{a}}{t} \cdot \od{t}{\bar{t}} &= \frac{t_c}{a_c}\cdot\od{a}{t} \\
    &= \frac{1}{a_c}\od{a}{t} \cdot t_c \\
    \od{\bar{b}}{\bar{t}} &= \od{\bar{b}}{t} \cdot \od{t}{\bar{t}} &= \frac{t_c}{b_c}\cdot\od{b}{t} \\
  \end{aligned}
\end{equation}

Equations
\begin{equation}
  \begin{aligned}
    \frac{a_c}{t_c}\cdot\od{\bar{a}}{\bar{t}} &= \frac{\eta}{V}(a_{in}-a_c\bar{a})-\frac{k_1}{V}a_c\bar{a}b_c^2\bar{b}^2 \\
    &- \frac{2k_2}{V}a_c^2a^{-2},\qquad t,\bar{t}\ge 0 \\
  \end{aligned}
\end{equation}

\begin{equation}
  \begin{aligned}
    \frac{b_c}{t_c}\cdot\od{\bar{b}}{\bar{t}} &= \frac{\eta}{V}(b_{in}-b_c\bar{b})-\frac{2k_1}{V}a_c\bar{a}b_c^2\bar{b}^2 \\
    &- \frac{k_2}{V}a_c^2a^{-2},\qquad t,\bar{t}\ge 0 \\
  \end{aligned}
\end{equation}

Where

\begin{equation}
  a_c\bar{a} = a_0, \qquad b_c\bar{b} = b_0, \quad\text{ at } t_c\bar{t} = 0
\end{equation}

Simplifying gives

\begin{equation}
  \begin{aligned}
    \od{\bar{a}}{\bar{t}} &= \mu(1-\bar{a}) - \sigma\bar{a}\bar{b}^2 - 2\lambda\bar{a}^2, &\quad t&\ge0 \\
    \od{\bar{b}}{\bar{t}} &= \mu(1-\bar{b}) - 2\bar{a}\bar{b}^2 - 2\frac{\lambda}{\sigma}\bar{a}^2, &\quad t&\ge0 \\
  \end{aligned}
\end{equation}

Where
\begin{equation}
  \bar{a} = \frac{a_0}{a_{in}}, \quad \bar{b} = \frac{b_0}{b_{in}} \quad \text{ at } t=0
\end{equation}


\subsection{Observation}
We expect reaction \#2 terms to have small influence on solution (solution
viewed in scales $t_c, a_c, b_c$) when $\bar{a}$ equation
$$2\lambda << \sigma, \mu \quad\longrightarrow\quad \lambda << \frac{\sigma}{2},\frac{\mu}{2},$$
$\bar{b}$ equation
$$\frac{\lambda}{\sigma} << 2, \mu \quad\longrightarrow\quad \lambda << 2\sigma, \mu\sigma.$$
All conditions are met when
$$\lambda << \frac{\mu}{2},\; \frac{\sigma}{2},\;\mu\sigma$$ which implies
$$k_2 << \frac{\eta}{2a_{in}},\; \frac{k_1b_{in}^2}{2a_{in}},\; \frac{\eta b_{in}}{a_{in}^2}$$

\newpage
\section{Dynamical Systems in 1 Dimension}
\subsection{Setup}
A dynamical system for a variable $u=u(t)$ is a system in the form
\begin{equation} \text{IVP}\left\{
  \begin{aligned}
    \od{u}{t} &= f(u),&\qquad t\ge t_0 \\
    u(t_0) &= u_0 \\
  \end{aligned}\right.
\end{equation}
We seek to understand behavior of solutions for different $u_0$.

Solutions can be viewed in two ways
\begin{enumerate}
\item $u$ vs $t$ graph. Reference fig. 5.1
\item motion of point $u(t)$ on $u$-axis. Reference fig. 5.2
\end{enumerate}

\subsection{Solvability Theorem}
Consider (IVP) and assume $f(u)$, $f'(u)$ are continuous for all $u_0$.
Then:

\begin{enumerate}
\item $\exists$ a unique solution $u(t)$ for any given $t_0$, $u_0$.
\item Solution $u(t)$ is defined for $t$ in some interval $[t_0,t_0+T]$
  where $T$ depends on $u_0$.
\item Either $T=\infty$ or $T$ is finite; if finite, then
  $|u(t)| \rightarrow \infty$ as $t\rightarrow (t_0+T)$.
\item For a given $t_0$, solutions with different $u_0$ cannot intersect or touch.
\end{enumerate}

\subsection{Examples}
\subsubsection{Example 1}

\begin{equation} \left\{
  \begin{aligned}
    \od{u}{t} &= 2u,&\qquad t\ge0 \\
    u(0) &= u_0 \\
  \end{aligned}\right.
\end{equation}

\begin{equation}
  \begin{aligned}
    f(u) &= 2u \\
    f'(u) &= 2 \\
  \end{aligned}
\end{equation}
This implies $f(u)$ and $f'(u)$ is continuous.

Therefore the solution is defined for all $t\in[0,\infty)$ so $T=\infty$ in this
case.
\begin{equation}
  u(t) = u_0e^{2t}
\end{equation}
Reference figs. 5.3, 5.4.

\subsubsection{Example 2}
\begin{equation} \left\{
  \begin{aligned}
    \od{u}{t} &= u^2, &\qquad t\ge0 \\
    u(0) &= u_0 \\
  \end{aligned} \right.
\end{equation}

\begin{equation}
  \begin{aligned}
    f(u) &= u^2 \\
    f'(u) &= 2u \\
  \end{aligned}
\end{equation}
This implies $f(u)$ and $f'(u)$ is continuous. The solution is defined for
$t\in[0,\frac{1}{u_0})$ if $u_0>0$ and for $t\in[0,\infty)$ if $u_0\le0$.
\begin{equation}
  \begin{aligned}
    u^{-2} \dd{u} &= \dd{t} \\
    \int u^{-2} \dd{u} &= \int \dd{t} \\
    -u^{-1} &= t+C \\
    \implies u(t) &= \frac{-1}{t+C} \\
  \end{aligned}
\end{equation}

$$u(0) = u_0 \rightarrow u_0 = \frac{-1}{C}$$
$$\therefore u(t) = \frac{-1}{t-\frac{1}{u_0}} = \frac{u_0}{1-u_0t}$$

\newpage
\section{Definition}
An equilibrium or steady-state solutoin of (IVP) is a solution of the form

$$u(t) \equiv u_*,\qquad t\in[t_0,\infty)$$
Since $\od{u}{t}=f(u)$ we have
\begin{equation}
  \left[
    u_* \;\text{equilibrium solution}
    \right]
  \Longleftrightarrow
  \left[
    f(u_*)=0
    \right]
\end{equation}

\subsection{Examples}
\subsubsection{Example 1}
\begin{equation} \left\{
  \begin{aligned}
    \od{u}{t} &= 2u, &\qquad t\ge0 \\
    u(0) &= u_0 \\
  \end{aligned} \right.
\end{equation}
\begin{equation}
  \begin{aligned}
    f(u) &= 2u \\
    f(u_*) &= 0 \\
    \implies u_* &= 0 \\
  \end{aligned}
\end{equation}

\subsubsection{Example 2}
\begin{equation} \left\{
  \begin{aligned}
    \od{u}{t} &= u^2-3u &\qquad t\ge0 \\
    u(0) &= u_0 \\
  \end{aligned} \right.
\end{equation}
\begin{equation}
  \begin{aligned}
    f(u) &= u^2-3u \\
    f(u_*) &= 0 \rightarrow u^2-3u_* = 0 \\
    \implies u_* &= 0,3 \\
  \end{aligned}
\end{equation}
This has two constant equilibrium solutions which exist for all time.
Reference fig. 5.7.

\subsection{Characterization Theorem}
Consider (IVP) with $f(u)$, $f'(u)$ continuous and define the sets
\begin{equation}
  \begin{aligned}
    \text{Equil} &= {u | f(u) = 0} \\
    I^+ &= {u | f(u) > 0} \\
    I^- &= {u | f(u) < 0} \\
  \end{aligned}
\end{equation}
Then $u\in\text{Equil} \implies u(t)\equiv u_0$ is an equal solution.
$u_0\in I^+ \implies u(t)$ increases as it is defined. $u_0\in I^-
\implies u(t)$ decreases while it is defined. Since solutions with
different $u_0$ cannot interset, this gives a complete portrait of
behavior.

\subsubsection{Example 1}
Sketch time and phase portait of solutions for
\begin{equation} \left\{
  \begin{aligned}
    \od{u}{t} &= u^3-8u^2,&\qquad t\ge0 \\
    u(0) &= u_0 \\
  \end{aligned} \right.
\end{equation}

$$f(u)=u^3-8u^2 = u^2(u-8)$$ Reference figs. 5.8--5.10.

\subsubsection{Example 2}
\begin{equation} \left\{
  \begin{aligned}
    \od{u}{t} &= u-e^{-u},&\qquad t\ge0 \\
    u(0) &= u_0 \\
  \end{aligned} \right.
\end{equation}
Does $f(u)=u-e^{-u}$ have any roots? Reference figs. 5.11, 5.12.

\newpage
\section{Stability of Equilibria}
\subsubsection*{Recall}
We consider the system where $f(u)$, $f'(u)$ continuous.
\begin{equation} \text{(IVP)} \quad
  \left\{
  \begin{aligned}
    \od{u}{t}&=f(u),\quad t&\ge t_0, \\
    u(t_0) &= u_0 \\
  \end{aligned} \right.
\end{equation}

\subsection{Definition}
\begin{enumerate}
\item An equilibrium solution $u_*$ is called asymptotically stable if there is
  an interval $I$ such that
  \begin{equation}
    u(t) \rightarrow u_* \text{ as } t\rightarrow\infty
  \end{equation}
  For every $u_0\in I$. Reference fig. 7.1. This is usually called an
  ``attractor'' or ``sink''.
\item An equilibrium solution $u_*$ is called neutrally stable if for any given
  interval $G$ there is an interval $I \subset G$ such that $u(t) \in G$ for all
  $t\ge t_0$ for every $u_0\in I$. Reference fig. 7.2.
\item An equilibrium solution $u_*$ is called unstable if it is not
  asymptotically or neutrally stable. Reference figs. 7.3, 7.4. This is called a
  ``repellor'', or ``source''.
\end{enumerate}

Asymptotically stable $\implies$ neutrally stable, but the converse is not true.

\subsection{Examples}
\subsubsection*{Example 1}
Find equilibrium solutions; determine stability
\begin{equation} \text{(IVP)}\quad
  \left\{
  \begin{aligned}
    \od{u}{t} &= 4u^2-u^3,&\quad t&\ge0, \\ \quad u(0)&=u_0 \\
  \end{aligned} \right.
\end{equation}

\begin{equation}
  \begin{aligned}
    f(u_*) &= 0 &\rightarrow 4u_*^2-u_*^3 &= 0 \\
    & & u_*^2(4-u_*) &= 0 \\
    & & u_* &= 0, 4 \\
  \end{aligned}
\end{equation}

Reference figs. 7.5, 7.6. However, this ``sign table'' does not generalize to
many variables.

\subsection{Derivative Method For Stablity}
\label{sec:derivative-method}
This method generalizes for many variables. We'll start by showing a single
variable. Let $u_*$ be an equilibrium solution and let $\lambda_*=f'(u_*)$.
Then
\begin{equation}
  \begin{aligned}
    \lambda_* < 0 &\implies u_* \text{ is asymptotically stable (attractor)} \\
    \lambda_* > 0 &\implies u_* \text{ is unstable (repeller)} \\
    \lambda_* = 0 &\implies u_* \text{ no conclusion} \\
  \end{aligned}
\end{equation}

\subsubsection*{Idea}
We want to know what $f$ looks like near $u_*$. Near $u_*$, we can use the
Taylor series to write
\begin{equation}
  f(u) = f(u_*) + f'(u_*)(u-u_*) + R(u-u_*)
\end{equation}


From this we can construct sign tables

When $\lambda_*<0$, $f(u)$ is positive until it passes through $u_*$, after
which the sign becomes negative. Thus $f$ has negative slope. Therefore $u_*$ is
asympotitally stable. Reference figs. 7.7, 7.8.

When $\lambda_*>0$, $f(u)$ is negative until it passes through $u_*$, after
which the sign becomes positive. Thus $f$ has positive slope. Therefore $u_*$
is unstable. Reference figs. 7.9, 7.10.

When $\lambda_*=0$, no conclusion can be made. Reference fig. 7.11.

\subsection{Examples}
\subsubsection*{Example 1}
\begin{equation} \text{(IVP)} \quad
  \left\{
  \begin{aligned}
    \od{u}{t} &= \sin u, &\quad t&\ge 0 \\
    u(0) &= u_0 \\
  \end{aligned} \right.
\end{equation}

\begin{equation}
  \begin{aligned}
    f(u_*) = 0 \rightarrow \sin u_* & = 0 \\
    u_*  &= n\pi \\
  \end{aligned}
\end{equation}

\begin{equation}
  \begin{aligned}
    f'(u) &= \cos u \\
    f'(n\pi) &= \cos n\pi \\
    &= \left\{
    \begin{aligned}
      +1,\quad n&=0,\pm2,\pm4,\ldots \\
      -1,\quad n&=\pm1,\pm3,\ldots \\
    \end{aligned}
    \right.
  \end{aligned}
\end{equation}
Therefore, $u_*=n\pi$ is asymptotically stable if $n=\pm1,\pm3,\ldots$, and
$u_*=n\pi$ is unstable if $n=0,\pm2,\pm3,\ldots$.

\subsection{Lyapunov Method For Stablity}
Let $u_*$ be an equilibrium solution. Suppose we can find a function $E(u)$ such
that
\begin{enumerate}
\item $E(u)$ has a strict local minimum at $u=u_*$.
\item $E'(u)f(u)$ does not change sign in some interval $I$ containing $u_*$.
\end{enumerate}

\begin{itemize}
\item If $E'(u)f(u)<0\,\; \forall u\in I, u\ne u_* \implies u_* $ is asymptotically
stable.
\item If $E'(u)f(u)\le 0,\; \forall u\in I, u\ne u_* \implies u_*$ is neutrally
stable.
\item If $E'(u)f(u)> 0,\; \forall u\in I, u\ne u_* \implies u_*$ is unstable
\end{itemize}

\subsubsection*{Idea}
Suppose $E(u)$ has a minimum at $u=u_*$ and $E'(u)f(u)<0$ for
$u\in I, u\ne u_*$. If $u(t)$ is a solution of $\od{u}{t}=f(u)$, then

\begin{equation}
  \begin{aligned}
    \od{}{t}E(u(t)) &= E'(u(t))\od{u}{t}(t) \\
    &= E'(u(t))f(u(t)) \\
    &<0 \\
  \end{aligned}
\end{equation}
This says that $E(u(t))$ is strictly negative and decreases in time. Thus $u(t)$
gets closer to $u_*$ in time. Reference fig. 7.12.

\subsection{Example}
\begin{equation}
  \begin{aligned}
    \od{u}{t} &= -5u^3,\quad t\ge0 \\
    u(0) &= u_0 \\
  \end{aligned}
\end{equation}
$u_*=0$ is equilibrium. Consider $E(u)=u^2$.

$$E'(u)f(u) = (2u)(-5u^3) = -10u^4 < 0 \qquad\forall\;u\in(-\infty,\infty),$$
$u\ne u_*$. $u_*$ is asymptotically stable.

\newpage
\section{Bifurcation of Equilibria}
\subsection{Definition}
Consider a dynamical system of the form
\begin{equation} \left\{
  \begin{aligned}
    \od{u}{t} &= f(u,h), &\quad t\ge t_0 \\
    u(u) &= u_0 &\quad h \in \mathbb{R} \\
  \end{aligned} \right.
\end{equation}
Where $h$ is some constant of interest. The equilibrium solutions $u_*$
satisfy $$f(u_*,h)=0.$$ The set of all $(u_*,h)$ satisfying this equation, with
stability indicated, is called a bifurcation diagram.

\subsection{Examples}
\subsubsection*{Example 1}
\begin{equation}
  \begin{aligned}
    \od{u}{t} &= u^3-uh, &\quad t&\ge0 \\
    u(0) &= u_0, &\quad h&\in(-\infty,\infty) \\
  \end{aligned}
\end{equation}
Equilibria solution
\begin{equation}
  \begin{aligned}
    f(u,h) \quad \rightarrow \quad & u^3-uh=0 \\
    & u(u^2-h)=0 \\
  \end{aligned}
\end{equation}
$u_*=0$ is an equilibrium solution for $h\in(-\infty,\infty)$. $u_*=\pm\sqrt{h}$
are equilibria for all $h\ge0$. This solution has 1 or 3 equilibria depending on
$h$. Reference fig. 8.1.

\subsection{Definition of Stability}
Let $$\lambda_*=\pd{f}{u}(u_*,h)=3u_*^2-h.$$ Recall
\cref{sec:derivative-method}.

\begin{enumerate}
\item $u_*=0$, $h\in(-\infty,\infty)$, then $\lambda_*=-h$.
  $\therefore u_*=0$ is asymtotically stable if $h>0$.
  $u_*$ is unstable for $h<0$.

  For $h=0$, consider $f(u,0)=u^3$. Reference fig. 8.2. Therefore $u_*=0$ is
  unstable if $h=0$.
\item $u_*=+\sqrt{h},\, h>0$, then $\lambda_*=3u_*^2-h=2h$.
  $\therefore u_*=+\sqrt{h}$ is unstable for $h>0$.
\item $u_*=-\sqrt{h},\, h>0$, then $\lambda_*=3u_*^2-h=2h$.
  $\therefore u_*=-\sqrt{h}$ is unstable for $h>0$.
\end{enumerate}

We can put all of this information into a bifurcation diagram for this system.
Reference fig. 8.3. We have multiple solution portraits. For $h\le0$, reference
fig. 8.4. For $h>0$, reference fig. 8.5.

\subsubsection*{Example}
\begin{equation} \left\{
  \begin{aligned}
    \od{u}{t} &= hu-(u-2)(a-u), &\quad t&\ge0 \\
    u(0) &= u_0, &\quad a&>2, h>0 \text{ constants}. \\
  \end{aligned} \right.
\end{equation}
Find the bifucation diagram in terms of $h$ (for fixed $a$).

The equilibrium solution $$f(u,h)=0\quad\rightarrow\quad hu-(u-2)(a-u)=0.$$
First plot one easily available solution $hu=(u-2)(a-u)$. Reference fig. 8.6.
The $hu$ line may or may not intersect the $(u-2)(a-u)$ parabola.

\begin{description}
\item[Case 1] $h=h_{\text{tang}}$ (some number $>0$). Reference fig. 8.7. In
  this case we have one equilibrium solution $u_*=u_{\text{tang}}$
  ($2<u_{\text{tang}}<a$), where $u_{\text{tang}}$ is some number. Reference the
  sign table in fig. 8.8. This equilibrium solution is hyperbolic and is
  therefore unstable.
\item[Case 2] $0<h<h_{\text{tang}}$. Reference fig. 8.9. In this case we have
  two equilibria, $u_*=u_L,u_R$ ($2<u_L<u_R<a$). Reference the sign table in
  fig. 8.10. $u_L$ is asymptotically stable; $u_R$ is unstable.
  $u_L\rightarrow2$, $u_R\rightarrow a$ as $h\rightarrow0$ and
  $u_L,u_R\rightarrow u_{\text{tang}}$ as $h\rightarrow h_{\text{tang}}$.
\item[Case 3] $h>h_{\text{tang}}$. Reference fig. 8.11. No equilibrium
  solutions; all solutions increase. Reference the sign table in fig. 8.12.
\end{description}

Reference the Bifurcation Diagram in fig. 8.13. Solution portraits:
\begin{itemize}
\item $0<h<h_{\text{tang}}$. Reference fig. 8.14.
\item $h=h_{\text{tang}}$. Reference fig. 8.15.
\item $h>h_{\text{tang}}$. Reference fig. 8.16.
\end{itemize}

\newpage
\section{Case Study: Plant-Herbivore System}
\subsection{Problem}
Use bifurcation analysis to study model of a simple ecosystem.
We'll define $P(t)$ as  number of plants, and $q$ as the number of herbivores.
Assume $q$ is constant.

$$[p]=\text{Plant}\quad[q]=\text{Herbivore}\quad[t]=\text{Time}$$

\subsection{Basic Form of Model}
\begin{equation}
  \left( \text{\parbox{10em}{net rate of change of \# plants}} \right) = \left( \text{\parbox{10em}{\centering rate of change due to births, deaths}} \right) - \left( \text{\parbox{10em}{\centering rate of consumption by herbivores}} \right).
\end{equation}

Is equivalent to $$\od{p}{t}=F-G*q$$
We'll define:
\begin{equation}
  \begin{aligned}
    F &= \text{net birth rate} &\quad [F] &= \frac{\text{Plants}}{\text{Time}} \\
    G &=  \text{net consumption rate per herbivore} &\quad [G]&= \frac{\text{Plants}}{\text{Time} \cdot \text{Herbivore}} \\
  \end{aligned}
\end{equation}
Ecologists have suggested different models for $F$, $G$.

\subsection{Examples For F}
\begin{enumerate}
\item Malthus Model
$$F = rp, \quad r>0\;\text{const}.$$
In this case,
$$\left(\text{more plants}\right) \implies \left(\text{\parbox{15em}{\centering more plants
    born per unit time}}\right)$$
Reference fig. 9.1.
\item Logistic Model
$$F= r(p-p^2/k, \quad r,k>0\;\text{const}.$$
In this case,
$$\left(\text{too many plants}\right) \implies
\left(\text{\parbox{15em}{\centering net birth rate becomes negative due to
      competition, overcrowding}}\right)$$
Reference fig. 9.2.
\item Holling Type I / Lotka-Volterra
$$G = ap,\quad a>0\;\text{const}.$$
In this case,
$$\left(\text{more plants}\right) \implies \left(\text{\parbox{15em}{\centering
      each herbivore consumes more per unit time}}\right)$$
Reference fig. 9.3.
\item Holling Type II
$$G = \frac{ap}{1-bp}\quad a,b>0\;\text{const}.$$
In this case,
$$\left(\text{more plants}\right) \implies \left(\text{\parbox{15em}{\centering
      each herbivore consumes more per unit time, up to a limit}}\right)$$
Reference fig. 9.4.
\item Holling Type 3
$$G=\frac{ap^2}{1+bp^2},\quad a,b>0\;\text{const}$$
Reference fig. 9.5.
\end{enumerate}

\subsection{Catalog of Models}
\begin{description}
\item[$F1, G2$:] $$\od{p}{t}=rp-\frac{ap}{a+bp}\cdot q$$
\item[$F2, G3$:] $$\od{p}{t} = rp(1-p/k)-\frac{ap^2}{1-bp^2}\cdot q$$
\todo[etc]
\end{description}

\subsection{Analysis of the $F1,G2$ Model}
Let $t_c=1/r$, $p_c=1/b$, and define $\tau=t/t_c$, $u=p/p_c$. Then the model
becomes $$\od{u}{\tau}=u-\frac{hu}{1+u},\quad h>0\;\text{const}\; (h=aq/r)$$ We
only consider solutions with $u\ge0$, since it doesn't make physical sense
otherwise.

\subsubsection*{Equilibria}
\begin{equation}
  \begin{aligned}
    f(u,h)=u-\frac{hu}{1-u}&=\frac{u(1+u)-hu}{1+u} \\
    &= \frac{g(u,h)}{1+u} \\
  \end{aligned}
\end{equation}
Since $f=0$ iff $g=0$, we get
\begin{equation}
  \begin{aligned}
    f=0 \quad\Longleftrightarrow &\quad u(1+u)-hu=0 \\
    &\quad u(1-h+u)=0 \\
  \end{aligned}
\end{equation}

\begin{equation}
  \begin{aligned}
    \therefore\; u_*&=0 \quad\forall\;h>0  \\
    u_*&= h-1 \quad\forall\;h>1 \\
  \end{aligned}
\end{equation}

\subsubsection*{Stability}
Since $f=\frac{g}{1+u}$ and $u\ge0$, we get $\text{sign}(f)=\text{sign}(g)$.
Hence we consider table for $g$.

\begin{description}
\item[$0<h\le1$:] One equilibrium $u_*=0$, which is unstable. Reference fig. 9.6.
\item[$h>0$:] Two equilibria $u_*=0, h-1$, which are stable, unstable
  respectively. Reference fig. 9.7.
\end{description}
Reference fig. 9.8 for bifurcation diagram. Reference figs. 9.9, 9.10 for
solution pics.

\subsubsection*{Question}
Suppose $u_0=1/3$ is given. What happens to plant population for different $h$?
Reference fig. 8.11.

\begin{equation}
  \begin{aligned}
    u_0 = \frac{1}{3} \quad\longrightarrow \quad& \text{pop grows for}\; 0<h<4/3 \\
    & \text{pop remains const if}\; h=4/3 \\
    & \text{pop decays to extinction for any}\; h>4/3 \\
  \end{aligned}
\end{equation}

\section{Dynamical Systems in 2D}
\todo[Write section!]
\section{Phase Diagrams for Linear Systems}
\subsection{Definition}
By a linear dynamical system for $x=x(t)$, and $y=y(t)$, we mean
\begin{equation}
  \begin{aligned}
    \od{x}{t} &= ax+by \\
    \od{y}{t} &= cx+dy \\
  \end{aligned}
\end{equation}

Where $a,b,c,d$ are constants.

Introducing $V= \begin{pmatrix} x \\ y \end{pmatrix}$ and $A = \begin{pmatrix} a
  & b \\ c & d \\ \end{pmatrix}$, then the above becomes $$\od{v}{t}=Av.$$

\subsection{Equilibria}
Equlibrium solutions $v(t)=v_*$ (const) satisfy $$Av_*=0.$$

\begin{itemize}
\item If $\det A\ne0$, then $v_*=\begin{pmatrix} x_* \\
      y_* \end{pmatrix}=0$ is the only equlibrium.
\item If $\det A=0$, then $v_*=\hat{v}$ is an equilibrium solution for any null
  vector $\hat{v}$ of A.

  \begin{equation}
    \begin{aligned}
      \dot{x}&=x-y \quad& A&=
      \begin{pmatrix}
        1 & -1 \\ 2 & -2 \\
      \end{pmatrix}
      \quad A\hat{v} = 0 \\
      \dot{y} &= 2x-2y \quad& \hat{v} &= \alpha
      \begin{pmatrix}
        1 \\ 1
      \end{pmatrix},\; \alpha\;\text{free}
    \end{aligned}
  \end{equation}
\end{itemize}

$
\begin{pmatrix}
  x_* \\ y_* \\
\end{pmatrix}
$ is an equilibrium solution for any $\alpha$.

\subsection{General Solution}
We consider a non-degenerate linear system
\begin{equation} \text{(IVP)}\;\left\{
  \begin{aligned}
    \od{v}{t} &= Av, \quad& t&\ge0 \\
    v(t_0)=v_0 \\
  \end{aligned} \right.
\end{equation}

Where $\det A\ne0, v_*=0$ only equilibrium. Assuming solution in the form
$v(t)=e^{\lambda t}\hat{u}$, ($\lambda$, $\hat{u}$ constants), we gets

\begin{equation}
  \begin{aligned}
    \od{}{t}(e^{\lambda t}\hat{u}) &= A(e^{\lambda t}\hat{u}) \\
    e^{\lambda t}\lambda\hat{u} &= e^{\lambda t}A\hat{u} \\
    A\hat{u} &= \lambda\hat{u}
  \end{aligned}
\end{equation}

\begin{equation}
  \left[ \text{\parbox{12em}{\centering $v(t)=e^{\lambda t}\hat{u}$ is a solution}}
 \right] \Leftrightarrow
  \left[ \parbox{12em}{\centering \text{$\lambda$ is an eigenvalue of $A$} \\
      \text{$\hat{u}$ is an eigenvector of $A$}} \right]
\end{equation}

Hence general solution of system can be built from eigenvalue/eigenvectors of
$\det A\ne0$ implies $\lambda\ne0$.

\subsection{Case I}
Suppose $A$ has real, distinct eigenvalues $\lambda_1\ne\lambda_2$ with
eigenvectors $\hat{u}_1,\hat{u}_2$. Then general solution isfy

$$v(t)= C_1e^{\lambda_1 t}\hat{u_1} + C_2e^{\lambda_1 t}\hat{u_1}$$ where
$C_1,C_2$ are aribtrary constants.

\begin{itemize}
\item If $\lambda_1<0$ and $\lambda_2<0$, then equilibrium $v_*=0$ is
  asymptotically stable. Reference fig. 11.1.
\item If $\lambda_1>0$ and $\lambda_2<0$, then equilibrium $v_*=0$ is
  asymptotically unstable. Reference fig. 11.2. This solution is called
  ``hyperbolic'' or a ``saddle''.
\item If $\lambda_1>0$ and $\lambda_2>0$, then equilibrium $v_*=0$ is
  asymptotically unstable. Reference fig. 11.3.
\end{itemize}

\subsubsection{Example}
\begin{equation}
  \begin{aligned}
    \od{x}{y} &= x+y,\quad& \od{y}{t} &=4x-2y, \quad t\ge0 \\
    x(0) &= 2, \quad& y(0) &= -3.
  \end{aligned}
\end{equation}

The eigenvalues and eigenvectors take the form
$$A = \begin{pmatrix} 1 & 1 \\ 4 & -2 \\ \end{pmatrix}$$
Where $$\det(A-\lambda I)=0.$$

\begin{equation}
  \det \begin{pmatrix}
    1-\lambda & 1 \\
    4 & -2-\lambda \\
  \end{pmatrix}
\end{equation}
$$(1-\lambda)(-2-\lambda)-4=0$$
$$\lambda_1=2,\quad \lambda_2=-3.$$

\begin{equation}
  \begin{aligned}
    A\hat{u}_1 &= \lambda_1\hat{u}_1 \\
    (A-\lambda_1I)\hat{u}_1 &= 0 \\
    \begin{pmatrix}
      -1 & 1 \\ 4 & -4 \\
    \end{pmatrix}
    \begin{pmatrix}
      \hat{x} \\ \hat{y} \\
    \end{pmatrix} &=
    \begin{pmatrix}
      0 \\ 0
    \end{pmatrix} \\
    \hat{u}_1 &= \hat{y}
    \begin{pmatrix}
      1 \\ 1
    \end{pmatrix}, \quad \hat{y}\;\text{free}
  \end{aligned}
\end{equation}

Take $\hat{u}_1=\begin{pmatrix} 1 \\ 1 \end{pmatrix}$.
\begin{equation}
  \begin{aligned}
    A\hat{u}_2 &= \lambda_2\hat{u}_2 \\
    (A-\lambda_2I)\hat{u}_2 &= 0 \\
    \begin{pmatrix}
      4 & 4 \\ 1 & 1 \\
    \end{pmatrix}
    \begin{pmatrix}
      \hat{x} \\ \hat{y} \\
    \end{pmatrix} &=
    \begin{pmatrix}
      0 \\ 0
    \end{pmatrix} \\
    \hat{u}_2 &= \hat{y}
    \begin{pmatrix}
      -1/4 \\ 1
    \end{pmatrix}, \quad \hat{y}\;\text{free}
  \end{aligned}
\end{equation}

Take $\hat{u}_2=\begin{pmatrix} -1 \\ 4 \end{pmatrix}$.

\begin{equation}
  \begin{aligned}
    x(t) &= C_1e^{2t}-C_2e^{-3t} \\
    y(t) &= C_1e^{2t}-4C_2e^{-3t} \\
  \end{aligned}
\end{equation}

The phase diagram looks like fig. 11.4 and is hyperbolic.

The initial condition $x(0)=2$, $y(y)=-3$ makes the above a linear system.
Solving that system yields $C_1=1$, $C_2=-1$. Therefore

\begin{equation}
  \begin{aligned}
    x(t) &= e^{2t}+e^{-3t} \\
    y(t) &= e^{2t}-4e^{-3t} \\
  \end{aligned}
\end{equation}

Reference fig. 11.5.

\subsection{Case II}
Suppose $A$ has real, repeated eigenvalues $\lambda_1=\lambda_2=\lambda$. In
this case, one of the following holds:
\begin{enumerate}
\item $A$ has two independent eigenvectors $\hat{u_1}, \hat{u_2}$ and the
  general solution is
$$v(t)=C_1e^{\lambda t}\hat{u_1}+C_2e^{\lambda t}\hat{u_2}.$$
\label{itm:case-2-item-1}
\item A has only one independent eigenvector $\hat{u}$ and the general solution
  is $$v(t)=C_1e^{\lambda t}\hat{u}+C_2e^{\lambda t}(t\hat{u}+\hat{w}),$$ where
  $\hat{w}$ is any solution of $(A-\lambda I)\hat{w}=\hat{u}$.
\label{itm:case-2-item-2}
\end{enumerate}

In \cref{case_2-item-1}, the equilibrium $v_*=0$ is asympotitally stable (node)
if $\lambda<0$; unstable (node) if $\lambda>0$. Reference fig. 11.6.

In \cref{case-2-item-2}, the equlibrium $v_*=0$ is asympotitally stable (node)
if $\lambda<0$; unstable (node) if $\lambda>0$. Reference fig. 11.7.

\subsubsection{Example 1}
\begin{equation}
  A =
  \begin{pmatrix}
    2 & 0 \\ 0 & 2 \\
  \end{pmatrix}
\quad \lambda_1 = \lambda_2 = 2 = \lambda.
\end{equation}

\begin{equation}
  (A-\lambda I)\hat{u}=0 \longrightarrow
  \begin{pmatrix}
    0 & 0 \\ 0 & 0 \\
  \end{pmatrix}
  \begin{pmatrix}
    \hat{x} \\ \hat{y}
  \end{pmatrix} =
  \begin{pmatrix}
    0 \\ 0
  \end{pmatrix}
\end{equation}
This has no pivots, \todo[something free?]

\begin{equation}
  \hat{u} =
  \begin{pmatrix}
    \hat{x} & \hat{y}
  \end{pmatrix} =
  \hat{x}
  \begin{pmatrix}
    1 \\ 0
  \end{pmatrix} +
  \hat{y}
  \begin{pmatrix}
    0 \\ 1
  \end{pmatrix}
\end{equation}
\todo[missed the rest of that blackboard]

\subsubsection{Example 2}
\begin{equation}
  A =
  \begin{pmatrix}
    2 & 0 \\ 1 & 2 \\
  \end{pmatrix}
\quad \lambda_1 = \lambda_2 = 2 = \lambda.
\end{equation}

\begin{equation}
  (A-\lambda I)\hat{u}=0 \longrightarrow
  \begin{pmatrix}
    0 & 0 \\ 1 & 0 \\
  \end{pmatrix}
  \begin{pmatrix}
    \hat{x} \\ \hat{y}
  \end{pmatrix} =
  \begin{pmatrix}
    0 \\ 0
  \end{pmatrix}
\end{equation}

\begin{equation}
  \hat{u} =
  \begin{pmatrix}
    0 & \hat{y}
  \end{pmatrix} =
  \hat{y}
  \begin{pmatrix}
    0 \\ 1
  \end{pmatrix}, \hat{y}\;\text{free}.
\end{equation}

$\exists$ only one independent eigenvector: $\hat{u}=
\begin{pmatrix}
  0 \\ 1
\end{pmatrix}$.

To construct the general solution, we need to find $\hat{w}$.
\begin{equation}
  \begin{aligned}
    (A-\lambda I)\hat{w}&=\hat{u}
  \end{aligned}
\begin{pmatrix}
    0 & 0 \\ 1 & 0 \\
  \end{pmatrix}
  \begin{pmatrix}
    \hat{x} \\ \hat{y}
  \end{pmatrix} =
  \begin{pmatrix}
    0 \\ 1
  \end{pmatrix}
\end{equation}
$\hat{x}=1$, $\hat{y}$ is free.

\begin{equation}
  \hat{w} =
  \begin{pmatrix}
    1 \\ \hat{y}
  \end{pmatrix} =
  \begin{pmatrix}
    1 \\ 0
  \end{pmatrix} +
  \hat{y}\begin{pmatrix}
    0 \\ 1
  \end{pmatrix}
\end{equation} where $\hat{y}$ is free.

\begin{equation}
  \hat{y} =0 \rightarrow \hat{w}=
  \begin{pmatrix}
    1 \\ 0
  \end{pmatrix}.
\end{equation}

\subsection{Case III}
Suppose $A$ has complex eigenvalues $\lambda_1=\alpha+i\beta$,
$\lambda_2=\alpha-i\beta$, $\beta\ne0$ with eigenvectors
$\hat{u_1}=\hat{\gamma}+i\hat{\eta}, \hat{u_2}=\hat{\gamma}-i\hat{\eta}$. Then
the general solution
\begin{equation}
  v(t) = C_1e^{\alpha t}[(\cos\beta t)\hat{\gamma} - \sin\beta t)\hat{\eta}] +
  C_2e^{\alpha t}[(\sin\beta t)\hat{\gamma} - \cos\beta t)\hat{\eta}].
\end{equation}

\begin{itemize}
\item The equilibrium $v_*=9$ is asympotitally stable (spiral) if $\alpha<0$;
  unstable (spiral) if $\alpha>0$. Reference fig. 11.8.
\item The equilibrium $v_*=9$ is neutrally stable (spiral) if $\alpha=0$; all
  non-equilibrium solutions are closed periodic curves about the equilibrium.
  Reference fig. 11.9.
\end{itemize}

\subsubsection{Example}
\begin{equation}
  \od{x}{t} = 4x-5y,\quad \od{y}{t}=2x-2y.
\end{equation}

\begin{equation}
  \begin{pmatrix}
    4 & -5 \\ 2 & -2
  \end{pmatrix}
\end{equation}

\begin{equation}
  \lambda_1=1+i,\quad \lambda_2=1-i
\end{equation}
This implies
\begin{equation}
  \alpha=1,\quad \beta=1
\end{equation}

\begin{equation}
  \begin{aligned}
    (A-\lambda_1I)\hat{u_1} &= 0 \\
    \begin{pmatrix}
      3-i & -5 \\ 2 & -3-i \\
    \end{pmatrix}
    \begin{pmatrix}
      \hat{x} & \hat{y}
    \end{pmatrix} &=
    \begin{pmatrix}
      0 & 0
    \end{pmatrix} \\
    \begin{pmatrix}
      3-i & -5 \\ 0 & 0 \\
    \end{pmatrix}
    \begin{pmatrix}
      \hat{x} & \hat{y}
    \end{pmatrix} &=
    \begin{pmatrix}
      0 & 0
    \end{pmatrix} \\
    (3-i)\hat{x}-5\hat{y}&=0 \\
  \end{aligned}
\end{equation}

\begin{equation}
  \begin{aligned}
    \hat{x} = \frac{5}{3-i},\quad \hat{y}&=\frac{3+i}{3+i}\frac{5}{3-1}\hat{y} \\
    &=\hat{3+i}{2}\hat{y}
  \end{aligned}
\end{equation}

\begin{equation}
  \begin{aligned}
    \hat{u_1} =
    \begin{pmatrix}
      \frac{3+i}{2}\hat{y} \\
      \hat{y} \\
    \end{pmatrix} =
    \hat{y}
    \begin{pmatrix}
      \frac{3+i}{2} \\ 1
    \end{pmatrix}, \hat{y}\;\text{free}.
  \end{aligned}
\end{equation}

Take
\begin{equation}
  \hat{u_1} =
  \begin{pmatrix}
    \frac{3+i}{2} \\ 1
  \end{pmatrix} =
  \begin{pmatrix}
    \frac{3}{2} \\ 1
  \end{pmatrix} +
  i\begin{pmatrix}
    \frac{1}{2} \\ 0
  \end{pmatrix} =
  \hat{\gamma} + i\hat{\eta}
\end{equation}

$\hat{u_2}$ is conjugate of
$\hat{u_1}\rightarrow\hat{u_2}=\hat{\gamma}-i\hat{\eta}$
$$\therefore \alpha=1,\quad \beta=1,\quad \hat{\gamma}=
\begin{pmatrix}
  \frac{3}{2} \\ 1
\end{pmatrix},\quad \hat{\eta} =
\begin{pmatrix}
  \frac{1}{2} \\ 0
\end{pmatrix}
$$

The phase diagram is shown in fig. 11.10.

The general solution

\begin{equation}
  v(t) = C_1e^{t}[(\cos t)
  \begin{pmatrix}
    \frac{3}{2} \\ 1
  \end{pmatrix} - (\sin t)
  \begin{pmatrix}
    \frac{1}{2} \\ 0
  \end{pmatrix}
] + C_2e^{t}[
(\sin t)
  \begin{pmatrix}
    \frac{3}{2} \\ 1
  \end{pmatrix} - (\cos t)
  \begin{pmatrix}
    \frac{1}{2} \\ 0
  \end{pmatrix}
]
\end{equation}
\todo[check this8]
\section{Phase Diagrams For Linear Systems Continued}
Recall

\begin{equation} \text{(IVP)} \quad
  \left\{
  \begin{aligned}
    \od{v}{t}&=Av,\quad t&\ge t_0, \\
    v(t_0) &= u_0 \\
  \end{aligned} \right.
\end{equation}
$$v =
\begin{pmatrix}
  x \\ y
\end{pmatrix}, A =
\begin{pmatrix}
  a & b \\ c & d
\end{pmatrix}
$$
And $$\det A \ne 0$$
\todo[some other condition]



\end{document}
