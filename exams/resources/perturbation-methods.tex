
\item[Perturbation Methods] A perturbed equation is regular if
  $\text{(Eq)}_{\epsilon=0}$ is of same type as $\text{(Eq)}_{\epsilon>0}$.

\item[Regularly Perturbed w/ Algebraic Solutions] \hfill
  \begin{enumerate}
  \item Pick an $\alpha>0$ and let $x(\epsilon) = x_0+\epsilon^{\alpha}x_1 +
    \epsilon^{2\alpha}x_2 + \cdots$.
  \item Substitute into the original equation. Taylor expand nonlinear
    expresions in terms of $\epsilon$. Remember to use the chain and product
    rule and derive in terms of $\epsilon$. E.g.\ for
    $f(\epsilon)={x(\epsilon)}^2$,
    \begin{align*}
      f(\epsilon)
      &= {\left[f(\epsilon)\right]}_{\epsilon=0} + \epsilon {\left[f'(0)\right]}_{\epsilon=0}
        + \frac{\epsilon^2}{2!}{\left[f''(0)\right]}_{\epsilon=0}+\cdots \\
      &= x^2_{0} + \epsilon{[2x(\epsilon)x'(\epsilon)]}_{\epsilon=0} +
        \frac{\epsilon^2}{2}{[2x'(\epsilon)x'(\epsilon)+2x(\epsilon)x''(\epsilon)]}_{\epsilon=0}
        + \cdots \\
      &= x_0^2+\epsilon[2x_0x_1] + \frac{\epsilon^2}{2}[2x_1^2 + 2x_0 2x_2] + \cdots \\
    \end{align*}
  \item Collect powers of $\epsilon$
  \item Solve each independent $\epsilon^n$ equation.
  \end{enumerate}

\item[Regularly Perturbed w/ Differential Solutions] \hfill
  \begin{enumerate}
  \item Pick an $\alpha>0$ and let $x(t,\epsilon) =
    x_0(t)+\epsilon^{\alpha}x_1(t) + \epsilon^{2\alpha}x_2(t) + \cdots$.
  \item Substitute into the original equation (IVP + ICs). Taylor expand
    nonlinear expresions in terms of $\epsilon$. Remember to use the chain and
    product rule and derive in terms of $\epsilon$. E.g.\ for
    $f(t,\epsilon)=x{(t,\epsilon)}^3$
    \begin{align*}
      f(t,\epsilon)
      &= {\left[f(t,\epsilon)\right]}_{\epsilon=0} + \epsilon {\left[f(t,0)\right]}_{\epsilon=0}
        + \frac{\epsilon}{2!}{\left[f''(t,0)\right]}_{\epsilon=0} + \cdots \\
      &= {[x^3(t,\epsilon)]}_{\epsilon=0} + \epsilon {[3{x(t,\epsilon)}^2x'(t,\epsilon)]}_{\epsilon=0} + \cdots \\
      &= x^3_0(t) + \epsilon[3x_0^2(t)x_1(t)] + \cdots \\
    \end{align*}
  \item Collect powers of $\epsilon$
  \item Solve each independent $\epsilon^n$ ODE.\@
  \end{enumerate}

\item[Singularly Perturbed w/ Algebraic Solutions] \hfill
  \begin{enumerate}
  \item Given a singularly perturbed algebraic equation (P), introduce a change
    of variables to make the problem regular, i.e. $z=\epsilon^\alpha x$,
    $x=\epsilon^{-\alpha}z$, $\alpha\ge0$. Substitute into (P) and choose the
    smallest $\alpha$ to make the problem regular. E.g.
    \begin{align*}
      \epsilon x^4-x-1&=0 \\
      \epsilon{(\epsilon^{-\alpha}z)}^4-(\epsilon^{-\alpha}z)-1&=0 \\
      \epsilon^{1-4\alpha}z^4-\epsilon^{\alpha}z-1&=0 \\
      z^4-\epsilon^{3\alpha-1}z-\epsilon^{4\alpha-1}&=0 \\
    \end{align*}
    Choose smallest $\alpha$:
    \begin{align*}
      3\alpha-1&\ge0 \quad&\text{and}\quad 4\alpha-1&\ge0 \\
      \alpha&\ge1/3 \quad&\text{and}\quad \alpha&\ge1/4 \\
    \end{align*}
    $\therefore$ choose $\alpha=1/3$.
  \item Solve like regularly perturbed algebraic equation. May choose
    $\delta=\epsilon^{\beta}$ for some leftover $\epsilon$ terms to clean up the
    solution.
  \end{enumerate}

\item[Boundary Layers] Also known as singularly perturbed with differential
  solutions. Usually ``outer'' layer is away from $\epsilon\ll0$ (i.e.
  $t>\delta\epsilon$), and the ``inner'' layer is close to $\epsilon\ll0$ (i.e.
  $\epsilon\downarrow0$).
  \begin{description}
  \item [Outer Boundary Approximation] \hfill
    \begin{enumerate}
    \item Collect powers of $\epsilon$.
    \item Solve for $y$ and invoke outer boundary condition. This is
      $y_{\text{outer}}$.
    \end{enumerate}
  \item[Inner Boundary Condition (Problematic Layer)] \hfill
    \begin{enumerate}
    \item Introduce a change of variables $t=\epsilon^{\alpha}s$,
      $s=\epsilon^{-\alpha}t$, $t=\epsilon^{\alpha}s$. Note that derivatives
      follow this rule:
      \begin{align*}
        \od{y}{t} &= \od{y}{s}\od{s}{t} \\
        \od[2]{y}{t} &= \od{}{t}\od{y}{t} = \od{}{s}\od{y}{t}\od{s}{t}
      \end{align*}
      e.g.
      \begin{align*}
        y'(x) &= \epsilon^{-\alpha}y'(s) \\
        y''(x) &= \epsilon^{-2\alpha}y''(s) \\
      \end{align*}
    \item Remove the $\epsilon$ coefficient on the highest derivative; find the
      minimum $\alpha$ that allows the $\epsilon$ powers to be $\ge1$.
    \item Collect powers of $\epsilon$.
    \item Solve for $y$ and invoke inner boundary condition. This is
      $y_{\text{inner}}$. You should have one remaining constant.
    \end{enumerate}
  \item[Matching] \hfill
    \begin{enumerate}
    \item set up
      $\lim_{x\rightarrow0}y_{\text{inner}}=\lim_{s\rightarrow\infty}y_{\text{outer}}$
      and solve for remaining constant.
    \item The uniform approximation is $y(x)=y_{\text{inner}}(x) +
      y_{\text{outer}}(x)-C$, where $C$ is the common limit.
    \end{enumerate}

    {\tiny Use singular perturbation methods to obtain a uniform approximate
      solution. Assume boundary layer is on left side of interval and find
      leading-order composite approximation.
      \begin{equation}
        \label{eq:3.3.1g-problem}
        \epsilon y''+2y'+e^y=0,\quad y(0)=y(1)=0.
      \end{equation}
      First we'll obtain the outer approximation. We'll collect coefficients of
      $\epsilon^0$ and solve the resulting ODE.\@
      \begin{equation*}
        \begin{aligned}
          2y'+e^y&=0 \\
          2\od{y}{x}+e^y&=0 \\
          2\od{y}{x} &= -e^y \\
          \int -e^{-y} \dd{y} &= \int\frac{1}{2} \dd{x} \\
          e^{-y} &= \frac{x}{2}+c \\
          \ln e^{-y} &= \ln\left( \frac{x}{2}+c \right) \\
          -y &= \ln\left( \frac{x}{2}+c \right) \\
          \implies y &= -\ln\left( \frac{x}{2}+c \right).
        \end{aligned}
      \end{equation*}
      Invoking the initial condition on the outer boundary resolves $c$:
      \begin{equation*}
        \begin{aligned}
          y(1) &= \ln\left( \frac{1}{2}+c \right) \\
          0 &= \ln\left( \frac{1}{2}+c \right) \\
          \implies c &= \frac{1}{2}.
        \end{aligned}
      \end{equation*}
      So the outer boundary approximation is
      \begin{equation*}
        \boxed{y_0(x) = -\ln\left( \frac{x+1}{2} \right).}
      \end{equation*}
      To resolve the inner (boundary) layer we will introduce a change of variables.
      Let
      \begin{equation*}
        x=\epsilon^{\alpha}s,\quad s=\epsilon^{-\alpha}x
      \end{equation*}
      so that \cref{eq:3.3.1g-problem} becomes
      \begin{equation*}
        \epsilon^{-2\alpha+1}y''(s) + 2\epsilon^{-\alpha}y'(s) + e^y = 0.
      \end{equation*}
      Multiplying both sides by $\epsilon^{2\alpha-1}$ yields
      \begin{equation}
        \label{eq:3.3.1g-ys-alpha}
        y''(s) + 2\epsilon^{\alpha-1}y'(s) + \epsilon^{2\alpha-1}e^y=0
      \end{equation}
      so that $\alpha$ must conform to
      \begin{equation}
        \label{eq:3.3.1g-alpha-conditions}
        \begin{aligned}
          \alpha-1&\ge0, &\quad 2\alpha-1&\ge0 \\
          \implies \alpha&\ge1, &\quad \implies \alpha&\ge1/2. \\
        \end{aligned}
      \end{equation}
      The minimum $\alpha$ value which conforms to the
      \cref{eq:3.3.1g-alpha-conditions} is $\alpha=1$. Thus \cref{eq:3.3.1g-ys-alpha}
      becomes
      \begin{equation*}
        \label{eq:3.3.1g-ys}
        y''(s) + 2y + \epsilon e^y=0.
      \end{equation*}
      Collecting coefficients of $\epsilon^0$ yeilds a linear second-order ODE\@
      \begin{equation*}
        y''(s) + 2y'(s) = 0
      \end{equation*}
      whose characteristic equation of
      \begin{equation*}
        r^2+2r=0
      \end{equation*}
      yields a solution in the form
      \begin{equation*}
        y(s) = c_1 + c_2e^{-2s}
      \end{equation*}
      Invoking the initial condition in the boundary (inner) layer resolves $c_2$:
      \begin{equation*}
        \begin{aligned}
          y(0) &= c_1+c_2 \\
          \implies c_2 &= -c_1.
        \end{aligned}
      \end{equation*}
      So that the boundary (inner) approximation becomes
      \begin{equation*}
        \boxed{y_i(s) = c_1(1-e^{-2s})}
      \end{equation*}
      Now we'll perform the matching step to resolve $c_1$.
      \begin{equation*}
        \begin{aligned}
          \lim_{x\rightarrow0}-\ln\left(\frac{x+1}{2}\right) &= \lim_{s\rightarrow\infty}c_1(1-e^{-2s}) \\
          -\ln\left(\frac{1}{2}\right) &= c_1
          \implies c_1 &= -\ln\left(\frac{1}{2}\right)
        \end{aligned}
      \end{equation*}
      Thus the uniform approximation in the form of $y(x)=y_0(x)+y_i(x)-C$, where $C$
      is the common limit, is
      \begin{equation*}
        \boxed{y(x)=-\ln\left( \frac{x+1}{2} \right) -
          \ln\left(\frac{1}{2}\right)(1-e^{-2x/\epsilon}) +
          \ln\left( \frac{1}{2} \right).}
      \end{equation*}
    }

  \end{description}

\item[Taylor Series]
  {\tiny
    \begin{align*}
      f(x) &= f(a)+f'(a)(x-a)+\frac{f''(a)}{2!}{(x-a)}^2+\frac{f^{(3)}(a)}{3!}{(x-a)}^3+\cdots \\
      \frac{1}{1+f(x)} &= \frac{1}{{f(0)}+1}-\frac{x f'(0)}{{(f(0)+1)}^2}+x^2
                         \left(\frac{{f'(0)}^2}{{(f(0)+1)}^3}-\frac{f''(0)}{2 {(f(0)+1)}^2}\right)- {\mathcal{O}(x)}^3 \\
      \frac{1}{1+x f(x)} &= 1 + x^2 \left({f(0)}^2-f'(0)\right)+x^3
                           \left(-\frac{f''(0)}{2}+2 f(0) f'(0)-{f(0)}^3\right)-f(0) x + {\mathcal{O}(x)}^4
    \end{align*}
  }
\item[Poincar\'e-Lindstedt Method] \hfill
  \begin{enumerate}
  \item Introduce a distorted time scale in the perturbation series. Let
    $y(t,\epsilon)=y_0(s) + \epsilon y_1(s) + \cdots$ where $s=\omega t$ with
    $\omega(\epsilon) = \omega_0 + \epsilon\omega_1 + \cdots$. Usually we let
    $\omega_0=1$.
  \item Let $\od{y}{t}=\od{y}{s}\od{s}{t}=\left[\od{y_0}{s} +
      \epsilon\od{y_1}{s} + \cdots\right] \omega(\epsilon)$ and
    $\od[2]{y}{t}=\left[\od[2]{y_0}{s} +
      \epsilon\od[2]{y_1}{s}\right]{\omega(\epsilon)}^2$. Refactor the ODE and
    ICs with this relation.
  \item Taylor expand each nonlinear expression. E.g.
    \begin{align*}
      \left[ \od[2]{y_0}{s}+\epsilon\od[2]{y_1}{s} \right]{\omega(\epsilon)}^2
      &= f(s,0)+\epsilon\od{f}{\epsilon}(s,0) +
        \frac{\epsilon^2}{2}\od[2]{f}{\epsilon}(s,0) \\
      &= \omega_0^2 \od[2]{y_0}{s} + \epsilon\left(2\omega_0\omega_1\od[2]{y_0}{s} +
        \omega_0^2\od[2]{y_1}{s}\right) \
    \end{align*}
  \item Collect powers of $\epsilon$
  \item Solve each $\epsilon^n$ equation. Choose $\omega_m$ to ensure no secular
    terms.
  \end{enumerate}
